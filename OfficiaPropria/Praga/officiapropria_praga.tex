% LuaLaTeX

\documentclass[a4paper, twoside, 10pt]{article}

\usepackage[latin]{babel} 
\usepackage{ecclesiastic}

\usepackage{geometry}
%\geometry{papersize={7.444in,9.681in},total={4.8in,6.8in}}
\geometry{papersize={10.5cm,17.3cm},left=1cm,right=1cm,top=1.2cm,bottom=1.2cm}

\usepackage{fontspec}
\setmainfont[Ligatures={Common, TeX}]{Junicode}

\usepackage{multicol}
\usepackage{color}
\usepackage{lettrine}
\usepackage{fancyhdr}

\newcommand{\rubricatum}[1]{\textcolor{red}{#1}}

% kalendarium
\newenvironment{kalendarium}{\setlength{\parindent}{0cm}}{}
\newcommand{\kMensis}[1]{
  \vspace{2mm}
  \hfill {\large\rubricatum{#1}} \hfill
  \vspace{1mm}}
\newcommand{\kDies}[3]{#1 \hspace{2mm} #2 \rubricatum{#3}

}

\begin{document}

\pagestyle{empty}

\begin{center}
{\Huge 
  Officia propria

  Provinciæ Pragensis}

{\Large 
  a S. Sede approbata et indulta}

\vfill

tomus alter

\vfill

Editio Sancti Wolfgangi
\end{center}

\cleardoublepage

\section*{Kalendarium perpetuum\\Provinciæ Pragensis}

\begin{kalendarium}
% TODO - 1962 reformed feast ranks
% TODO - commemorationes?

% 1928, aestiva, without commemorations and suppressed octaves
\kMensis{Majus}

\kDies{17}{S. Paschalis Baylon Conf.}{duplex}
\kDies{18}{S. Venantii Mart.}{duplex}
\kDies{19}{S. Petri Cælestini Papæ et Conf.}{duplex}
\kDies{20}{S. Bernardini Senensis Conf.}{semiduplex}
\kDies{25}{S. Gregorii VII. Papæ et Conf.}{duplex}
\kDies{26}{S. Philippi Nerii Conf.}{duplex}
\kDies{27}{S. Bedæ Venerabilis Conf. et Eccl. Doct.}{duplex}

\end{kalendarium}


\end{document}
