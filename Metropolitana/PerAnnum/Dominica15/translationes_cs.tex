%%%% Preklady jednotlivych zpevu (nektere se opakuji, a je dobre mit je
% vsechny na jedne hromade)

\newcommand{\trIntroductio}{\translatioCantus{Bože pospěš mi na pomoc.
\Rbardot{} Slyš naše volání. Sláva Otci i Synu i Duchu svatému,
jako byla na počátku, i nyní i vždycky a na věky věků. Amen. Aleluja.}}

\newcommand{\trAntI}{\translatioCantus{Ať je jméno Páně \grestar{}
požehnáno na věky.}}

\newcommand{\trAntII}{\translatioCantus{Vezmu kalich spásy a~budu vzývat
jméno Hospodinovo.}}

\newcommand{\trAntIII}{\translatioCantus{Pán Ježíš se ponížil,
proto také ho Bůh povýšil na věky.}}

\newcommand{\trResp}{\translatioCantus{Kterak jsou skvělá, \grestar{}
Pane, tvá díla. \Vbardot{} Všechno jsi stvořil a~učinil moudře.
\Vbardot{} Sláva Otci i~Synu i~Duchu Svatému.}}

\newcommand{\trAntMagnificat}{\translatioCantus{Ježíš dal \grestar{}
učedníkům sílu a~moc nade všemi zlými duchy, k~léčení nemocí a~k hlásání
Božího slova, aleluja.}}

\newcommand{\trLectioI}{Z druhé knihy Makabejské\\
Nedlouho potom poslal král starého Athéňana, aby přinutil židy odvrátit se
od zákonů otců a~nežít podle Božích zákonů, zneuctít dokonce jeruzalémskou
svatyni a~pojmenovat ji po Diovi Olympském, svatyni na Gerizímu pak
pojmenovat po Diovi Pohostinném proto, že byli pohostinní obyvatelé toho
místa.\\
Těžký a~odporný byl všem nápor tolikerého zla. Vždyť i~svatyni naplnili
pohané pro\-sto\-páš\-nos\-tí a~nevázanou bujností, hledali povyražení u~nevěstek
a~na svatých nádvořích se pelešili s~ženštinami; dovnitř vnášeli, co tam
nepatřilo. I~zápalný oltář byl přeplněn nečistými věcmi, které zákony
zakazovaly. Nebylo dovoleno světit sobotu ani dodržovat svátky otců, ba
vůbec se přiznat k~tomu, že je někdo žid. Každý měsíc násilně nahnali lid na
obětní slavnost králových narozenin, a~když nadešla slavnost Dionýsova,
nutili židy, aby s~břečťanovými věnci oslavovali Dionýsa.\\
Z~popudu Ptolemaiova přišlo nařízení i~do sousedních helénských měst, aby se
s~židy jednalo stejným způsobem, kdykoli se budou pořádat obětní slavnosti;
ti, kdo se nerozhodnou přejít na helénský způsob života, měli být zabiti.
Všem bylo zřejmé, že nastává čas velikého utrpení.\\
Dvě ženy byly předvedeny před soud, že daly obřezat své syny. S~kojenci
u~prsů je vodili veřejně městem a~pak je svrhli z~hradeb. Židé se sešli do
blízkých jeskyní, aby tajně oslavili sobotu. Byli udáni Filipovi a~společně
zaživa upáleni, protože se z~úcty k~nejposvátnějšímu dni odmítli bránit.
Nyní vybízím ty, kdo při četbě knihy došli až sem, aby se nedali zkrušit
tím, co se stalo, ale aby pochopili, že to trestání nebylo ke zkáze, nýbrž
k~výchově našeho rodu. Je známkou velikého dobrodiní bezbožníkům, nejsou-li
dlouhý čas ponecháni v~pohodlí hříchu, ale postihne-li je vzápětí trest.
Zatímco u~jiných národů Hospodin trpělivě čeká, až se naplní míra jejich
hříchů, a~teprve pak je potrestá, s~námi se rozhodl naložit jinak, aby nás
nemusel trestat později, až naše hříchy dosáhnou nejzazší meze. Proto od nás
nikdy neoddaluje milosrdenství; svůj lid vychovává neštěstími, ale neopouští
jej. Jen tolik budiž řečeno na vysvětlenou; po tomto krátkém odbočení
pokračujeme ve vyprávění.\\
Jistému Eleazarovi, jednomu z~předních znalců Zákona, který byl pokročilého
věku a~uš\-lech\-ti\-lých rysů tváře, otevřeli násilím ústa a~nutili ho pozřít
vepřové maso. On však zvolil raději čestnou smrt než život s~potupou
a~dobrovolně šel na popraviště. Pokrm, který měl sníst, vyplivl. Těm, kdo
zůstanou, tím dal příklad, jak odmítat zakázaná jídla i~přes všechnu lásku
k~životu.\\
Dohlížitelé nad bezbožnými obětními slavnostmi, kteří se s~tím mužem znali
ještě z~dří\-věj\-ších dob, ho vzali stranou a~přemlouvali, aby si dal přinést
kusy masa, z~něhož má dovoleno jíst, aby si je sám připravil a~přitom
předstíral, že jí maso z~obětí nařízených králem. Když to prý udělá, unikne
smrti a~pro staré přátelství s~nimi dojde vlídného zacházení. On však učinil
významné rozhodnutí, hodné věku a~důstojnosti stáří, vznešených šedin a~od
mládí nejvzornějšího chování; zvláště pak dbal svatého a~Bohem daného
zákonodárství, když odpověděl, aby jej poslali do podsvětí. „Nesluší se,“
řekl, „abychom se ve svém věku přetvařovali, protože by mnozí z~mladých
soudili, že Eleazar se ve svých devadesáti letech stal odpadlíkem. Ty bych
pro své pokrytectví a~okamžik prchavého ži\-vo\-ta hanebně oklamal a~na své
stáří tak přivodil potupnou skvrnu. A~i kdybych pro přítomnost unikl trestu
lidí, neuniknu rukám Vševládného, ať živý, nebo mrtvý. Proto se nyní
statečně vzdávám života, abych se ukázal hoden svého stáří a~mladým zanechal
ušlechtilý příklad, aby ochotně a~důstojně šli na smrt za vznešené a~svaté
zákony.“ To řekl a~hned vstoupil na popraviště.\\
Blahosklonnost, kterou mu ještě krátce předtím projevili jeho průvodci, se
změnila v~nepřátelství pro ta slova, která pokládali za šílenství. Když
skonával pod ranami, se sténáním pravil: „Hospodinu, který má svaté poznání,
je zjevné, že jsem mohl uniknout smrti; snáším na svém těle krutá muka
bičování a~v duši to rád vytrpím z~posvátné bázně k~němu.“\\
Tak skonal a~zanechal svou smrtí příklad hrdinství a~památku ctnosti nejen
mladým, ale i~většině národa.}

\newcommand{\trPreces}{Bratři a~sestry, jsme shromážděni na místě, kde
požehnaně působil a~krátce před svou smrtí trpěl kněz Josef Toufar.
V~duchovním spojení s ním prosme Boha za naši dnešní církev i~za celý svět
a~společně volejme:\\
\\
\Rbardot{} Kyrie, eleison.\\
\\
{\color{red}--} Prosme za oddanou víru, pevnou naději a~věrnou lásku všech křesťanů. \Rbardot{}\\
{\color{red}--} Prosme za svatost a~obětavost biskupů, kněží a~všech služebníků církve. \Rbardot{}\\
{\color{red}--} Prosme za pokoj a~úctu v rodinách a~dobrou budoucnost příštích generací. \Rbardot{}\\
{\color{red}--} Prosme za vzájemný respekt a~ohleduplnost lidí různých kultur a~různé víry. \Rbardot{}\\
{\color{red}--} Prosme ukončení válečných konfliktů a~nastolení trvalého míru. \Rbardot{}\\
{\color{red}--} Prosme za ty, kdo jsou nuceni uprchnout ze své vlasti. \Rbardot{}\\
{\color{red}--} Prosme za národy utištěné totalitním režimem. \Rbardot{}\\
{\color{red}--} Prosme za sílu a~statečnost těch, kdo trpí násilí. \Rbardot{}\\
{\color{red}--} Prosme za osvobození těch, kdo slouží zlu v jakékoli jeho podobě. \Rbardot{}\\
{\color{red}--} Prosme o~požehnání pro obyvatele této obce, tohoto kraje a~celé naší vlasti. \Rbardot{}}

\newcommand{\trOratio}{Bože, ty vedeš svou církev, aby ze slavného svědectví
mučedníků čerpala sílu a~odvahu; pomáhej nám, ať také my žijeme z~víry
a~v~jejich přímluvě ať máme oporu. Prosíme o~to skrze tvého Syna Ježíše Krista,
našeho Pána, neboť on s~tebou v~jednotě Ducha svatého žije a~kraluje po
všechny věky věků.}
