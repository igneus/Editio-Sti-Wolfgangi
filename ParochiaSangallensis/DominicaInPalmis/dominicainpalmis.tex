% LuaLaTeX

\documentclass[a4paper, twoside, 12pt]{article}
\usepackage[latin]{babel}
%\usepackage[landscape, left=3cm, right=1.5cm, top=2cm, bottom=1cm]{geometry} % okraje stranky
\usepackage[portrait, a4paper, mag=1300, truedimen, left=0.8cm, right=0.8cm, top=0.8cm, bottom=0.8cm]{geometry} % okraje stranky

\usepackage{fontspec}
\setmainfont[FeatureFile={junicode.fea}, Ligatures={Common, TeX}, RawFeature=+fixi]{Junicode}
%\setmainfont{Junicode}

% shortcut for Junicode without ligatures (for the Czech texts)
\newfontfamily\nlfont[FeatureFile={junicode.fea}, Ligatures={Common, TeX}, RawFeature=+fixi]{Junicode}

\usepackage{multicol}
\usepackage{color}
\usepackage{lettrine}
\usepackage{fancyhdr}

% usual packages loading:
\usepackage{luatextra}
\usepackage{graphicx} % support the \includegraphics command and options
\usepackage{gregoriotex} % for gregorio score inclusion
\usepackage{gregoriosyms}
\usepackage{wrapfig} % figures wrapped by the text
\usepackage{parcolumns}
\usepackage[contents={},opacity=1,scale=1,color=black]{background}
\usepackage{tikzpagenodes}
\usepackage{calc}
\usepackage{longtable}

\setlength{\headheight}{12pt}

% Commands used to produce a typical "Conventus" booklet

\newenvironment{titulusOfficii}{\begin{center}}{\end{center}}
\newcommand{\dies}[1]{#1

}
\newcommand{\nomenFesti}[1]{\textbf{\Large #1}

}
\newcommand{\celebratio}[1]{#1

}

\newcommand{\hora}[1]{%
\vspace{0.5cm}{\large \textbf{#1}}

\fancyhead[LE]{\thepage\ / #1}
\fancyhead[RO]{#1 / \thepage}
\addcontentsline{toc}{subsection}{#1}
}

% larger unit than a hora
\newcommand{\divisio}[1]{%
\begin{center}
{\Large \textsc{#1}}
\end{center}
\fancyhead[CO,CE]{#1}
\addcontentsline{toc}{section}{#1}
}

% rubricated inline text
\newcommand{\rubricatum}[1]{\textit{#1}}

% standalone rubric
\newcommand{\rubrica}[1]{\vspace{3mm}\rubricatum{#1}}

\newcommand{\notitia}[1]{\textcolor{red}{#1}}

\newcommand{\scriptura}[1]{\hfill \small\textit{#1}}

\newcommand{\translatioCantus}[1]{\vspace{1mm}%
{\noindent\footnotesize \nlfont{#1}}}

% pruznejsi varianta nasledujiciho - umoznuje nastavit sirku sloupce
% s prekladem
\newcommand{\psalmusEtTranslatioB}[3]{
  \vspace{0.5cm}
  \begin{parcolumns}[colwidths={2=#3}, nofirstindent=true]{2}
    \colchunk{
      \input{#1}
    }

    \colchunk{
      \vspace{-0.5cm}
      {\footnotesize \nlfont
        a
        \input{#2}
      }
    }
  \end{parcolumns}
}

\newcommand{\psalmusEtTranslatio}[2]{
  \psalmusEtTranslatioB{#1}{#2}{8.5cm}
}

% volne misto nad antifonami, kam si zpevaci dokresli neumy
\newcommand{\hicSuntNeumae}{\vspace{0.5cm}}

% prepinani mista mezi notovymi osnovami: pro neumovane a neneumovane zpevy
\newcommand{\cantusCumNeumis}{
  \setgrefactor{17}
  \global\advance\grespaceabovelines by 5mm%
}
\newcommand{\cantusSineNeumas}{
  \setgrefactor{17}
  \global\advance\grespaceabovelines by -5mm%
}

% znaky k umisteni nad inicialu zpevu
\newcommand{\superInitialam}[1]{\gresetfirstlineaboveinitial{\small {\textbf{#1}}}{\small {\textbf{#1}}}}

% pars officii, i.e. "oratio", ...
\newcommand{\pars}[1]{\textbf{#1}}

\newenvironment{psalmus}{
  \setlength{\parindent}{0pt}
  \setlength{\parskip}{5pt}
}{
  \setlength{\parindent}{10pt}
  \setlength{\parskip}{10pt}
}

%%%% Prejmenovat na latinske:
\newcommand{\nadpisZalmu}[1]{
  \hspace{2cm}\textbf{#1}\vspace{2mm}%
  \nopagebreak%

}

% mode, score, translation
\newcommand{\antiphona}[3]{%
\hicSuntNeumae
\superInitialam{#1}
\includescore{#2}

#3
}
 % Often used macros
%%%% Translations of the proper chants

% HOURS ---

% Translated by Václav Ondráček

\newcommand{\translatioAntI}{\translatioCantus{
Hle v oblacích z nebes přijde Pán s velikou mocí, all.
}}
\newcommand{\translatioAntII}{\translatioCantus{
Město a pevnost naše je Sión, 
Spasitel jej opevní příkopem a zdí. 
Brány otevřete, neboť s námi je Bůh, all.
}}
\newcommand{\translatioAntIII}{\translatioCantus{
Hle Pán se zjeví, to není lež; 
jestliže prodlí, čekej jej, 
vždyť přijde a nebude meškat, all.
}}
\newcommand{\translatioAntIV}{\translatioCantus{
Hory a vrchy před Bohem chválu zpívat budou 
a lesní dříví rukama zatleská, 
neboť Pán a Vládce přijde na věky kralovat, all. all.
}}
\newcommand{\translatioAntV}{\translatioCantus{
Ejhle náš Pán s mocí přijde, by rozzářil oči sluhů svých, all.
}}

\newcommand{\translatioCapituli}{\translatioCantus{}}

\newcommand{\translatioRespVesp}{\translatioCantus{
Ukaž nám Pane své milosrdenství – a uděl nám spasení své.
}}

\newcommand{\translatioRespLaud}{\translatioCantus{
Přijď a vysvoboď nás, Hospodine, Bože silný. – Ukaž svou tvář a budeme spaseni.
}}

\newcommand{\translatioVersus}{\translatioCantus{
Hlas volajícího na poušti: Připravte cestu Páně
Urovnejte jeho cestu.
}}

\newcommand{\translatioAntMagnificatI}{\translatioCantus{
Přijď Pane, navštiv nás v míru, ať srdcem celistvým před tebou zajásáme.
}}

\newcommand{\translatioAntBenedictus}{\translatioCantus{
Když viděl Jan v řetězech Kristovy skutky, poslal svých učedníků dvé říci mu: Ty jsi ten, kdo přichází, nebo jiného čekat máme?
}}

\newcommand{\translatioAntMagnificatII}{\translatioCantus{
Ty jsi ten, kdo přichází, nebo jiného čekat máme? 
Rcete Janovi, co jste viděli: 
Světlo vzchází slepým, mrtví vstávají, chudým se hlásá radostná zvěst, aleluja.
}}

\newcommand{\translatioOrationis}{\translatioCantus{
Probuď Pane naše srdce k přípravě cesty tvého Jednorozeného, 
abychom díky jeho příchodu ti mohli sloužit s očištěnou myslí.
}}

\newcommand{\translatioHymnusVesp}{
Štědrý nebes stvořiteli
Věčný osvítiteli
Kriste vykupiteli
prosby vyslyš věrných milý.

Nad zánikem věku v smrti
smiloval ses, milý choti
a světu spásy lačnému
lék jsi přinesl vinnému.

Když se večer světa chýlí,
vyšels jako ženich milý
z domu matky, panny čestné,
z její komnaty milostné.

Mocí tvojí velmi silnou
kolena se k zemi ohnou
a nebe, země tvorové
se poddávají vůli tvé.

Jenž přicházíš věku soudce, 
prosíme tě světovládce,
uchovej nás v našem čase1 
před úkladnou ranou zhoubce.

Chválu, sílu, čest a slávu,
Bohu Otci, jeho Synu ,
Duchu také rady svaté 
vzdej na věky věků světe.
Amen.
}

\newcommand{\translatioHymnusLaud}{
Jasný hlas k nebi zní
temnoty všechny zahání
a trestá sen, ten prchá, jak
z výšin se Kristův zaskví zrak 

Probuď se mysli strnulá,
hluchotou povstaň zraněná,
nová se hvězda zažíhá,
škody od tebe odnímá.

Shůry přichází Beránek 
polehčit vinným jejich stesk,
v slzách o milost volejme,
smilování si žádejme.

Až pak podruhé zabuší
a svět jeho hrůza zkruší
za hříchy nás nepotrestal
a ochranu svou by nám dal.

Chválu, sílu, čest a slávu,
Bohu Otci, jeho Synu ,
Duchu také rady svaté 
vzdej na věky věků světe.
Amen.
}

% MASS ---

\newcommand{\translatioIntroitus}{\translatioCantus{}}

\newcommand{\translatioGraduale}{\translatioCantus{}}

\newcommand{\translatioAlleluia}{\translatioCantus{}}

\newcommand{\translatioOffertorium}{\translatioCantus{}}

\newcommand{\translatioCommunio}{\translatioCantus{}}
 % Czech translations of the proper texts
\newfontface\GreGall{gregall.ttf}
\newfontface\GreGallModern{SGModern.ttf}
\directlua{dofile('gregall.lua')}
\newcommand{\gregallcharno}[3]{{\directlua{
  tex.sprint(gregallparse_neumes("\luaescapestring{#1}", "\luaescapestring{#2}", \luaescapestring{#3}))
}}}
\def\gregallchar{%
  \begingroup %
    \catcode`\~=12{}%
    \fontsize{8}{8}%
    \color{red}%
    \dogregallchar%
}
\def\dogregallchar#1{
    \gregallcharno{#1}{gregall}{0.8}%
  \endgroup %
}
\def\gregallmodchar{%
  \begingroup %
    \catcode`\~=12{}%
    \fontsize{16}{16}%
    \color{red}%
    \dogregallmodchar%
}
\def\dogregallmodchar#1{
    \gregallcharno{#1}{gregallmod}{1.6}%
  \endgroup %
}


\setlength{\columnsep}{15pt} % prostor mezi sloupci

%%%%%%%%%%%%%%%%%%%%%%%%%%%%%%%%%%%%%%%%%%%%%%%%%%%%%%%%%%%%%%%%%%%%%%%%%%%%%%%%%%%%%%%%%%%%%%%%%%%%%%%%%%%%%
\begin{document}

% Here we set the space around the initial.
% Please report to http://home.gna.org/gregorio/gregoriotex/details for more details and options
\setspaceafterinitial{2.2mm plus 0em minus 0em}
\setspacebeforeinitial{2.2mm plus 0em minus 0em}

% Here we set the initial font. Change 38 if you want a bigger initial.
% Emit the initials in red.
\def\greinitialformat#1{%
{\color{red}\fontsize{38}{38}\selectfont #1}%
}

\renewcommand{\headrulewidth}{0pt} % no horiz. rule at the header
\pagestyle{empty}

\begin{center}
{\large DOMINICA IN PALMIS DE PASSIONE DOMINI}

Commemoratio ingressus Domini in Ierusalem
\end{center}

\rubrica{Dum celebrans accedit, cantatur:}

\pars{Antiphona} \scriptura{Cf. Mt. 21, 9; Mc. 11, 10; \textbf{H175}}

\antiphona{VII a}{temporalia/ant-hosanna.tex}

\trHosanna

\rubrica{Post antiphonam ab omnibus decantatam, cantor, vel cantores,
versum integrum canunt. Intonatio rursus cantatur initio cuiusque
versiculi.}

\pars{Psalmus 117.} \scriptura{Ps. 117, 1.22.23.27.28}

\includescore{temporalia/ps117-initium-vii-a-auto.tex}

\psalmusEtTranslatioT{temporalia/ps117-comb.tex}{6cm}

\rubrica{Omittitur \textnormal{Glória Patri.}}

\vspace{2mm}

\begin{center}
{\large AD PROCESSIONEM}
\end{center}

\vspace{2mm}

\rubrica{Progrediente processione, canuntur a schola et populo cantus sequentes (vel alii apti):}

\pars{Antiphona} \scriptura{\textbf{H175}}

\antiphona{I f}{temporalia/ant-pueriportantes.tex}

\trPueriPortantes

\pars{Psalmus 23.} \scriptura{Ps. 23, 1-10}

\includescore{temporalia/ps23-initium-i-f-auto.tex}

\psalmusEtTranslatioT{temporalia/ps23-comb.tex}{6.5cm}

\rubrica{Omittitur \textnormal{Glória Patri.}}

\vspace{4mm}

\pars{Antiphona} \scriptura{\textbf{H175}}

\antiphona{I f}{temporalia/ant-puerivestimenta.tex}

\trPueriVestimenta

\pars{Psalmus 46.} \scriptura{Ps. 46, 2-10}

\includescore{temporalia/ps46-initium-i-f-auto.tex}

\psalmusEtTranslatioT{temporalia/ps46-comb.tex}{6cm}

\rubrica{Omittitur \textnormal{Glória Patri.}}

\vspace{4mm}

\pars{Hymnus ad Christum Regem}

\rubrica{Chorus:} \scriptura{Theodulphus, episcopus Aurelianensis \olddag{} 821; \textbf{E381}}

\antiphona{I}{temporalia/hym-glorialaus.tex}

\trGloriaLaus \scriptura{Omnes: \textnormal{Glória, laus.} ut supra.}

\rubrica{Chorus:}

\includescore{temporalia/hym-glorialaus1.tex}

\trGloriaLausI \scriptura{Omnes: \textnormal{Glória, laus.} ut supra.}

\rubrica{Chorus:}

\includescore{temporalia/hym-glorialaus2.tex}

\trGloriaLausII \scriptura{Omnes: \textnormal{Glória, laus.} ut supra.}

\rubrica{Chorus:}

\includescore{temporalia/hym-glorialaus3.tex}

\trGloriaLausIII \scriptura{Omnes: \textnormal{Glória, laus.} ut supra.}

\rubrica{Chorus:}

\includescore{temporalia/hym-glorialaus4.tex}

\trGloriaLausIV \scriptura{Omnes: \textnormal{Glória, laus.} ut supra.}

\rubrica{Chorus:}

\includescore{temporalia/hym-glorialaus5.tex}

\trGloriaLausV \scriptura{Omnes: \textnormal{Glória, laus.} ut supra.}

\vfill
\pagebreak

\rubrica{Intrante processione in ecclesiam, cantatur:}

\pars{Responsorium} \scriptura{H172}

\antiphona{II}{temporalia/resp-ingrediente.tex}

\trIngrediente

\vspace{4mm}

\rubrica{Post lectionem II:} \scriptura{Phil. 2, 8 \Vbardot{} ibid. 9; \textbf{C96}}

\antiphona{GR. V}{temporalia/graduale-christus.tex}

\trChristus

\pars{Credo I.} \scriptura{XI. s.}

\superInitialam{IV}
\includescore{temporalia/credo-i.tex}

\trCredo
  
\vspace{4mm}

\pars{Antiphona} \scriptura{Cf. Ps. 73, 22; \textbf{H174}}

\antiphona{{\scriptsize{VIII G}}}{temporalia/ant-iudica.tex}

\trIudica

\pars{Psalmus 21.} \scriptura{Ps. 21, 2.3.21.22}

\includescore{temporalia/ps21-initium-viii-G-auto.tex}

\psalmusEtTranslatioT{temporalia/ps21-comb.tex}{6cm}

\vspace{4mm}

\pars{Sanctus XVII.} \scriptura{XI. s.}

\superInitialam{V}
\includescore{temporalia/xvii-sanctus.tex}

\vspace{4mm}

\pars{Oratio Dominica}

\includescore{temporalia/oratio-dominica.tex}

\trOratioDominica

\vspace{4mm}
\pars{Agnus Dei XVII.} \scriptura{XIII. s.}

\superInitialam{V}
\includescore{temporalia/xvii-agnusdei.tex}

\vspace{2mm}

\pars{Antiphona ad Communio} \scriptura{Mt. 26, 42; \textbf{E186}}

\antiphona{VIII G}{temporalia/communio-pater.tex}

\trPater

\pars{Psalmus 115.} \scriptura{Ps. 115, 1.3.4.5.6.7.8}

\includescore{temporalia/ps115-initium-viii-g-auto.tex}

\psalmusEtTranslatioT{temporalia/ps115-comb.tex}{6cm}

\vspace{2mm}

\pars{Attende Domine}

\antiphona{{\scriptsize{V}}}{temporalia/ant-attende.tex}

\trAttende \scriptura{Omnes: \textnormal{\Rbardot{} Atténde.} ut supra.}

\includescore{temporalia/vers-adterex.tex}

\psalmusEtTranslatioT{attende-comb.tex}{6cm}

\end{document}
