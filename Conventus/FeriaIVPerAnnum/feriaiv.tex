% LuaLaTeX

\documentclass[a4paper, twoside, 12pt]{article}
\usepackage[latin]{babel}
%\usepackage[landscape, left=3cm, right=1.5cm, top=2cm, bottom=1cm]{geometry} % okraje stranky
\usepackage[landscape, a4paper, mag=1166, truedimen, left=2cm, right=1.5cm, top=1.6cm, bottom=0.95cm]{geometry} % okraje stranky

\usepackage{fontspec}
\setmainfont[FeatureFile={junicode.fea}, Ligatures={Common, TeX}, RawFeature=+fixi]{Junicode}
%\setmainfont{Junicode}

% shortcut for Junicode without ligatures (for the Czech texts)
\newfontfamily\nlfont[FeatureFile={junicode.fea}, Ligatures={Common, TeX}, RawFeature=+fixi]{Junicode}

\usepackage{multicol}
\usepackage{color}
\usepackage{lettrine}
\usepackage{fancyhdr}

% usual packages loading:
\usepackage{luatextra}
\usepackage{graphicx} % support the \includegraphics command and options
\usepackage{gregoriotex} % for gregorio score inclusion
\usepackage{gregoriosyms}
\usepackage{wrapfig} % figures wrapped by the text
\usepackage{parcolumns}
\usepackage[contents={},opacity=1,scale=1,color=black]{background}
\usepackage{tikzpagenodes}
\usepackage{calc}
\usepackage{longtable}
\usetikzlibrary{calc}

\setlength{\headheight}{14.5pt}

% Commands used to produce a typical "Conventus" booklet

\newenvironment{titulusOfficii}{\begin{center}}{\end{center}}
\newcommand{\dies}[1]{#1

}
\newcommand{\nomenFesti}[1]{\textbf{\Large #1}

}
\newcommand{\celebratio}[1]{#1

}

\newcommand{\hora}[1]{%
\vspace{0.5cm}{\large \textbf{#1}}

\fancyhead[LE]{\thepage\ / #1}
\fancyhead[RO]{#1 / \thepage}
\addcontentsline{toc}{subsection}{#1}
}

% larger unit than a hora
\newcommand{\divisio}[1]{%
\begin{center}
{\Large \textsc{#1}}
\end{center}
\fancyhead[CO,CE]{#1}
\addcontentsline{toc}{section}{#1}
}

% rubricated inline text
\newcommand{\rubricatum}[1]{\textit{#1}}

% standalone rubric
\newcommand{\rubrica}[1]{\vspace{3mm}\rubricatum{#1}}

\newcommand{\notitia}[1]{\textcolor{red}{#1}}

\newcommand{\scriptura}[1]{\hfill \small\textit{#1}}

\newcommand{\translatioCantus}[1]{\vspace{1mm}%
{\noindent\footnotesize \nlfont{#1}}}

% pruznejsi varianta nasledujiciho - umoznuje nastavit sirku sloupce
% s prekladem
\newcommand{\psalmusEtTranslatioB}[3]{
  \vspace{0.5cm}
  \begin{parcolumns}[colwidths={2=#3}, nofirstindent=true]{2}
    \colchunk{
      \input{#1}
    }

    \colchunk{
      \vspace{-0.5cm}
      {\footnotesize \nlfont
        a
        \input{#2}
      }
    }
  \end{parcolumns}
}

\newcommand{\psalmusEtTranslatio}[2]{
  \psalmusEtTranslatioB{#1}{#2}{8.5cm}
}

% volne misto nad antifonami, kam si zpevaci dokresli neumy
\newcommand{\hicSuntNeumae}{\vspace{0.5cm}}

% prepinani mista mezi notovymi osnovami: pro neumovane a neneumovane zpevy
\newcommand{\cantusCumNeumis}{
  \setgrefactor{17}
  \global\advance\grespaceabovelines by 5mm%
}
\newcommand{\cantusSineNeumas}{
  \setgrefactor{17}
  \global\advance\grespaceabovelines by -5mm%
}

% znaky k umisteni nad inicialu zpevu
\newcommand{\superInitialam}[1]{\gresetfirstlineaboveinitial{\small {\textbf{#1}}}{\small {\textbf{#1}}}}

% pars officii, i.e. "oratio", ...
\newcommand{\pars}[1]{\textbf{#1}}

\newenvironment{psalmus}{
  \setlength{\parindent}{0pt}
  \setlength{\parskip}{5pt}
}{
  \setlength{\parindent}{10pt}
  \setlength{\parskip}{10pt}
}

%%%% Prejmenovat na latinske:
\newcommand{\nadpisZalmu}[1]{
  \hspace{2cm}\textbf{#1}\vspace{2mm}%
  \nopagebreak%

}

% mode, score, translation
\newcommand{\antiphona}[3]{%
\hicSuntNeumae
\superInitialam{#1}
\includescore{#2}

#3
}
 % Often used macros
%%%% Translations of the proper chants

% HOURS ---

% Translated by Václav Ondráček

\newcommand{\translatioAntI}{\translatioCantus{
Hle v oblacích z nebes přijde Pán s velikou mocí, all.
}}
\newcommand{\translatioAntII}{\translatioCantus{
Město a pevnost naše je Sión, 
Spasitel jej opevní příkopem a zdí. 
Brány otevřete, neboť s námi je Bůh, all.
}}
\newcommand{\translatioAntIII}{\translatioCantus{
Hle Pán se zjeví, to není lež; 
jestliže prodlí, čekej jej, 
vždyť přijde a nebude meškat, all.
}}
\newcommand{\translatioAntIV}{\translatioCantus{
Hory a vrchy před Bohem chválu zpívat budou 
a lesní dříví rukama zatleská, 
neboť Pán a Vládce přijde na věky kralovat, all. all.
}}
\newcommand{\translatioAntV}{\translatioCantus{
Ejhle náš Pán s mocí přijde, by rozzářil oči sluhů svých, all.
}}

\newcommand{\translatioCapituli}{\translatioCantus{}}

\newcommand{\translatioRespVesp}{\translatioCantus{
Ukaž nám Pane své milosrdenství – a uděl nám spasení své.
}}

\newcommand{\translatioRespLaud}{\translatioCantus{
Přijď a vysvoboď nás, Hospodine, Bože silný. – Ukaž svou tvář a budeme spaseni.
}}

\newcommand{\translatioVersus}{\translatioCantus{
Hlas volajícího na poušti: Připravte cestu Páně
Urovnejte jeho cestu.
}}

\newcommand{\translatioAntMagnificatI}{\translatioCantus{
Přijď Pane, navštiv nás v míru, ať srdcem celistvým před tebou zajásáme.
}}

\newcommand{\translatioAntBenedictus}{\translatioCantus{
Když viděl Jan v řetězech Kristovy skutky, poslal svých učedníků dvé říci mu: Ty jsi ten, kdo přichází, nebo jiného čekat máme?
}}

\newcommand{\translatioAntMagnificatII}{\translatioCantus{
Ty jsi ten, kdo přichází, nebo jiného čekat máme? 
Rcete Janovi, co jste viděli: 
Světlo vzchází slepým, mrtví vstávají, chudým se hlásá radostná zvěst, aleluja.
}}

\newcommand{\translatioOrationis}{\translatioCantus{
Probuď Pane naše srdce k přípravě cesty tvého Jednorozeného, 
abychom díky jeho příchodu ti mohli sloužit s očištěnou myslí.
}}

\newcommand{\translatioHymnusVesp}{
Štědrý nebes stvořiteli
Věčný osvítiteli
Kriste vykupiteli
prosby vyslyš věrných milý.

Nad zánikem věku v smrti
smiloval ses, milý choti
a světu spásy lačnému
lék jsi přinesl vinnému.

Když se večer světa chýlí,
vyšels jako ženich milý
z domu matky, panny čestné,
z její komnaty milostné.

Mocí tvojí velmi silnou
kolena se k zemi ohnou
a nebe, země tvorové
se poddávají vůli tvé.

Jenž přicházíš věku soudce, 
prosíme tě světovládce,
uchovej nás v našem čase1 
před úkladnou ranou zhoubce.

Chválu, sílu, čest a slávu,
Bohu Otci, jeho Synu ,
Duchu také rady svaté 
vzdej na věky věků světe.
Amen.
}

\newcommand{\translatioHymnusLaud}{
Jasný hlas k nebi zní
temnoty všechny zahání
a trestá sen, ten prchá, jak
z výšin se Kristův zaskví zrak 

Probuď se mysli strnulá,
hluchotou povstaň zraněná,
nová se hvězda zažíhá,
škody od tebe odnímá.

Shůry přichází Beránek 
polehčit vinným jejich stesk,
v slzách o milost volejme,
smilování si žádejme.

Až pak podruhé zabuší
a svět jeho hrůza zkruší
za hříchy nás nepotrestal
a ochranu svou by nám dal.

Chválu, sílu, čest a slávu,
Bohu Otci, jeho Synu ,
Duchu také rady svaté 
vzdej na věky věků světe.
Amen.
}

% MASS ---

\newcommand{\translatioIntroitus}{\translatioCantus{}}

\newcommand{\translatioGraduale}{\translatioCantus{}}

\newcommand{\translatioAlleluia}{\translatioCantus{}}

\newcommand{\translatioOffertorium}{\translatioCantus{}}

\newcommand{\translatioCommunio}{\translatioCantus{}}
 % Czech translations of the proper texts
\newfontface\GreGall{gregall.ttf}
\newfontface\GreGallModern{SGModern.ttf}
\directlua{dofile('gregall.lua')}
\newcommand{\gregallcharno}[3]{{\directlua{
  tex.sprint(gregallparse_neumes("\luaescapestring{#1}", "\luaescapestring{#2}", \luaescapestring{#3}))
}}}
\def\gregallchar{%
  \begingroup %
    \catcode`\~=12{}%
    \fontsize{8}{8}%
    \color{red}%
    \dogregallchar%
}
\def\dogregallchar#1{
    \gregallcharno{#1}{gregall}{0.8}%
  \endgroup %
}
\def\gregallmodchar{%
  \begingroup %
    \catcode`\~=12{}%
    \fontsize{16}{16}%
    \color{red}%
    \dogregallmodchar%
}
\def\dogregallmodchar#1{
    \gregallcharno{#1}{gregallmod}{1.6}%
  \endgroup %
}


\newcommand{\annusEditionis}{2015}

%%%% Vicekrat opakovane kousky

\newcommand{\anteOrationem}{
  \rubrica{Ante Orationem, cantatur a Superiore:}

  \pars{Supplicatio Litaniæ.}

  \includescore{temporalia/supplicatiolitaniae.tex}

  \pars{Oratio Dominica.}

  \includescore{temporalia/oratiodominica.tex}

  \rubrica{Deinde dicitur ab Hebdomadario:}

  \includescore{temporalia/dominusvobiscum-solemnis.tex}

  \rubrica{In choro monialium loco Dominus vobiscum dicitur:}

  \includescore{temporalia/domineexaudi.tex}
}

\setlength{\columnsep}{30pt} % prostor mezi sloupci

%%%%%%%%%%%%%%%%%%%%%%%%%%%%%%%%%%%%%%%%%%%%%%%%%%%%%%%%%%%%%%%%%%%%%%%%%%%%%%%%%%%%%%%%%%%%%%%%%%%%%%%%%%%%%
\begin{document}

% Here we set the space around the initial.
% Please report to http://home.gna.org/gregorio/gregoriotex/details for more details and options
\setspaceafterinitial{2.2mm plus 0em minus 0em}
\setspacebeforeinitial{2.2mm plus 0em minus 0em}

% Here we set the initial font. Change 38 if you want a bigger initial.
% Emit the initials in red.
\def\greinitialformat#1{%
{\color{red}\fontsize{38}{38}\selectfont #1}%
}

\pagestyle{empty}

%%%% Titulni stranka
\begin{titulusOfficii}
\nomenFesti{Feria IV per Annum.}
\end{titulusOfficii}

% graphic
\vspace{1.5cm}
\begin{center}
%\includegraphics[width=15cm]{imagines/feriaiv.jpg}
\end{center}

\vfill

\begin{center}
Ad usum et secundum consuetudines chori \guillemotright{}Conventus Choralis\guillemotleft.

Editio Sancti Wolfgangi \annusEditionis
\end{center}

\pagebreak

\renewcommand{\headrulewidth}{0pt} % no horiz. rule at the header
\fancyhf{}
\pagestyle{fancy}

\pars{Oratio ante divinum Officium.}

\lettrine{{\color{red}A}}{peri,} Dómine, os meum ad benedicéndum nomen sanctum tuum:
munda quoque cor meum ab ómnibus vanis, pervérsis, et aliénis
cogitatiónibus:
intelléctum illúmina, afféctum inflámma,
ut digne, atténte ac devóte hoc Offícium recitáre váleam,
et exaudíri mérear ante conspéctum Divínæ Majestátis tuæ.
Per Christum, Dominum nostrum.
\Rbardot{} Amen.

Dómine, in unióne illíus divínæ intentiónis,
qua ipse in terris laudes Deo persolvísti,
has tibi Horas \rubricatum{(vel \textnormal{hanc tibi Horam})} persólvo.

\trOratioAnteOfficium

\vfill

\pars{Oratio post divinum Officium.}

\rubrica{
  Orationem sequentem devote post Officium recitantibus
  Leo Papa X. defectus, et culpas in eo persolvendo ex humana
  fragilitate contractas, indulsit, et dicitur flexis genibus.
}

\lettrine{{\color{red}S}}{acrosánctæ} et indivíduæ Trinitáti,
crucifíxi Dómini nostri Jesu Christi humanitáti,
beatíssimæ et gloriosíssimæ sempérque Vírginis Maríæ
fecúndæ integritáti, 
et ómnium Sanctórum universitáti
sit sempitérna laus, honor, virtus et glória
ab omni creatúra,
nobísque remíssio ómnium peccatórum,
per infiníta sǽcula sæculórum.
\Rbardot{} Amen.

\noindent \Vbardot{} Beáta víscera Maríæ Virginis, quæ portavérunt
ætérni Patris Fílium.\\
\Rbardot{} Et beáta úbera, quæ lactavérunt Christum Dominum.

\rubrica{Et dicitur secreto \textnormal{Pater noster.} et \textnormal{Ave María.}}

\trOratioPostOfficium

\vfill

\hora{Ad Laudes.} %%%%%%%%%%%%%%%%%%%%%%%%%%%%%%%%%%%%%%%%%%%%%%%%%%%%%
\sideThumbs{Laudes}

\cantusSineNeumas

% Psalmi festivi (AM33, pg. 52):
% 50, 63, 64, Cant. Annae, 148+149+150

\vspace{1cm}
\includescore{temporalia/deusinadiutorium-communis.tex}
\vspace{1cm}

\vfill
\pagebreak

\pars{Psalmus 1.}

\scriptura{Psalmus 66.}

\includescore{temporalia/ps66-initium-dir-auto.tex}

\psalmusEtTranslatioT{temporalia/ps66-comb.tex}{10cm}

\vfill
\pagebreak

\cantusSineNeumas

\pars{Psalmus 2.} \scriptura{Psalmus 50, 4; \textbf{H95}}

\antiphona{VII a}{temporalia/ant1.tex}

\trAntI

\scriptura{Psalmus 50.}

\includescore{temporalia/ps50-initium-vii-a-auto.tex}

\psalmusEtTranslatioT{temporalia/ps50-comb.tex}{10cm}

\includescore{temporalia/ant1.tex} % repeat the antiphon - new page

\vfill
\pagebreak

\pars{Psalmus 3.} \scriptura{Psalmus 63, 2; \textbf{H96}}

\antiphona{II D}{temporalia/ant2.tex}

\trAntII

\scriptura{Psalmus 63.}

\includescore{temporalia/ps63-initium-ii-D-auto.tex}

\psalmusEtTranslatioT{temporalia/ps63-comb.tex}{10cm}

\vfill
\pagebreak

\pars{Psalmus 4.} \scriptura{Psalmus 64, 2; \textbf{H96}}

\antiphona{VIII C}{temporalia/ant3.tex}

\trAntIII

\scriptura{Psalmus 64.}

\includescore{temporalia/ps64-initium-viii-C-auto.tex}

\psalmusEtTranslatioT{temporalia/ps64-comb.tex}{10cm}

\includescore{temporalia/ant3.tex} % repeat the antiphon - new page

\vfill
\pagebreak

\pars{Psalmus 5.} \scriptura{1 Sam. 2, 10; \textbf{H96}}

\antiphona{I g2}{temporalia/ant4.tex}

\trAntIV

\scriptura{Canticum Annæ, 1 Sam. 2, 1-10}

\includescore{temporalia/anna-initium-i-g2-auto.tex}

\psalmusEtTranslatioT{temporalia/anna-comb.tex}{10cm}

\includescore{temporalia/ant4.tex} % repeat the antiphon - new page

\vfill
\pagebreak

\pars{Psalmus 6.} \scriptura{Psalmus 148, 4; \textbf{H96}}

\antiphona{II D}{temporalia/ant5.tex}

\trAntV

\scriptura{Psalmus 148.}

\includescore{temporalia/ps148-initium-ii-D-auto.tex}

\psalmusEtTranslatioT{temporalia/ps148-comb.tex}{10cm}

\rubrica{Hic non dicitur Gloria Patri.}

\scriptura{Psalmus 149.}

\includescore{temporalia/ps149-initium-ii-D-auto.tex}

\psalmusEtTranslatioT{temporalia/ps149-comb.tex}{10cm}

\rubrica{Hic non dicitur Gloria Patri.}

\vfill
\pagebreak

\scriptura{Psalmus 150.}

\includescore{temporalia/ps150-initium-ii-D-auto.tex}

\psalmusEtTranslatioT{temporalia/ps150-comb.tex}{10cm}

\includescore{temporalia/ant5.tex} % repeat the antiphon - new page

\vfill
\pagebreak

\pars{Capitulum.} \scriptura{Rom. 13, 12 - 13}

\includescore{temporalia/capitulum-Nox.tex}

% preklad Jeruz. bible
\trCapituli

\vspace{1cm}
\pars{Responsorium breve.}

\superInitialam{VI}
\includescore{temporalia/respbr.tex}

\trRespLaudes
\vfill
\pagebreak

\pars{Hymnus, tonus in Hieme.}

\includescore{temporalia/hym-NoxEtTenebraeHieme.tex}
\begin{translatioMulticol}{3}
{\color{red}\textit{1.}} Noci a mlžné temnoty,\\
zmatená světa nestálost,\\
světlem jasní se báň nebe\\
Kristus je tu! Vy ustupte!\\
\\
{\color{red}\textit{2.}} Pozemská temnost se tříští\\
ostrými Slunce paprsky\\
věcem již barvy navrací\\
zářící denní hvězdy tvář.\columnbreak

{\color{red}\textit{3.}} Jen Tebe, Kriste, chceme znát\\
myslí čistou a upřímnou:\\
V nářku i zpěvu Tě prosíme:\\
Buď našich smyslů posilou.\\
\\
{\color{red}\textit{4.}} Mnohé je na náš líčené -\\
to vše tvé světlo očistí.\\
Ty pravé světlo nebeské\\
jasnou nám svou tvář rozjasni.\columnbreak

{\color{red}\textit{5.}} Bohu Otci vždy sláva buď\\
i Synu jeho jednomu\\
s Duchem veškeré útěchy\\
nyní i vždy i navěky.\\
Amen.
\end{translatioMulticol}


\vfill

\pars{Versus.} \scriptura{Psalmus 89, 14}

% Versus. %%%
\includescore{temporalia/versus-repleti.tex}

\noindent \trVersusLaudes

\vfill
\pagebreak

\pars{Hymnus, tonus in Æstate.}

\includescore{temporalia/hym-NoxEtTenebraeAEstate.tex}
\begin{translatioMulticol}{3}
{\color{red}\textit{1.}} Noci a mlžné temnoty,\\
zmatená světa nestálost,\\
světlem jasní se báň nebe\\
Kristus je tu! Vy ustupte!\\
\\
{\color{red}\textit{2.}} Pozemská temnost se tříští\\
ostrými Slunce paprsky\\
věcem již barvy navrací\\
zářící denní hvězdy tvář.\columnbreak

{\color{red}\textit{3.}} Jen Tebe, Kriste, chceme znát\\
myslí čistou a upřímnou:\\
V nářku i zpěvu Tě prosíme:\\
Buď našich smyslů posilou.\\
\\
{\color{red}\textit{4.}} Mnohé je na náš líčené -\\
to vše tvé světlo očistí.\\
Ty pravé světlo nebeské\\
jasnou nám svou tvář rozjasni.\columnbreak

{\color{red}\textit{5.}} Bohu Otci vždy sláva buď\\
i Synu jeho jednomu\\
s Duchem veškeré útěchy\\
nyní i vždy i navěky.\\
Amen.
\end{translatioMulticol}


\vfill

\pars{Versus.} \scriptura{Psalmus 89, 14}

% Versus. %%%
\includescore{temporalia/versus-repleti.tex}

\noindent \trVersusLaudes

\vfill
\pagebreak

%\cantusCumNeumis

\pars{Canticum Zachariæ.} \scriptura{Lucæ 1, 71; \textbf{H423}}

\antiphona{I g5}{temporalia/ant-ben.tex}

\trAntBenedictus

\scriptura{Lucæ 1, 68-79}

\includescore{temporalia/benedictus-initium-i-g5-auto.tex}

\psalmusEtTranslatioT{temporalia/benedictus-comb.tex}{10cm}

\includescore{temporalia/ant-ben.tex} % repeat the antiphon - new page

\vfill
\pagebreak

\anteOrationem

\pagebreak

% Oratio. %%%
\pars{Oratio.}

\includescore{temporalia/oratio.tex}
\trOrationis

\vspace{1cm}
\rubrica{Hebdomadarius dicit iterum Dominus vobiscum. Postea cantatur a cantore:}
\vspace{2mm}

\includescore{temporalia/benedicamus-feria-laudes.tex}

\vfill
\pagebreak

\pars{Antiphona finalis B. M. V. pro tempore per annum.}

\includescore{temporalia/ant-salveregina-simplex.tex}

\vfill
\pagebreak

\newpage
\RemoveSideThumbs
\pagestyle{empty}

%%% COLOPHON
\vfill

Fontes.
Textus et cantus officii divini secundum
Antiphonale Sacrosanctæ Romanæ Eclesiæ Pro Diurnis Horis, Romæ 1912
et Nocturnale Romanum, 2002, præter: psalmi 149 et 150 post
psalmum 148 in Laudibus additi secundum Antiphonale Monasticum pro Diurnis Horis,
Solesmis 1934; lectio sancti Evangelii et hymnus Te Decet Laus post hymnum
Ambrosianum additi secundum ritum monasticum vetum; responsorium breve
in Laudibus et Vesperis additum secundum Antiphonale Monasticum. /
Textus et cantus missæ secundum
Graduale triplex, Solesmis 1979. /
Translatio capituli et lectionis sumpta est ex:
Jeruzalémská bible, Praha-Kostelní Vydří 2009. /
Translationes psalmorum ex
Hejčl Jan: Žaltář čili Kniha žalmů, Praha 1922. /
Neumæ super canto missæ de codicibus Cantatorium, Stiftsbibl. 359 et Einsiedeln,
Stiftsbibl. 121 et neumæ super canto officii divini de codice Hartker,
Stiftsbibl. 390.

Collaborantes.
Textus latinos cantusque transcripsit et omnem laborem typographicum peregit
Jakub Jelínek et Jakub Pavlík. /
Psalmos in lingua bohemica de libro supra dicto transcripsit
Barbora Maturová et idem Jakub Jelínek. /
Václav Ondráček textus hymnorum, antiphonarum, homiliarum, benedictionum etc.
in linguam bohemicam transtulit. /
Nea Marie Kuchařová textus sermonum in linguam bohemicam transtulit. /
Filip Srovnal librum istum præparare mandavit et laborem exprobrationibus
utilissimis comitabatur. /
Štěpán Němec librum istum diligentissime examinavit, errores multos
inveniens. /
Terezie Regnerová imaginem titulum libri ornantem pinxit.

Instrumenta adhibita.
LuaTeX, %http://www.luatex.org /
Gregorio, %http://home.gna.org/gregorio /
typi Junicode. %http://junicode.sourceforge.net

\begin{center}
Liber hic imprimis ad usum chori
\guillemotright Conventus Choralis\guillemotleft\
paratus est
et secundum eius consuetudines.
http://www.introitus.cz

\vfill

{\large Editio Sancti Wolfgangi \annusEditionis.}

\vfill

Series \guillemotright Conventus\guillemotleft, vol. VI.

\vfill

http://stiwolfgangi.xf.cz

\end{center}

\vfill

\end{document}
