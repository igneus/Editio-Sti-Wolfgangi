% LuaLaTeX

\documentclass[a4paper, twoside, 12pt]{article}

\usepackage[latin]{babel} 
\usepackage{ecclesiastic}

\usepackage[landscape, left=3cm, right=1.5cm, top=2cm, bottom=1cm]{geometry}

\usepackage{fontspec}
\setmainfont[Ligatures={Common, TeX}]{Junicode}

% shortcut for Junicode without ligatures (for the Czech texts)
\newfontfamily\nlfont[Ligatures={Common, TeX}]{Junicode}

\usepackage{multicol}
\usepackage{color}
\usepackage{lettrine}
\usepackage{fancyhdr}

% usual packages loading:
\usepackage{luatextra}
\usepackage{graphicx} % support the \includegraphics command and options
\usepackage{gregoriotex} % for gregorio score inclusion
\usepackage{gregoriosyms}
\usepackage{parcolumns}
\usepackage{multicol}

% Commands used to produce a typical "Conventus" booklet

\newenvironment{titulusOfficii}{\begin{center}}{\end{center}}
\newcommand{\dies}[1]{#1

}
\newcommand{\nomenFesti}[1]{\textbf{\Large #1}

}
\newcommand{\celebratio}[1]{#1

}

\newcommand{\hora}[1]{%
\vspace{0.5cm}{\large \textbf{#1}}

\fancyhead[LE]{\thepage\ / #1}
\fancyhead[RO]{#1 / \thepage}
\addcontentsline{toc}{subsection}{#1}
}

% larger unit than a hora
\newcommand{\divisio}[1]{%
\begin{center}
{\Large \textsc{#1}}
\end{center}
\fancyhead[CO,CE]{#1}
\addcontentsline{toc}{section}{#1}
}

% rubricated inline text
\newcommand{\rubricatum}[1]{\textit{#1}}

% standalone rubric
\newcommand{\rubrica}[1]{\vspace{3mm}\rubricatum{#1}}

\newcommand{\notitia}[1]{\textcolor{red}{#1}}

\newcommand{\scriptura}[1]{\hfill \small\textit{#1}}

\newcommand{\translatioCantus}[1]{\vspace{1mm}%
{\noindent\footnotesize \nlfont{#1}}}

% pruznejsi varianta nasledujiciho - umoznuje nastavit sirku sloupce
% s prekladem
\newcommand{\psalmusEtTranslatioB}[3]{
  \vspace{0.5cm}
  \begin{parcolumns}[colwidths={2=#3}, nofirstindent=true]{2}
    \colchunk{
      \input{#1}
    }

    \colchunk{
      \vspace{-0.5cm}
      {\footnotesize \nlfont
        a
        \input{#2}
      }
    }
  \end{parcolumns}
}

\newcommand{\psalmusEtTranslatio}[2]{
  \psalmusEtTranslatioB{#1}{#2}{8.5cm}
}

% volne misto nad antifonami, kam si zpevaci dokresli neumy
\newcommand{\hicSuntNeumae}{\vspace{0.5cm}}

% prepinani mista mezi notovymi osnovami: pro neumovane a neneumovane zpevy
\newcommand{\cantusCumNeumis}{
  \setgrefactor{17}
  \global\advance\grespaceabovelines by 5mm%
}
\newcommand{\cantusSineNeumas}{
  \setgrefactor{17}
  \global\advance\grespaceabovelines by -5mm%
}

% znaky k umisteni nad inicialu zpevu
\newcommand{\superInitialam}[1]{\gresetfirstlineaboveinitial{\small {\textbf{#1}}}{\small {\textbf{#1}}}}

% pars officii, i.e. "oratio", ...
\newcommand{\pars}[1]{\textbf{#1}}

\newenvironment{psalmus}{
  \setlength{\parindent}{0pt}
  \setlength{\parskip}{5pt}
}{
  \setlength{\parindent}{10pt}
  \setlength{\parskip}{10pt}
}

%%%% Prejmenovat na latinske:
\newcommand{\nadpisZalmu}[1]{
  \hspace{2cm}\textbf{#1}\vspace{2mm}%
  \nopagebreak%

}

% mode, score, translation
\newcommand{\antiphona}[3]{%
\hicSuntNeumae
\superInitialam{#1}
\includescore{#2}

#3
}

%%%% Translations of the proper chants

% HOURS ---

% Translated by Václav Ondráček

\newcommand{\translatioAntI}{\translatioCantus{
Hle v oblacích z nebes přijde Pán s velikou mocí, all.
}}
\newcommand{\translatioAntII}{\translatioCantus{
Město a pevnost naše je Sión, 
Spasitel jej opevní příkopem a zdí. 
Brány otevřete, neboť s námi je Bůh, all.
}}
\newcommand{\translatioAntIII}{\translatioCantus{
Hle Pán se zjeví, to není lež; 
jestliže prodlí, čekej jej, 
vždyť přijde a nebude meškat, all.
}}
\newcommand{\translatioAntIV}{\translatioCantus{
Hory a vrchy před Bohem chválu zpívat budou 
a lesní dříví rukama zatleská, 
neboť Pán a Vládce přijde na věky kralovat, all. all.
}}
\newcommand{\translatioAntV}{\translatioCantus{
Ejhle náš Pán s mocí přijde, by rozzářil oči sluhů svých, all.
}}

\newcommand{\translatioCapituli}{\translatioCantus{}}

\newcommand{\translatioRespVesp}{\translatioCantus{
Ukaž nám Pane své milosrdenství – a uděl nám spasení své.
}}

\newcommand{\translatioRespLaud}{\translatioCantus{
Přijď a vysvoboď nás, Hospodine, Bože silný. – Ukaž svou tvář a budeme spaseni.
}}

\newcommand{\translatioVersus}{\translatioCantus{
Hlas volajícího na poušti: Připravte cestu Páně
Urovnejte jeho cestu.
}}

\newcommand{\translatioAntMagnificatI}{\translatioCantus{
Přijď Pane, navštiv nás v míru, ať srdcem celistvým před tebou zajásáme.
}}

\newcommand{\translatioAntBenedictus}{\translatioCantus{
Když viděl Jan v řetězech Kristovy skutky, poslal svých učedníků dvé říci mu: Ty jsi ten, kdo přichází, nebo jiného čekat máme?
}}

\newcommand{\translatioAntMagnificatII}{\translatioCantus{
Ty jsi ten, kdo přichází, nebo jiného čekat máme? 
Rcete Janovi, co jste viděli: 
Světlo vzchází slepým, mrtví vstávají, chudým se hlásá radostná zvěst, aleluja.
}}

\newcommand{\translatioOrationis}{\translatioCantus{
Probuď Pane naše srdce k přípravě cesty tvého Jednorozeného, 
abychom díky jeho příchodu ti mohli sloužit s očištěnou myslí.
}}

\newcommand{\translatioHymnusVesp}{
Štědrý nebes stvořiteli
Věčný osvítiteli
Kriste vykupiteli
prosby vyslyš věrných milý.

Nad zánikem věku v smrti
smiloval ses, milý choti
a světu spásy lačnému
lék jsi přinesl vinnému.

Když se večer světa chýlí,
vyšels jako ženich milý
z domu matky, panny čestné,
z její komnaty milostné.

Mocí tvojí velmi silnou
kolena se k zemi ohnou
a nebe, země tvorové
se poddávají vůli tvé.

Jenž přicházíš věku soudce, 
prosíme tě světovládce,
uchovej nás v našem čase1 
před úkladnou ranou zhoubce.

Chválu, sílu, čest a slávu,
Bohu Otci, jeho Synu ,
Duchu také rady svaté 
vzdej na věky věků světe.
Amen.
}

\newcommand{\translatioHymnusLaud}{
Jasný hlas k nebi zní
temnoty všechny zahání
a trestá sen, ten prchá, jak
z výšin se Kristův zaskví zrak 

Probuď se mysli strnulá,
hluchotou povstaň zraněná,
nová se hvězda zažíhá,
škody od tebe odnímá.

Shůry přichází Beránek 
polehčit vinným jejich stesk,
v slzách o milost volejme,
smilování si žádejme.

Až pak podruhé zabuší
a svět jeho hrůza zkruší
za hříchy nás nepotrestal
a ochranu svou by nám dal.

Chválu, sílu, čest a slávu,
Bohu Otci, jeho Synu ,
Duchu také rady svaté 
vzdej na věky věků světe.
Amen.
}

% MASS ---

\newcommand{\translatioIntroitus}{\translatioCantus{}}

\newcommand{\translatioGraduale}{\translatioCantus{}}

\newcommand{\translatioAlleluia}{\translatioCantus{}}

\newcommand{\translatioOffertorium}{\translatioCantus{}}

\newcommand{\translatioCommunio}{\translatioCantus{}}

\newcommand{\antiphonaI}{
  \antiphona{VII c}{cantus/amon33/crux_laud_ant1.tex}{\trAntiphonaI}}
\newcommand{\antiphonaII}{
  \antiphona{III a}{cantus/amon33/crux_laud_ant2.tex}{\trAntiphonaII}}
\newcommand{\antiphonaIII}{
  \antiphona{I f}{cantus/amon33/crux_laud_ant3.tex}{\trAntiphonaIII}}
\newcommand{\antiphonaIV}{
  \antiphona{VII c}{cantus/amon33/crux_laud_ant4.tex}{\trAntiphonaIV}}
\newcommand{\antiphonaV}{
  \antiphona{II D}{cantus/amon33/crux_laud_ant5.tex}{\trAntiphonaV}}

\newcommand{\capitulumHocEnim}{
  \scriptura{Phil 2, 5-7.}
  
  \includescore{cantus/amon33/capitulum-HocEnimSentite}

  \trCapitulum
}

\newcommand{\anteOrationem}{
  \rubrica{Ante Orationem, cantatur a Superiore:}

  \pars{Supplicatio Litaniæ.}

  \includescore{\ccommunesAM/supplicatiolitaniae.tex}

  \pars{Oratio Dominica.}

  \includescore{\ccommunesAM/oratiodominica.tex}

  \rubrica{Deinde dicitur ab Hebdomadario:}

  \includescore{\ccommunesAM/dominusvobiscum-solemnis.tex}

  % original rubric from the Antiphonale Monasticum:
  %\rubrica{In choro monialium loco Dominus vobiscum dicitur:}

  \rubrica{Absente sacerdote vel diacono, loco \textnormal{Dóminus vobíscum}
  dicitur:}

  \includescore{\ccommunesAM/domineexaudi.tex}
}

\newcommand{\oratio}{
  \pars{Oratio.}

  \includescore{cantus/amon33/crux_oratio}

  \trOratio
}

\newcommand{\rubricaBenedicamus}{
  \rubrica{Repetito \textnormal{Dóminus vobíscum} in eodem tono ut antea,
  cantatur a cantore:}
}

\newcommand{\postBenedicamus}{
  \rubrica{Postea dicitur voce recta et paululum depressa:}

  \noindent ℣. Fidélium ánimæ per misericórdiam Dei requiéscant in pace.\\
  ℟. Amen.

  \noindent Pater noster. \rubricatum{totum secreto.}

  \rubrica{Deinde, si discedendum est a Choro:}

  \noindent ℣. Dóminus det nobis suam pacem.\\
  ℟. Et vitam ætérnam. Amen.

  \rubrica{Tunc dicitur Antiphona B.~M.~V. pro tempore, 
    cum \textnormal{℣.} et Oratione
    in Tono simplici, pg. \pageref{antiphonafinalis}. Deinde:}

  \noindent ℣. Divínum auxílium máneat semper nobíscum.\\
  ℟. Et cum frátribus nostris abséntibus. Amen.
}

% the two following prayers are from
% Breviarium monasticum, Romae sumptibus Josephi Salviucci 1831, p. lxxxiv.
% http://books.google.cz/books?id=GnFG6Z4Xuc8C&dq=breviarium%20monasticum&hl=cs&pg=PR1#v=onepage&q=breviarium%20monasticum&f=false
% but - this needs to be verified - it seems that exactly the same prayers
% were in the secular as well as monastic breviary, possibly since the
% post-Tridentine reform
\newcommand{\anteOfficiumOratio}{
\rubrica{Oratio dicenda ante inchoationem Divini Officii.}

\lettrine{A}{peri} Dómine os meum ad benedicéndum nomen sanctum tuum:
munda quoque cor meum ab ómnibus vanis, pervérsis, et aliénis
cogitatiónibus;
intelléctum illúmina, afféctum inflámma,
ut digne, atténte ac devóte hoc Offícium recitáre váleam,
et exaudíri mérear ante conspéctum Divínæ Majestátis tuæ.
Per Christum, Dominum nostrum.
℟. Amen.

Dómine, in unióne illíus Divínæ intentiónis,
qua ipse in terris laudes Deo persolvísti,
has tibi Horas \rubricatum{(vel \textnormal{hanc tibi Horam})} persólvo.
}

\newcommand{\postOfficiumOratio}{
\rubrica{
  Orationem sequentem devote post Officium recitantibus
  Leo Papa X. defectus, et culpas in eo persolvendo ex humana
  fragilitate contractas, indulsit, et dicitur flexis genibus.
}

\lettrine{S}{acrosánctæ} et indivíduæ Trinitáti,
crucifixi Domini nostri Jesu Christi humanitáti,
beatissimæ et gloriosíssimæ sempérque Virginis Maríæ
fecúndæ integritáti, 
et omnium Sanctórum universitáti,
sit sempitérna laus, honor, virtus et gloria
ab omni creatúra,
nobísque remíssio omnium peccatórum,
per infiníta sæcula sæculórum.
℟. Amen.

\noindent ℣. Beáta viscera Maríæ Virginis, quæ portavérunt
ætérni Patris Fílium.\\
℟. Et beáta ubera, quæ lactavérunt Christum Dominum.

Pater noster. Ave María.
}


\newcommand{\annusEditionis}{2013}
% directories
\newcommand{\ccommunesAM}{../../cantuscommunes/amon33}
\newcommand{\ccommunesAR}{../../cantuscommunes/arom12}

\begin{document}

% GREGORIO general settings:
% staff size
\setgrefactor{15}

\newcommand{\spacearoundinitialNormal}{
  % space around the initial.
  \setspaceafterinitial{2.2mm plus 0em minus 0em}
  \setspacebeforeinitial{2.2mm plus 0em minus 0em}
}

\spacearoundinitialNormal

% initial font. 
\def\greinitialformat#1{%
{\fontsize{43}{43}\selectfont #1}%
}

\pagestyle{empty}

% multicols setting
\setlength{\columnseprule}{1pt} % cara oddelujici sloupce
\setlength{\columnsep}{20pt} % prostor mezi sloupci

%%%% Titulni stranka
\begin{titulusOfficii}
\nomenFesti{Officium Parvum Beatæ Mariæ Virginis.}
\end{titulusOfficii}

\vfill

\begin{center}
Ad usum et secundum consuetudines chori \guillemotright Conventus Choralis\guillemotleft.

Editio Sancti Wolfgangi \annusEditionis
\end{center}

\pagebreak

\quasiHora{Regulæ generales}

\begin{multicols}{2}

1. Officium parvum B. Mariæ Virg. octo complectitur partes, quas Horas vocant;
Matutinum nempe, Laudes, Primam, Tertiam, Sextam, Nonam, Vesperas, denique
Completorium. Quamlibet autem Horam separatim ab alia recitare licet.

2. Horas hoc fere tempore recitare convenit, nempe Matutinum cum Laudibus 
pridie ab hora secunda pomeridiana, Primam, Tertiam, Sextam et Nonam mane,
Vesperas et Completorium post meridiem.
Tempore autem Quadragesimæ, id est a Sabbato~I in Quadragesima usque ad
Sabbatum sanctum inclusive, satius est Vesperas ante comestionem recitare.

3. Ad lucrandas indulgentias Officio parvo B.~Mariæ Virg. annexas, in publica
recitatione omnes lingua latina uti debent, in privata autem recitatione
qualibet versione fideli et probata uti valent.
Porro recitatio retinenda est adhuc privata, quamvis locum habeat in communi
intra sæpta domus religiosæ, immo et in ipsa Ecclesia vel publico Oratorio
prædictæ domui annexis, sed januis clausis (S.~C. Indulg. 13~Sept. 1888,
28~Aug. 1903, 18~Dec. 1906).

4. Quamvis Officium parvum B.~Mariæ Virg. infra annum sit unum idemque,
exceptis partibus propriis Tempore Adventus et post Nativitatem Domini,
communiter tamen in tria Officia distinguitur:
Primum ergo Officium est dicendum a Matutino diei 3 Februarii usque ad Nonam
Sabbati ante Dominicam I Adventus inclusive, præterquam in Festo Annuntiationis
B.~M.~V., in quo dicitur Officium ut in Adventu.
Secundum recitandum est a Vesperis Sabbati ante Dominicam~I Adventus usque ad
Nonam Vigiliæ Nativitatis Domini inclusive et in Festo Annuntiationis
B.~Mariæ Virg.
Tertium persolvendum est a Vesperis Diei 24 Decembris usque ad Completorium
diei 2 Februarii inclusive.

5. Hoc Officium quovis tempore persolvendum est plane, prout in Breviario
præscribitur. Quamobrem Tempore Passionis, incluso etiam ultimo Triduo 
sacro Majoris Hebdomadæ, non omittitur \rubricatum{Glória Patri} in Invitatorio
ac tertio Responsorio; nec Tempore Paschali numerus Antiphonarum imminuitur,
neque Invitatorio, Antiphonis, Versibus et Responsoriis additur in fine
\rubricatum{Allelúja} (Rubr. gen. Brev. tit.~37, num.~2. S.~R.~C. num. 1334 ad~6).

6. Tempore Paschali, idest a Vesperis Sabbati sancti usque ad Nonam Sabbati
infra Octavam Pentecostes inclusive, Officium parvum B.~Mariæ Virg.
dicitur sicuti per Annum; sed ad \rubricatum{Benedíctus}, ad \rubricatum{Magníficat}
et ad \rubricatum{Nunc dimíttis} dicitur Antiphona \rubricatum{Regína cæli}.

7. In Festo Annuntiationis (a Matutino usque ad Completorium inclusive)
idem Officium recitatur, quod præscribitur Tempore Adventus, et in fine
dicitur Antiphona \rubricatum{Ave, Regína cælórum} vel \rubricatum{Regina cæli}
juxta temporis diversitatem. Si hoc Festum celebretur Tempore Quadragesimæ,
loco \rubricatum{Allelúja} in principio omnium Horarum dicitur
\rubricatum{Laus tibi, Dómine, Rex ætérnæ glóriæ}.

8. Quando Officium parvum B.~Mariæ Virg. recitatur separatim ab Officio divino,
Hymnus \rubricatum{Te Deum} dicitur 
a Nativitate Domini usque ad Sabbatum ante Dominicam Septuagesimæ inclusive
et a Dominica Resurrectionis usque ad Sabbatum ante Dominicam I Adventus
pariter inclusive;
in Adventu autem et a Septuagesima usque ad Pascha nonnisi in Festis B.~Mariæ
Virg. (S.~R.~C. n.~3572 ad~1 et 3659), quæ in Ecclesia universali
celebrantur vel in Kalendario approbato respectivæ Dioecesis vel Instituti
religiosi vel Ecclesiæ sive Oratorii assignantur,
atque in Festo sancti Joseph.
Quando dicitur \rubricatum{Te Deum}, in fine secundi Responsorii adjungitur
℣. \rubricatum{Glória Patri, et Fílio, et Spirítui Sancto}, ac repetitur
altera pars Responsorii.

9. Ante Orationem, etiam quando quis solus recitat Officium,
semper dicitur Versus \rubricatum{Dominus vobiscum} et respondetur
\rubricatum{Et cum spiritu tuo}. Qui Versus non dicitur ab eo, qui non est saltem
in ordine Diaconatus. Si quis autem ad Diaconatus ordinem non pervenerit,
ejus loco dicat Versum \rubricatum{Dómine, exáudi oratiónem meam},
et ei respondetur \rubricatum{Et clamor meus ad te véniat}.
Deinde dicitur \rubricatum{Orémus}, postea Oratio.
Et post ultimam Orationem repetitur ℣.~\rubricatum{Dóminus vobíscum}
vel \rubricatum{Dómine exáudi}. (Rubr. gen. tit.~30, n.~3.)

10. Omnes manu extensa se signent signo crucis a fronte ad pectus et a sinistro
humero ad dexterum
ad \rubricatum{Benedíctus} et ad \rubricatum{Magníficat}. Ceterum servetur
consuetudo quoad sugnum crucis communiter faciendum ad alias Officii partes,
scilicet \rubricatum{Dómine, lábia mea; Convérte nos, Deus; Deus, in adjutórium;
Nunc dimíttis;} et ad benedictionem in fine Completorii.

\end{multicols}


\vfill

\pagebreak

\renewcommand{\headrulewidth}{0pt} % no horiz. rule at the header
\fancyhf{}
\pagestyle{fancy}

\divisio{Extra Adventum.}

\rubrica{Quod dicitur a Matutino diei 3 Februarii usque ad Nonam
Sabbati ante Dominicam I Adventus inclusive, 
præterquam in Festo Annuntiationis B.~M.~V., in quo dicitur,
ut in Adventu, pg. \pageref{tempus:adventus}.}

% Extra Adventum.

\hora{Ad Vesperas.} %%%%%%%%%%%%%%%%%%%%%%%%%%%%%%%%%%%%%%%%

\vspace{1cm}
\deusInAdiutorium

\vfill

\pagebreak

%%% Psalms

\pars{Psalmus 1.}

\antiphonaI

\scriptura{Psalmus 109.}

\includescore{temporalia/ps109-initium-iii-a-auto.tex}

\psalmusEtTranslatio{temporalia/ps109-iii-a.tex}{temporalia/ps109-boh.tex}

\vfill

\pagebreak

\pars{Psalmus 2.}

\antiphonaII

\scriptura{Psalmus 112.}

\includescore{temporalia/ps112-initium-iv-A-auto.tex}

\psalmusEtTranslatio{temporalia/ps112-iv-a.tex}{temporalia/ps112-boh.tex}

\vfill

\pagebreak

\pars{Psalmus 3.}

\antiphonaIII

\scriptura{Psalmus 121.}

\includescore{temporalia/ps121-initium-iii-b-auto.tex}

\psalmusEtTranslatio{temporalia/ps121-iii-b.tex}{temporalia/ps121-boh.tex}

\vfill

\pagebreak

\pars{Psalmus 4.}

\antiphonaIV

\scriptura{Psalmus 126.}

\includescore{temporalia/ps126-initium-viii-G-auto.tex}

\psalmusEtTranslatio{temporalia/ps126-viii-g.tex}{temporalia/ps126-boh.tex}

\vfill

\pagebreak

\pars{Psalmus 5.}

\antiphonaV

\scriptura{Psalmus 147.}

\includescore{temporalia/ps147-initium-iv-A-auto.tex}

\psalmusEtTranslatio{temporalia/ps147-iv-a.tex}{temporalia/ps147-boh.tex}

\vfill

\pagebreak

%%% capitulum

\pars{Capitulum.}

\label{vesperaecapitulum}

\capitulumAbInitio

%%% hymnus

\pars{Hymnus}

\superInitialam{VII}
\includescore{temporalia/hymnus-AveMarisStella.tex}

\versiculusDiffusaEst

\pagebreak

%%% Magnificat

\pars{Canticum Beatae Mariae Virginis}

\rubrica{Extra Tempus Paschale:}

\antiphona{II D}{cantus/arom12/antmag}{}

\scripturaMagnificat

\includescore{../../tonipsalmorum/arom12/magnificat-initium-ii-D.tex}

\psalmusEtTranslatioB{temporalia/magnificat-ii-d.tex}{temporalia/magnificat-boh.tex}{10cm}

\vfill

\pagebreak


\rubrica{Tempore Paschali:}

\paschaAntiphona

\scripturaMagnificat

\psalmusEtTranslatio{temporalia/magnificat-i-d2.tex}{temporalia/magnificat-boh.tex}

\vfill

\pagebreak

%%% oratio

\label{vesperaefinis}

\inFineHorarumExtraAdventum

\vfill



\pagebreak

\label{pars:completorium}

% Extra Adventum.

\hora{Ad Completorium.} %%%%%%%%%%%%%%%%%%%%%%%%%%%%%%%%%%%%%%%%

\vspace{1cm}

\includescore{\ccommunesAR/convertenosdeus.tex}

\vspace{5mm}

\deusInAdiutorium

\vfill

\pagebreak

\pars{psalmus 1.}

\scriptura{Psalmus 128.}

\includescore{temporalia/ps128-initium-dir-auto.tex}

\psalmusEtTranslatio{temporalia/ps128-dir.tex}{empty.tex}

\vfill


\pars{psalmus 2.}

\scriptura{Psalmus 129.}

\includescore{temporalia/ps129-initium-dir-auto.tex}

\psalmusEtTranslatio{temporalia/ps129-dir.tex}{empty.tex}


\pagebreak

\pars{psalmus 3.}

\scriptura{Psalmus 130.}

\includescore{temporalia/ps130-initium-dir-auto.tex}

\psalmusEtTranslatio{temporalia/ps130-dir.tex}{empty.tex}

\vfill

\pagebreak

\pars{Hymnus}

\superInitialam{II}
\includescore{temporalia/hymnus-MementoRerumConditor.tex}

\pars{Capitulum}

\pars{Versiculus}

\vfill

\pagebreak


\pagebreak

\label{pars:matutinum}

% Extra Adventum.

\hora{Ad Matutinum.}

\pars{Invitatorium.}

\superInitialam{VII}
\includescore{cantus/manuscripti/matinvit.tex}

\scriptura{Psalmus 95. (Textus antiquus latinus.)}

\superInitialam{VII}
\includescore{cantus/manuscripti/venite7/venite7a.tex}

\invitatoriumIntegrum

\includescore{cantus/manuscripti/venite7/venite7b.tex}

\invitatoriumAltera

\includescore{cantus/manuscripti/venite7/venite7c.tex}

\invitatoriumIntegrum

\includescore{cantus/manuscripti/venite7/venite7d.tex}

\invitatoriumAltera

\includescore{cantus/manuscripti/venite7/venite7e.tex}

\invitatoriumIntegrum

\includescore{cantus/manuscripti/venite7/venite7f.tex}

\rubrica{Repetitur altera pars Invitatorii. 
Denique repetitur integrum Invitatorium.}

\vfill \pagebreak

Dominica, Feria II et V.

\pars{psalmus 1.}

\antiphona{IV A*}{cantus/arom12/matant1.tex}{}

\scriptura{Psalmus 8.}

\includescore{temporalia/ps8-initium-iv-A-auto.tex}

\psalmusEtTranslatio{temporalia/ps8-iv-a.tex}{empty.tex}

\vfill \pagebreak

\pars{psalmus 2.}

\antiphona{IV A*}{cantus/manuscripti/matant2b.tex}{}

\scriptura{Psalmus 18.}

\includescore{temporalia/ps18-initium-iv-A-auto.tex}

\psalmusEtTranslatio{temporalia/ps18-iv-a.tex}{empty.tex}

\vfill \pagebreak

\pars{psalmus 3.}

\antiphona{IV A*}{cantus/manuscripti/matant3.tex}{}

\scriptura{Psalmus 23.}

\includescore{temporalia/ps23-initium-iv-A-auto.tex}

\psalmusEtTranslatio{temporalia/ps23-iv-a.tex}{empty.tex}

\vfill \pagebreak



Feria III et VI.

\pars{psalmus 1.}

\antiphona{VII c}{cantus/manuscripti/matant4.tex}{}

\scriptura{Psalmus 44.}

% \vfill \pagebreak

\pars{psalmus 2.}

\antiphona{VII c}{cantus/manuscripti/matant5.tex}{}

\scriptura{Psalmus 45.}

% \vfill \pagebreak

\pars{psalmus 3.}

\antiphona{VII c}{cantus/manuscripti/matant6.tex}{}

\scriptura{Psalmus 86.}

\vfill \pagebreak



Feria IV et Sabbato.

\pars{psalmus 1.}

\antiphona{IV A*}{cantus/manuscripti/matant7.tex}{}

\scriptura{Psalmus 95.}

% \vfill \pagebreak

\pars{psalmus 2.}

\antiphona{IV A*}{cantus/manuscripti/matant8.tex}{}

\scriptura{Psalmus 96.}

% \vfill \pagebreak

\pars{psalmus 3.}

\antiphona{IV A*}{cantus/manuscripti/matant9.tex}{}

\scriptura{Psalmus 97.}

\vfill \pagebreak



Todo: versiculus

\rubrica{\textnormal{Pater noster} secreto usque ad}
℣. Et ne nos indúcas in tentatiónem. ℟. Sed líbera nos a malo.

Todo: absolutio

\pars{lectio 1.}

Todo: Jube et benedictio

\superInitialam{II}
\includescore{cantus/manuscripti/matresp1.tex}

\vfill \pagebreak

\pars{lectio 2.}

Todo: Jube et benedictio

\superInitialam{I}
\includescore{cantus/manuscripti/matresp2.tex}

\vfill \pagebreak

\pars{lectio 3.}

Todo: Jube et benedictio

Todo: Te Deum


\pagebreak

% Extra Adventum.

\hora{Ad Laudes.} %%%%%%%%%%%%%%%%%%%%%%%%%%%%%%%%%%%%%%%%

\vspace{1cm}

\rubricaLaudesPrincipium

\deusInAdiutorium

\vfill

\pagebreak

\pars{Psalmus 1.}

\antiphonaLaudI

\scriptura{Psalmus 92.}

\includescore{temporalia/extra-adventum-laudes/ps92-initium-vii-a-auto.tex}

\psalmusEtTranslatio{temporalia/extra-adventum-laudes/ps92-vii-a.tex}{temporalia/extra-adventum-laudes/ps92-boh.tex}

\vfill

\pagebreak

\pars{Psalmus 2.}

\antiphonaLaudII

\scriptura{Psalmus 99.}

\includescore{temporalia/extra-adventum-laudes/ps99-initium-viii-G-auto.tex}

\psalmusEtTranslatio{temporalia/extra-adventum-laudes/ps99-viii-g.tex}{temporalia/extra-adventum-laudes/ps99-boh.tex}

\vfill

\pagebreak

\pars{Psalmus 3.}

\antiphonaLaudIII

\scriptura{Psalmus 62.}

\includescore{temporalia/extra-adventum-laudes/ps62-initium-iv-A-auto.tex}

\psalmusEtTranslatio{temporalia/extra-adventum-laudes/ps62-iv-a.tex}{temporalia/extra-adventum-laudes/ps62-boh.tex}

\vfill

\pagebreak

\pars{Canticum trium Puerorum}

\antiphonaLaudIV

\scriptura{Dan. 3, 57-88.56}

\includescore{temporalia/extra-adventum-laudes/dan3-initium-vii-c2-auto.tex}

\psalmusEtTranslatio{temporalia/extra-adventum-laudes/dan3-vii-c2.tex}{temporalia/extra-adventum-laudes/dan3-boh.tex}

\vfill

\pagebreak

\pars{Psalmus 4.}

\antiphonaLaudV

\scriptura{Psalmus 148.}

\includescore{temporalia/extra-adventum-laudes/ps148-initium-i-g2-auto.tex}

\psalmusEtTranslatio{temporalia/extra-adventum-laudes/ps148-i-g2.tex}{temporalia/extra-adventum-laudes/ps148-boh.tex}

\vfill

\pagebreak

\pars{Capitulum}

\label{laudescapitulum}

\scriptura{Cant. 6, 8}

\includescore{cantus/arom12/capitulum-VideruntEam.tex}

\pars{Hymnus}

\superInitialam{II}
\includescore{temporalia/hymnus-OGloriosaVirginum.tex}

\includescore{cantus/arom12/versiculus-BenedictaTu.tex}

\pagebreak

\pars{Canticum Zachariæ}

\rubrica{Extra Tempus Paschale:}

\antiphona{VIII G}{cantus/arom12/antlaudben}{}

\scripturaBenedictus

\includescore{temporalia/extra-adventum-laudes/benedictus-initium-viii-G-auto.tex}

\psalmusEtTranslatioB{temporalia/extra-adventum-laudes/benedictus-viii-g.tex}{temporalia/extra-adventum-laudes/benedictus-boh.tex}{10cm}

\vfill

\rubrica{Sequitur \textnormal{Kýrie eléison} etc. usque ad finem horæ
ut in Vesperis, p. \pageref{vesperaefinis}.}

\pagebreak

\rubrica{Tempore Paschali:}

\paschaAntiphona

\scripturaBenedictus

\includescore{temporalia/tempore-paschali/benedictus-initium-i-D2-auto.tex}

\psalmusEtTranslatioB{temporalia/tempore-paschali/benedictus-i-d2.tex}{temporalia/tempore-paschali/benedictus-boh.tex}{10cm}

\vfill

\pagebreak

\label{laudesfinis}
\inFineHorarumExtraAdventumLaudes

\rubrica{Si terminatur Officium, alioquin, si alia subsequatur Hora,
  in fine ultimæ Horæ dicitur:}

\rubrica{\textnormal{Pater noster,} totum secreto,
  eoque recitato subjungitur}

\noindent ℣. Dóminus det nobis suam pacem.\\
℟. Et vitam ætérnam. Amen.

\rubrica{Et immediate dicitur, cum suis Versu et Oratione,
una ex finalibus beatæ Mariæ Virginis Antiphonis, pro diversitate Temporis,
pg. \pageref{antiphonaefinales}, flexis quidem genibus;
diebus autem Dominicis, a Vesperis Sabbati inclusive et Tempore Paschali,
stando.}

\rubrica{Postea concluditur:}

\noindent ℣. Divínum auxílium maneat semper nobíscum.\\
℟. Amen.


\pagebreak

% Extra Adventum.

\hora{Ad Primam.} %%%%%%%%%%%%%%%%%%%%%%%%%%%%%%%%%%%%%%%%

\label{prima}

\vspace{1cm}

\deusInAdiutorium

\vfill

\pagebreak

\pars{Hymnus}

\includescore{temporalia/hymnus-MementoRerumConditor.tex}

\pars{Psalmi}

\antiphonaLaudI

\scriptura{Psalmus 53.}

\includescore{temporalia/extra-adventum-prima/ps53-initium-vii-a-auto.tex}

\psalmusEtTranslatio{temporalia/extra-adventum-prima/ps53-vii-a.tex}{temporalia/extra-adventum-prima/ps53-boh.tex}

\scriptura{Psalmus 84.}

\psalmusEtTranslatio{temporalia/extra-adventum-prima/ps84-vii-a.tex}{temporalia/extra-adventum-prima/ps84-boh.tex}

\scriptura{Psalmus 116.}

\psalmusEtTranslatio{temporalia/extra-adventum-prima/ps116-vii-a.tex}{temporalia/extra-adventum-prima/ps116-boh.tex}

\pars{Capitulum}

\label{primacapitulum}

\scriptura{Cant. 6, 9}

\includescore{cantus/arom12/capitulum-QuaeEst}

\noindent ℣. Dignáre me laudáre te, Virgo sa\textbf{crá}ta.\\
℟. Da mihi virtútem contra hostes \textbf{tu}os.

\finisHoraeMinores

\pars{Oratio}

Deus, qui virginálem aulam beátæ Maríæ, in qua habitáres,
elígere dignátus es: †
da, quǽsumus; ut sua nos defensióne munítos, *
jucúndos fácias suæ interésse commemoratióni.

Qui vivis et regnas cum Deo Patre in unitáte Spíritus Sancti Deus,
per ómnia sǽcula sæculórum.
℟. Amen.

\vfill


\pagebreak

% Extra Adventum.

\hora{Ad Tertiam.} %%%%%%%%%%%%%%%%%%%%%%%%%%%%%%%%%%%%%%%%

\initiumHoraeMinores

\pars{Psalmi}

\antiphonaLaudII

\scriptura{Psalmus 119.}

\includescore{temporalia/extra-adventum-tertia/ps119-initium-viii-G-auto.tex}

\psalmusEtTranslatioB{temporalia/extra-adventum-tertia/ps119-viii-g.tex}{temporalia/extra-adventum-tertia/ps119-boh.tex}{9cm}

\scriptura{Psalmus 120.}

\psalmusEtTranslatioB{temporalia/extra-adventum-tertia/ps120-viii-g.tex}{temporalia/extra-adventum-tertia/ps120-boh.tex}{9cm}

\scriptura{Psalmus 121.}

\psalmusEtTranslatioB{temporalia/extra-adventum-tertia/ps121-viii-g.tex}{temporalia/extra-adventum-tertia/ps121-boh.tex}{9cm}

\vfill

\pars{Capitulum}

\label{tertiacapitulum}

\scriptura{Eccli. 24, 15}

\includescore{cantus/arom12/capitulum-EtSicInSion}
\vspace{3mm}

\noindent ℣. Diffúsa est grátia in lábiis \textbf{tu}is.\\
℟. Proptérea benedíxit te Deus in æ\textbf{tér}num.

\finisHoraeMinores

\pars{Oratio}

Deus, qui salútis ætérnæ, beátæ Maríæ virginitáte fœcúnda,
humáno géneri prǽmia præstitísti: †
tríbue, quǽsumus; ut ipsam pro nobis intercédere sentiámus, *
per quam merúimus auctórem vitæ suscípere, 
Dóminum nostrum Jesum Christum Fílium tuum.

Qui tecum vivit et regnat in unitáte Spíritus Sancti Deus,
per ómnia sǽcula sæculórum.


\pagebreak

% Extra Adventum.

\hora{Ad Sextam.} %%%%%%%%%%%%%%%%%%%%%%%%%%%%%%%%%%%%%%%%

\initiumHoraeMinores

\pars{Psalmi}

\antiphonaLaudIII

\scriptura{Psalmus 122.}

\includescore{temporalia/extra-adventum-sexta/ps122-initium-iv-A-auto.tex}

\psalmusEtTranslatio{temporalia/extra-adventum-sexta/ps122-iv-a.tex}{temporalia/extra-adventum-sexta/ps122-boh.tex}


\scriptura{Psalmus 123.}

\psalmusEtTranslatio{temporalia/extra-adventum-sexta/ps123-iv-a.tex}{temporalia/extra-adventum-sexta/ps123-boh.tex}


\scriptura{Psalmus 124.}

\psalmusEtTranslatio{temporalia/extra-adventum-sexta/ps124-iv-a.tex}{temporalia/extra-adventum-sexta/ps124-boh.tex}

% stejna antifona, ale bez udani modu etc - tim se signalizuje konec
\antiphona{}{cantus/arom12/antlaud3.tex}{\translatioLaudAntIII}}


\vfill

\pars{Capitulum}

\label{sextacapitulum}

\scriptura{Eccli. 24, 16}

\includescore{cantus/arom12/capitulum-EtRadicavi.tex}
\vspace{3mm}

\noindent ℣. Benedícta tu in muli\textbf{é}ribus.\\
℟. Et benedíctus fructus ventris \textbf{tu}i.

\finisHoraeMinores

\pars{Oratio}

Concéde, miséricors Deus, fragilitáti nostræ præsídium:
ut, qui sanctæ Dei Genetrícis memóriam ágimus;
intercessiónis ejus auxílio, a nostris iniquitátibus resurgámus.

Per eúndem Dóminum nostrum Jesum Christum, Fílium tuum:
Qui tecum vivit et regnat in unitáte Spíritus Sancti Deus,
per ómnia sǽcula sæculórum.

\vfill


\pagebreak

% Extra Adventum.

\hora{Ad Nonam.} %%%%%%%%%%%%%%%%%%%%%%%%%%%%%%%%%%%%%%%%

\vspace{1cm}

\deusInAdiutorium

\vfill

\pagebreak

\pars{Hymnus}

\pars{Psalmi}

\antiphonaLaudV

\scriptura{Psalmus 125.}

\includescore{temporalia/ps125-initium-i-g2-auto.tex}

\psalmusEtTranslatio{temporalia/ps125-i-g2.tex}{empty.tex}


\scriptura{Psalmus 126.}

\includescore{temporalia/ps126-initium-i-g2-auto.tex}

\psalmusEtTranslatio{temporalia/ps126-i-g2.tex}{empty.tex}


\scriptura{Psalmus 127.}

\includescore{temporalia/ps127-initium-i-g2-auto.tex}

\psalmusEtTranslatio{temporalia/ps127-i-g2.tex}{empty.tex}


\pagebreak

%%%%%%%%%%%%%%%%%%%%%%%%%%%%%%%%%%%%%%%%%%%%%%%%%%%%%%%%%%%%%%%%%%%%%%

\divisio{In Adventu.}

\label{tempus:adventus}

\rubrica{Quod dicitur a Vesperis Sabbati ante Dominicam I Adventus 
usque ad Nonam Vigiliae Nativitatis Domini inclusive.}

% In Adventu.

\hora{Ad Vesperas.} %%%%%%%%%%%%%%%%%%%%%%%%%%%%%%%%%%%%%%%%

\cantusSineNeumas

\vspace{1cm}
\includescore{\ccommunesAR/deusinadiutorium-ferialis.tex}
\vspace{1cm}

\vfill

\pagebreak

\cantusCumNeumis

\pars{psalmus 1.}

\adventAntiphonaI

\scriptura{Psalmus 109.}

\includescore{temporalia/ps109-initium-viii-G-auto.tex}

\psalmusEtTranslatio{temporalia/ps109-viii-g.tex}{temporalia/ps109-boh.tex}

\vfill

\pagebreak

\pars{psalmus 2.}

\adventAntiphonaII

\scriptura{Psalmus 112.}

\includescore{temporalia/ps112-initium-i-g-auto.tex}

\psalmusEtTranslatio{temporalia/ps112-i-g.tex}{temporalia/ps112-boh.tex}

\vfill

\pagebreak

\pars{psalmus 3.}

\adventAntiphonaIII

\scriptura{Psalmus 121.}

\includescore{temporalia/ps121-initium-viii-G-auto.tex}

\psalmusEtTranslatio{temporalia/ps121-viii-g.tex}{temporalia/ps121-boh.tex}

\vfill

\pagebreak

\pars{psalmus 4.}

\adventAntiphonaIV

\scriptura{Psalmus 126.}

\includescore{temporalia/ps126-initium-i-f-auto.tex}

\psalmusEtTranslatio{temporalia/ps126-i-f.tex}{temporalia/ps126-boh.tex}

\vfill

\pagebreak

\pars{psalmus 5.}

\adventAntiphonaV

\scriptura{Psalmus 147.}

\includescore{temporalia/ps147-initium-viii-c-auto.tex}

\psalmusEtTranslatio{temporalia/ps147-viii-c.tex}{temporalia/ps147-boh.tex}

\vfill


\pagebreak

%\hora{Completorium}

%\rubrica{Completorium fit ut Extra Adventum, pg. \pageref{pars:completorium}, 
%tantum Capitulum
%cum Versiculo suo, Antiphona pro Nunc dimittis et Oratio sunt propria.}

\hora{Ad Matutinum.}

\rubrica{Matutinum fit ut Extra Adventum, pg. \pageref{pars:matutinum}.
Feria IV et Sabbato dicitur tertius psalmus cum antiphona speciali, 
\textnormal{Angelus Dómini}. Lectiones et Responsoria sunt propria,
Absolutiones atque Benedictiones ut Extra Adventum.}

\vfill

\pars{Lectio i}
\scriptura{Luc. 1, 26-28}

Missus est Angelus Gábriel a Deo in civitátem Galilǽæ,
cui nomen Názareth, ad Vírginem desponsátam viro, cui nomen erat Joseph,
de domo David, et nomen Vírginis María.
Et, ingréssus Angelus ad eam, dixit: Ave, grátia plena;
Dóminus tecum: benedícta tu in muliéribus.
\tuAutem

\vfill

\superInitialam{VII}
\includescore{cantus/nrom02/adv_matresp1}

\vfill

\pars{Lectio ii}
\scriptura{Luc 1, 29-33}

Quæ cum audísset, turbáta est in sermóne ejus, et cogitábat
quális esset ista salutátio. Et ait Angelus ei:
Ne tímeas, María; invenísti grátiam apud Deum: ecce, concípies
in útero et páries fílium, et vocábis nomen ejus Jesum.
Hic erit magnus, et Fílius Altíssimi vocábitur;
et dabit illi Dóminus Deus sedem David, patris ejus;
et regnábit in domo Jacob in ætérnum, et regni ejus non erit finis.
\tuAutem

\pagebreak

\rubrica{Quando dicitur \textnormal{Te Deum,} id est in Festis B. Mariæ
Virg., sequens responsorium dicitur cum \textnormal{Glória Patri.}
Cum autem \textnormal{Te Deum} non dicitur, \textnormal{Glória Patri}
et ultima repetitio \textnormal{Ave, María} omittitur.}

\superInitialam{VII}
\includescore{cantus/nrom02/adv_matresp2}

\vfill

\pars{Lectio iii}
\scriptura{Luc 1, 34-38}

Dixit autem María ad Angelum: Quómodo fiet istud, quóniam virum non
cognósco? Et respóndens Angelus dixit ei: Spíritus Sanctus supervéniet
in te, et virtus Altíssimi obumbrábit tibi. Ideóque et quod nascétur
ex te Sanctum, vocábitur Fílius Dei. Et ecce, Elísabeth, cognáta tua,
et ipsa concépit fílium en senectúte sua, et hic mensis sextus est illi,
quæ vocátur stérilis; quia non erit impossíbile apud Deum omne verbum.
Dixit autem María: Ecce ancílla Dómini: fiat mihi secúndum verbum tuum.
\tuAutem

\vfill

\rubrica{Extra Festa B. Mariæ Virg.:}

\superInitialam{IV}
\includescore{cantus/nrom02/adv_matresp3}

\vspace{1cm}

\rubrica{Dicto \textnormal{Te Deum} aut tertio Responsorio, 
statim incipiuntur Laudes a Versu \textnormal{Deus in adjutórium}.
Si autem Laudes a Matutino separantur, dicitur hic Oratio et finis
horæ ut in Vesperis, pg. \pageref{vesperaefinisadvent}.
(\textnormal{Kyrie eléison} non dicitur.)}


\pagebreak

% In Adventu.

\hora{Ad Laudes.} %%%%%%%%%%%%%%%%%%%%%%%%%%%%%%%%%%%%%%%%

\rubricaLaudesPrincipium


\pars{Psalmus 1.}

\adventAntiphonaI

\scriptura{Psalmus 92.}

\includescore{temporalia/advent-laudes/ps92-initium-viii-G-auto.tex}

\psalmusEtTranslatio{temporalia/advent-laudes/ps92-viii-g.tex}{temporalia/advent-laudes/ps92-boh.tex}

\vfill

\pagebreak

\pars{Psalmus 2.}

\adventAntiphonaII

\scriptura{Psalmus 99.}

\includescore{temporalia/advent-laudes/ps99-initium-i-g-auto.tex}

\psalmusEtTranslatio{temporalia/advent-laudes/ps99-i-g.tex}{temporalia/advent-laudes/ps99-boh.tex}

\vfill

\pagebreak

\pars{Psalmus 3.}

\adventAntiphonaIII

\scriptura{Psalmus 62.}

\includescore{temporalia/advent-laudes/ps62-initium-viii-G-auto.tex}

\psalmusEtTranslatio{temporalia/advent-laudes/ps62-viii-g.tex}{temporalia/advent-laudes/ps62-boh.tex}

\vfill

\pagebreak

\pars{Canticum trium puerorum.}

\adventAntiphonaIV

\scriptura{Dan. 57-88 et 56.}

\includescore{temporalia/advent-laudes/dan3-initium-i-f-auto.tex}

\psalmusEtTranslatio{temporalia/advent-laudes/dan3-i-f.tex}{temporalia/advent-laudes/dan3-boh.tex}

\vfill

\pagebreak

\pars{Psalmus 4.}

\adventAntiphonaV

\scriptura{Psalmus 148.}

\includescore{temporalia/advent-laudes/ps148-initium-viii-c-auto.tex}

\psalmusEtTranslatio{temporalia/advent-laudes/ps148-viii-c.tex}{temporalia/advent-laudes/ps148-boh.tex}

\vfill

\pagebreak

\pars{Capitulum}

\scriptura{Is. 11,1-2}

\includescore{cantus/arom12/capitulum-EgredieturVirga.tex}

\pars{Hymnus}

\superInitialam{II}
\includescore{temporalia/hymnus-OGloriosaVirginum}

\includescore{cantus/arom12/versiculus-BenedictaTu}

\pagebreak

\pars{Canticum Zachariæ}

\adventAntiphonaSpiritusSanctus

\scripturaBenedictus

\includescore{temporalia/advent-laudes/benedictus-initium-viiisoll-G-auto.tex}

\psalmusEtTranslatioB{temporalia/advent-laudes/benedictus-viiisoll-g.tex}{temporalia/advent-laudes/benedictus-boh.tex}{10cm}

\rubricaFinisHoraeAdvent

% Commemorations would follow, but these were cancelled 
% by the reforms of 1955. 


\pagebreak

% Advent.

\hora{Ad Primam.} %%%%%%%%%%%%%%%%%%%%%%%%%%%%%%%%%%%%%%%%

\initiumHoraeMinoresAdvent

\pars{Psalmi}

\adventAntiphonaI

\scriptura{Psalmus 53.}

\includescore{temporalia/advent-prima/ps53-initium-viii-G-auto.tex}

\psalmusEtTranslatioB{temporalia/advent-prima/ps53-viii-g.tex}{temporalia/advent-prima/ps53-boh.tex}{10cm}

\scriptura{Psalmus 84.}

\psalmusEtTranslatioB{temporalia/advent-prima/ps84-viii-g.tex}{temporalia/advent-prima/ps84-boh.tex}{10cm}

\scriptura{Psalmus 116.}

\psalmusEtTranslatioB{temporalia/advent-prima/ps116-viii-g.tex}{temporalia/advent-prima/ps116-boh.tex}{10cm}

\vfill

\pars{Capitulum}

\scriptura{Is 7, 14-15}

\includescore{cantus/arom12/capitulum-EcceVirgo}

\trCapitEcceVirgo

\vfill

\textusEtTranslatio{
  \noindent ℣. Dignáre me laudáre te, Virgo sa\textbf{crá}ta.
  ℟. Da mihi virtútem contra hostes \textbf{tu}os.
}{\trVersDignareMe}{11cm}

\vfill

\rubricaFinisHoraeAdvent


\pagebreak

% Advent.

\hora{Ad Tertiam.} %%%%%%%%%%%%%%%%%%%%%%%%%%%%%%%%%%%%%%%%

\initiumHoraeMinoresAdvent

\pars{Psalmi}

\adventAntiphonaII

\scriptura{Psalmus 119.}

\includescore{temporalia/advent-tertia/ps119-initium-i-g-auto.tex}

\psalmusEtTranslatioB{temporalia/advent-tertia/ps119-i-g.tex}{temporalia/advent-tertia/ps119-boh.tex}{10cm}

\scriptura{Psalmus 120.}

\psalmusEtTranslatioB{temporalia/advent-tertia/ps120-i-g.tex}{temporalia/advent-tertia/ps120-boh.tex}{10cm}

\scriptura{Psalmus 121.}

\psalmusEtTranslatioB{temporalia/advent-tertia/ps121-i-g.tex}{temporalia/advent-tertia/ps121-boh.tex}{10cm}

\vfill

\pars{Capitulum}

\scriptura{Is. 11, 1-2}

\includescore{cantus/arom12/capitulum-EgredieturVirga}

\vfill

\noindent ℣. Diffúsa est grátia in lábiis \textbf{tu}is.\\
℟. Proptérea benedíxit te Deus in æ\textbf{tér}num.

\vfill

\rubricaFinisHoraeAdvent



\pagebreak

% Advent.

\hora{Ad Sextam.} %%%%%%%%%%%%%%%%%%%%%%%%%%%%%%%%%%%%%%%%

\initiumHoraeMinoresAdvent

\pars{Psalmi}

\adventAntiphonaIII

\scriptura{Psalmus 122.}

\includescore{temporalia/advent-sexta/ps122-initium-viii-G-auto.tex}

\psalmusEtTranslatioB{temporalia/advent-sexta/ps122-viii-g.tex}{temporalia/advent-sexta/ps122-boh.tex}{9cm}


\scriptura{Psalmus 123.}

\psalmusEtTranslatioB{temporalia/advent-sexta/ps123-viii-g.tex}{temporalia/advent-sexta/ps123-boh.tex}{9cm}


\scriptura{Psalmus 124.}

\psalmusEtTranslatioB{temporalia/advent-sexta/ps124-viii-g.tex}{temporalia/advent-sexta/ps124-boh.tex}{9cm}

\vfill

\pars{Capitulum}

\scriptura{Luc. 1, 32-33}
% O_O is it normal to read a Gospel as a Capitulum? 

\includescore{cantus/arom12/capitulum-DabitIlli}

\vfill

\noindent ℣. Benedícta tu in muli\textbf{é}ribus.\\
℟. Et benedíctus fructus ventris \textbf{tu}i.

\vfill

\rubricaFinisHoraeAdvent


\pagebreak

% Advent.

\hora{Ad Nonam.} %%%%%%%%%%%%%%%%%%%%%%%%%%%%%%%%%%%%%%%%

\initiumHoraeMinoresAdvent

\pars{Psalmi}

\adventAntiphonaV

\scriptura{Psalmus 125.}

\includescore{temporalia/advent-nona/ps125-initium-viii-c-auto.tex}

\psalmusEtTranslatioB{temporalia/advent-nona/ps125-viii-c.tex}{temporalia/advent-nona/ps125-boh.tex}{10cm}


\scriptura{Psalmus 126.}

\psalmusEtTranslatioB{temporalia/advent-nona/ps126-viii-c.tex}{temporalia/advent-nona/ps126-boh.tex}{10cm}


\scriptura{Psalmus 127.}

\psalmusEtTranslatioB{temporalia/advent-nona/ps127-viii-c.tex}{temporalia/advent-nona/ps127-boh.tex}{10cm}

\vfill

\pars{Capitulum}

\scriptura{Is 7, 14-15}

\includescore{cantus/arom12/capitulum-EcceVirgo}

\vfill

\noindent ℣. Angelus Dómini nuntiávit Ma\textbf{rí}æ.\\
℟. Et concépit de Spíritu \textbf{Sanc}to.

\vfill

\rubricaFinisHoraeAdvent

\vfill


\pagebreak

%%%%%%%%%%%%%%%%%%%%%%%%%%%%%%%%%%%%%%%%%%%%%%%%%%%%%%%%%%%%%%%%%%%%%%

\divisio{Post Nativitatem.}

\rubrica{Quod dicitur a Vesperis diei 24 Decembris usque ad Completorium
diei 2 Februarii inclusive.}

% Post Nativitatem.

\hora{Ad Vesperas.} %%%%%%%%%%%%%%%%%%%%%%%%%%%%%%%%%%%%%%%%

\vspace{1cm}
\includescore{\ccommunesAR/deusinadiutorium-ferialis.tex}
\vspace{1cm}

\vfill

\pagebreak

\pars{Psalmus 1.}

\nativitasAntiphonaI

\scriptura{Psalmus 109.}

\includescore{temporalia/ps109-initium-vi-F-auto.tex}

\psalmusEtTranslatio{temporalia/ps109-vi-f.tex}{temporalia/ps109-boh.tex}

\vfill

\pagebreak

\pars{Psalmus 2.}

\nativitasAntiphonaII

\scriptura{Psalmus 112.}

\includescore{temporalia/ps112-initium-iii-a3-auto.tex}

\psalmusEtTranslatio{temporalia/ps112-iii-a3.tex}{temporalia/ps112-boh.tex}

\vfill

\pagebreak

\pars{Psalmus 3.}

\nativitasAntiphonaIII

\scriptura{Psalmus 121.}

\includescore{temporalia/ps121-initium-iv-E-auto.tex}

\psalmusEtTranslatio{temporalia/ps121-iv-e.tex}{temporalia/ps121-boh.tex}

\vfill

\pagebreak

\pars{Psalmus 4.}

\nativitasAntiphonaIV

\scriptura{Psalmus 126.}

\includescore{temporalia/ps126-initium-i-f-auto.tex}

\psalmusEtTranslatio{temporalia/ps126-i-f.tex}{temporalia/ps126-boh.tex}

\vfill

\pagebreak

\pars{Psalmus 5.}

\nativitasAntiphonaV

\scriptura{Psalmus 147.}

\includescore{temporalia/ps147-initium-ii-D-auto.tex}

\psalmusEtTranslatio{temporalia/ps147-ii-d.tex}{temporalia/ps147-boh.tex}

\vfill

\pagebreak

\pars{Capitulum et Hymnus}

\rubrica{Capitulum, Hymnus et Versiculus post Hymnum
  ut extra Adventum, pg. \pageref{vesperaecapitulum}.}


\pars{Canticum B. Mariæ Virginis}

\nativitasAntiphonaMagnificat

\scripturaMagnificat

\includescore{../../tonipsalmorum/arom12/magnificat-initium-ii-A.tex}

\psalmusEtTranslatioB{temporalia/magnificat-ii-a.tex}{temporalia/magnificat-boh.tex}{10cm}

\vfill

\pagebreak

\label{vesperaefinisnativitas}

\inFineHorarumPostNativitatem

\vfill


\pagebreak

\hora{Ad Matutinum.}

\rubrica{Totum ut Extra Adventum, pg. \pageref{matutinum}, cum Oratione
de Vesperis, pg. \pageref{vesperaefinisnativitas}.}

% Post Nativitatem.

\hora{Ad Laudes.} %%%%%%%%%%%%%%%%%%%%%%%%%%%%%%%%%%%%%%%%

\vspace{1cm}

\rubricaLaudesPrincipium

\deusInAdiutorium

\vfill

\pagebreak

\pars{Psalmus 1.}

\nativitasAntiphonaI

\scriptura{Psalmus 92.}

\includescore{temporalia/ps92-initium-vi-F-auto.tex}

\psalmusEtTranslatio{temporalia/ps92-vi-f.tex}{temporalia/ps92-boh.tex}

\vfill

\pagebreak

\pars{Psalmus 2.}

\nativitasAntiphonaII

\scriptura{Psalmus 99.}

\includescore{temporalia/ps99-initium-iii-a3-auto.tex}

\psalmusEtTranslatio{temporalia/ps99-iii-a3.tex}{temporalia/ps99-boh.tex}

\vfill

\pagebreak

\pars{Psalmus 3.}

\nativitasAntiphonaIII

\scriptura{Psalmus 62.}

\includescore{temporalia/ps62-initium-iv-E-auto.tex}

\psalmusEtTranslatio{temporalia/ps62-iv-e.tex}{temporalia/ps62-boh.tex}

\vfill

\pagebreak

\pars{Canticum trium Puerorum}

\nativitasAntiphonaIV

\scriptura{Dan. 3, 57-88.56}

\includescore{temporalia/dan3-initium-i-f-auto.tex}

\psalmusEtTranslatio{temporalia/dan3-i-f.tex}{temporalia/dan3-boh.tex}

\vfill

\pagebreak

\pars{Psalmus 4.}

\nativitasAntiphonaV

\scriptura{Psalmus 148.}

\includescore{temporalia/ps148-initium-ii-D-auto.tex}

\psalmusEtTranslatio{temporalia/ps148-ii-d.tex}{temporalia/ps148-boh.tex}

\vfill


\pars{Capitulum et Hymnus}

\rubrica{Capitulum, Hymnus et Versiculus post Hymnum
  ut extra Adventum, pg. \pageref{laudescapitulum}.}

\pagebreak

\pars{Canticum Zachariæ}

\nativitasAntiphonaBenedictus

\scripturaBenedictus

\includescore{temporalia/benedictus-initium-viii-G-auto}

\psalmusEtTranslatioB{temporalia/benedictus-viii-g.tex}{temporalia/benedictus-boh.tex}{10cm}

\vfill

\rubrica{Sequitur \textnormal{Kýrie eléison} etc. usque ad finem horæ
ut in Vesperis, p. \pageref{nativitasvesperaefinis}.}


\pagebreak

% Advent.

\hora{Ad Primam.} %%%%%%%%%%%%%%%%%%%%%%%%%%%%%%%%%%%%%%%%

\initiumHoraeMinoresAdvent % this is ok, really

\pars{Psalmi}

\nativitasAntiphonaI

\scriptura{Psalmus 53.}

\includescore{temporalia/ps53-initium-vi-F-auto.tex}

\psalmusEtTranslatio{temporalia/ps53-vi-f.tex}{empty.tex}

\scriptura{Psalmus 84.}

\psalmusEtTranslatio{temporalia/ps84-vi-f.tex}{empty.tex}

\scriptura{Psalmus 116.}

\psalmusEtTranslatio{temporalia/ps116-vi-f.tex}{empty.tex}

\pars{Capitulum}

\rubrica{Capitulum et Versiculus ut Extra Adventum, 
  pg. \pageref{primacapitulum}.}

\rubricaFinisHoraePostNativitatem

\vfill


\pagebreak

% Post Nativitatem.

\hora{Ad Tertiam.} %%%%%%%%%%%%%%%%%%%%%%%%%%%%%%%%%%%%%%%%

\initiumHoraeMinoresAdvent % this IS ok

\pars{Psalmi}

\nativitasAntiphonaII

\scriptura{Psalmus 119.}

\includescore{temporalia/post-nativitatem-tertia/ps119-initium-iii-a3-auto.tex}

\psalmusEtTranslatio{temporalia/post-nativitatem-tertia/ps119-iii-a3.tex}{temporalia/post-nativitatem-tertia/ps119-boh.tex}

\scriptura{Psalmus 120.}

\psalmusEtTranslatio{temporalia/post-nativitatem-tertia/ps120-iii-a3.tex}{temporalia/post-nativitatem-tertia/ps120-boh.tex}

\scriptura{Psalmus 121.}

\psalmusEtTranslatio{temporalia/post-nativitatem-tertia/ps121-iii-a3.tex}{temporalia/post-nativitatem-tertia/ps121-boh.tex}

\vfill

\pars{Capitulum}

\rubrica{Capitulum et Versiculus ut Extra Adventum, 
  pg. \pageref{tertiacapitulum}.}

\rubricaFinisHoraePostNativitatem

\vfill


\pagebreak

% Post Nativitatem.

\hora{Ad Sextam.} %%%%%%%%%%%%%%%%%%%%%%%%%%%%%%%%%%%%%%%%

\initiumHoraeMinoresAdvent % this IS ok

\pars{Psalmi}

\nativitasAntiphonaIII

\scriptura{Psalmus 122.}

\includescore{temporalia/post-nativitatem-sexta/ps122-initium-iv-E-auto.tex}

\psalmusEtTranslatio{temporalia/post-nativitatem-sexta/ps122-iv-e.tex}{temporalia/post-nativitatem-sexta/ps122-boh.tex}


\scriptura{Psalmus 123.}

\psalmusEtTranslatio{temporalia/post-nativitatem-sexta/ps123-iv-e.tex}{temporalia/post-nativitatem-sexta/ps123-boh.tex}


\scriptura{Psalmus 124.}

\psalmusEtTranslatio{temporalia/post-nativitatem-sexta/ps124-iv-e.tex}{temporalia/post-nativitatem-sexta/ps124-boh.tex}



\pars{Capitulum}

\rubrica{Capitulum et Versiculus ut Extra Adventum, 
  pg. \pageref{sextacapitulum}.}

\rubricaFinisHoraePostNativitatem

\vfill


\pagebreak

% Post Nativitatem.

\hora{Ad Nonam.} %%%%%%%%%%%%%%%%%%%%%%%%%%%%%%%%%%%%%%%%

\initiumHoraeMinoresAdvent % this IS ok

\pars{Psalmi}

\nativitasAntiphonaV

\scriptura{Psalmus 125.}

\includescore{temporalia/post-nativitatem-nona/ps125-initium-ii-D-auto.tex}

\psalmusEtTranslatio{temporalia/post-nativitatem-nona/ps125-ii-d.tex}{temporalia/post-nativitatem-nona/ps125-boh.tex}


\scriptura{Psalmus 126.}

\psalmusEtTranslatio{temporalia/post-nativitatem-nona/ps126-ii-d.tex}{temporalia/post-nativitatem-nona/ps126-boh.tex}


\scriptura{Psalmus 127.}

\psalmusEtTranslatio{temporalia/post-nativitatem-nona/ps127-ii-d.tex}{temporalia/post-nativitatem-nona/ps127-boh.tex}


\pars{Capitulum}

\rubrica{Capitulum et Versiculus ut Extra Adventum, 
  pg. \pageref{nonacapitulum}.}

\rubricaFinisHoraePostNativitatem

\vfill


% \pagebreak

%%%%%%%%%%%%%%%%%%%%%%%%%%%%%%%%%%%%%%%%%%%%%%%%%%%%%%%%%%%%%%%%%%%%%%

\divisio{Tempore Paschali.}

% no hour in the headers of pages
\fancyhead[LE]{\thepage\ / }
\fancyhead[RO]{ / \thepage}

\rubrica{Officium B. Mariæ Virginis dicitur sicut extra Adventum,
sed ad Benedictus, ad Magnificat et ad Nunc dimittis, dicitur
Ant. Regina cæli.}

% those special characters are undefined in the italic version of Junicode
\rubrica{Antiphonis autem, \textnormal{℣℣.} et \textnormal{℟℟.} 
non additur in fine Allelúia.}

\vfill



\pagebreak

% colophon
\pagestyle{empty}

\tableofcontents

\vfill

\pagebreak

Fontes. 
Textus secundum 
\textit{Officium parvum beatæ Mariæ Virginis et Officium
defunctorum cum Psalmis gradualibus et pænitentialibus ac Litaniis sanctorum
e Breviario romano a Pio papa X reformato excerpta.}
Editio I juxta typicam III Vaticanam Breviarii romani. 
Ratisbonæ, sumptibus et typis Friderici Pustet 1925. /
Ordo horarum incipiens a Vesperis et non a Matutino est secundum
\textit{Breviarium Romanum.} Ratisbonæ 1927. /
Omnes cantus Horarum cursus diurni atque Completorii 
necnon prima Antiphona Matutini secundum
\textit{Antiphonale sacrosanctæ Romanæ Ecclesiæ pro diurnis horis
SS. D. N. Pii X. Pontificis maximi jussu restitutum et editum.}
Romæ typis Polyglottis Vaticanis 1912. /
Responsoria pro Matutino secundum 
\textit{Nocturnale Romanum. Antiphonale sacrosanctæ Romanæ Ecclesiæ
pro nocturnis horis.}
2002. /
Antiphonæ reliquæ pro Matutino sumptæ sunt de manuscriptis:
\textit{Antiphonarium pro Ecclesia Einsidlensi.}
Einsiedeln, Stiftsbibliothek, Codex 611(89), CANTUS CH-E 611.
(Antiphonæ 2, 3, 7, 8, 9) /
\textit{Antiphonarium Fratrum Minorum Secundum Consuetudinem Romanae Curiae.}
Fribourg/Freiburg, Couvent des Cordeliers/Franziskanerkloster, Ms. 2,
CANTUS CH-Fco 2.
(Antiphonæ 4, 5, 6)

Translationes psalmorum ex
Hejčl Jan: Žaltář čili Kniha žalmů, Praha~1922.

Collaborantes.
Textus latinos cantusque transcripsit et omnem laborem typographicum peregit
Jakub Pavlík. Idem typographus indignus terminationem antiphonæ
\textit{Spiritus Sanctus} sine alleluia composuit, quam in libris invenire
non poterat. /
Psalmos in lingua bohemica de libro supra dicto transcripsit
Barbora Maturová et idem Jakub. /
Filip Srovnal librum istum pr"aeparare mandavit et laborem exprobrationibus
utilissimis comitabatur. /
%Imaginem, qu"ae paginam tituli ornat, Klára Jirsová pinxit.

Instrumenta adhibita.
LuaTeX, %http://www.luatex.org / 
Gregorio, %http://home.gna.org/gregorio /
typi Junicode. %http://junicode.sourceforge.net

\begin{center}
Liber hic imprimis ad usum chori 
\guillemotright Conventus Choralis\guillemotleft\ 
paratus est
et secundum eius consuetudines.
http://www.introitus.cz

\vspace{1cm}

{\large Editio Sancti Wolfgangi \annusEditionis .}

\vspace{2mm}

Series \guillemotright Conventus\guillemotleft, vol. III.

\vspace{1cm}

http://stiwolfgangi.xf.cz

\vfill

\today

\end{center}

\end{document}
