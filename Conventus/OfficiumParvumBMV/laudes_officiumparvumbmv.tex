% LuaLaTeX

\documentclass[a4paper, twoside, 12pt]{article}

\usepackage[latin]{babel} 
\usepackage{ecclesiastic}

\usepackage[landscape, left=3cm, right=1.5cm, top=2cm, bottom=1cm]{geometry}

\usepackage{fontspec}
\setmainfont[Ligatures={Common, TeX}]{Junicode}

% shortcut for Junicode without ligatures (for the Czech texts)
\newfontfamily\nlfont[Ligatures={Common, TeX}]{Junicode}

\usepackage{multicol}
\usepackage{color}
\usepackage{lettrine}
\usepackage{fancyhdr}

% usual packages loading:
\usepackage{luatextra}
\usepackage{graphicx} % support the \includegraphics command and options
\usepackage{gregoriotex} % for gregorio score inclusion
\usepackage{gregoriosyms}
\usepackage{parcolumns}

% Commands used to produce a typical "Conventus" booklet

\newenvironment{titulusOfficii}{\begin{center}}{\end{center}}
\newcommand{\dies}[1]{#1

}
\newcommand{\nomenFesti}[1]{\textbf{\Large #1}

}
\newcommand{\celebratio}[1]{#1

}

\newcommand{\hora}[1]{%
\vspace{0.5cm}{\large \textbf{#1}}

\fancyhead[LE]{\thepage\ / #1}
\fancyhead[RO]{#1 / \thepage}
\addcontentsline{toc}{subsection}{#1}
}

% larger unit than a hora
\newcommand{\divisio}[1]{%
\begin{center}
{\Large \textsc{#1}}
\end{center}
\fancyhead[CO,CE]{#1}
\addcontentsline{toc}{section}{#1}
}

% rubricated inline text
\newcommand{\rubricatum}[1]{\textit{#1}}

% standalone rubric
\newcommand{\rubrica}[1]{\vspace{3mm}\rubricatum{#1}}

\newcommand{\notitia}[1]{\textcolor{red}{#1}}

\newcommand{\scriptura}[1]{\hfill \small\textit{#1}}

\newcommand{\translatioCantus}[1]{\vspace{1mm}%
{\noindent\footnotesize \nlfont{#1}}}

% pruznejsi varianta nasledujiciho - umoznuje nastavit sirku sloupce
% s prekladem
\newcommand{\psalmusEtTranslatioB}[3]{
  \vspace{0.5cm}
  \begin{parcolumns}[colwidths={2=#3}, nofirstindent=true]{2}
    \colchunk{
      \input{#1}
    }

    \colchunk{
      \vspace{-0.5cm}
      {\footnotesize \nlfont
        a
        \input{#2}
      }
    }
  \end{parcolumns}
}

\newcommand{\psalmusEtTranslatio}[2]{
  \psalmusEtTranslatioB{#1}{#2}{8.5cm}
}

% volne misto nad antifonami, kam si zpevaci dokresli neumy
\newcommand{\hicSuntNeumae}{\vspace{0.5cm}}

% prepinani mista mezi notovymi osnovami: pro neumovane a neneumovane zpevy
\newcommand{\cantusCumNeumis}{
  \setgrefactor{17}
  \global\advance\grespaceabovelines by 5mm%
}
\newcommand{\cantusSineNeumas}{
  \setgrefactor{17}
  \global\advance\grespaceabovelines by -5mm%
}

% znaky k umisteni nad inicialu zpevu
\newcommand{\superInitialam}[1]{\gresetfirstlineaboveinitial{\small {\textbf{#1}}}{\small {\textbf{#1}}}}

% pars officii, i.e. "oratio", ...
\newcommand{\pars}[1]{\textbf{#1}}

\newenvironment{psalmus}{
  \setlength{\parindent}{0pt}
  \setlength{\parskip}{5pt}
}{
  \setlength{\parindent}{10pt}
  \setlength{\parskip}{10pt}
}

%%%% Prejmenovat na latinske:
\newcommand{\nadpisZalmu}[1]{
  \hspace{2cm}\textbf{#1}\vspace{2mm}%
  \nopagebreak%

}

% mode, score, translation
\newcommand{\antiphona}[3]{%
\hicSuntNeumae
\superInitialam{#1}
\includescore{#2}

#3
}

%%%% Translations of the proper chants

% HOURS ---

% Translated by Václav Ondráček

\newcommand{\translatioAntI}{\translatioCantus{
Hle v oblacích z nebes přijde Pán s velikou mocí, all.
}}
\newcommand{\translatioAntII}{\translatioCantus{
Město a pevnost naše je Sión, 
Spasitel jej opevní příkopem a zdí. 
Brány otevřete, neboť s námi je Bůh, all.
}}
\newcommand{\translatioAntIII}{\translatioCantus{
Hle Pán se zjeví, to není lež; 
jestliže prodlí, čekej jej, 
vždyť přijde a nebude meškat, all.
}}
\newcommand{\translatioAntIV}{\translatioCantus{
Hory a vrchy před Bohem chválu zpívat budou 
a lesní dříví rukama zatleská, 
neboť Pán a Vládce přijde na věky kralovat, all. all.
}}
\newcommand{\translatioAntV}{\translatioCantus{
Ejhle náš Pán s mocí přijde, by rozzářil oči sluhů svých, all.
}}

\newcommand{\translatioCapituli}{\translatioCantus{}}

\newcommand{\translatioRespVesp}{\translatioCantus{
Ukaž nám Pane své milosrdenství – a uděl nám spasení své.
}}

\newcommand{\translatioRespLaud}{\translatioCantus{
Přijď a vysvoboď nás, Hospodine, Bože silný. – Ukaž svou tvář a budeme spaseni.
}}

\newcommand{\translatioVersus}{\translatioCantus{
Hlas volajícího na poušti: Připravte cestu Páně
Urovnejte jeho cestu.
}}

\newcommand{\translatioAntMagnificatI}{\translatioCantus{
Přijď Pane, navštiv nás v míru, ať srdcem celistvým před tebou zajásáme.
}}

\newcommand{\translatioAntBenedictus}{\translatioCantus{
Když viděl Jan v řetězech Kristovy skutky, poslal svých učedníků dvé říci mu: Ty jsi ten, kdo přichází, nebo jiného čekat máme?
}}

\newcommand{\translatioAntMagnificatII}{\translatioCantus{
Ty jsi ten, kdo přichází, nebo jiného čekat máme? 
Rcete Janovi, co jste viděli: 
Světlo vzchází slepým, mrtví vstávají, chudým se hlásá radostná zvěst, aleluja.
}}

\newcommand{\translatioOrationis}{\translatioCantus{
Probuď Pane naše srdce k přípravě cesty tvého Jednorozeného, 
abychom díky jeho příchodu ti mohli sloužit s očištěnou myslí.
}}

\newcommand{\translatioHymnusVesp}{
Štědrý nebes stvořiteli
Věčný osvítiteli
Kriste vykupiteli
prosby vyslyš věrných milý.

Nad zánikem věku v smrti
smiloval ses, milý choti
a světu spásy lačnému
lék jsi přinesl vinnému.

Když se večer světa chýlí,
vyšels jako ženich milý
z domu matky, panny čestné,
z její komnaty milostné.

Mocí tvojí velmi silnou
kolena se k zemi ohnou
a nebe, země tvorové
se poddávají vůli tvé.

Jenž přicházíš věku soudce, 
prosíme tě světovládce,
uchovej nás v našem čase1 
před úkladnou ranou zhoubce.

Chválu, sílu, čest a slávu,
Bohu Otci, jeho Synu ,
Duchu také rady svaté 
vzdej na věky věků světe.
Amen.
}

\newcommand{\translatioHymnusLaud}{
Jasný hlas k nebi zní
temnoty všechny zahání
a trestá sen, ten prchá, jak
z výšin se Kristův zaskví zrak 

Probuď se mysli strnulá,
hluchotou povstaň zraněná,
nová se hvězda zažíhá,
škody od tebe odnímá.

Shůry přichází Beránek 
polehčit vinným jejich stesk,
v slzách o milost volejme,
smilování si žádejme.

Až pak podruhé zabuší
a svět jeho hrůza zkruší
za hříchy nás nepotrestal
a ochranu svou by nám dal.

Chválu, sílu, čest a slávu,
Bohu Otci, jeho Synu ,
Duchu také rady svaté 
vzdej na věky věků světe.
Amen.
}

% MASS ---

\newcommand{\translatioIntroitus}{\translatioCantus{}}

\newcommand{\translatioGraduale}{\translatioCantus{}}

\newcommand{\translatioAlleluia}{\translatioCantus{}}

\newcommand{\translatioOffertorium}{\translatioCantus{}}

\newcommand{\translatioCommunio}{\translatioCantus{}}

\newcommand{\antiphonaI}{
  \antiphona{VII c}{cantus/amon33/crux_laud_ant1.tex}{\trAntiphonaI}}
\newcommand{\antiphonaII}{
  \antiphona{III a}{cantus/amon33/crux_laud_ant2.tex}{\trAntiphonaII}}
\newcommand{\antiphonaIII}{
  \antiphona{I f}{cantus/amon33/crux_laud_ant3.tex}{\trAntiphonaIII}}
\newcommand{\antiphonaIV}{
  \antiphona{VII c}{cantus/amon33/crux_laud_ant4.tex}{\trAntiphonaIV}}
\newcommand{\antiphonaV}{
  \antiphona{II D}{cantus/amon33/crux_laud_ant5.tex}{\trAntiphonaV}}

\newcommand{\capitulumHocEnim}{
  \scriptura{Phil 2, 5-7.}
  
  \includescore{cantus/amon33/capitulum-HocEnimSentite}

  \trCapitulum
}

\newcommand{\anteOrationem}{
  \rubrica{Ante Orationem, cantatur a Superiore:}

  \pars{Supplicatio Litaniæ.}

  \includescore{\ccommunesAM/supplicatiolitaniae.tex}

  \pars{Oratio Dominica.}

  \includescore{\ccommunesAM/oratiodominica.tex}

  \rubrica{Deinde dicitur ab Hebdomadario:}

  \includescore{\ccommunesAM/dominusvobiscum-solemnis.tex}

  % original rubric from the Antiphonale Monasticum:
  %\rubrica{In choro monialium loco Dominus vobiscum dicitur:}

  \rubrica{Absente sacerdote vel diacono, loco \textnormal{Dóminus vobíscum}
  dicitur:}

  \includescore{\ccommunesAM/domineexaudi.tex}
}

\newcommand{\oratio}{
  \pars{Oratio.}

  \includescore{cantus/amon33/crux_oratio}

  \trOratio
}

\newcommand{\rubricaBenedicamus}{
  \rubrica{Repetito \textnormal{Dóminus vobíscum} in eodem tono ut antea,
  cantatur a cantore:}
}

\newcommand{\postBenedicamus}{
  \rubrica{Postea dicitur voce recta et paululum depressa:}

  \noindent ℣. Fidélium ánimæ per misericórdiam Dei requiéscant in pace.\\
  ℟. Amen.

  \noindent Pater noster. \rubricatum{totum secreto.}

  \rubrica{Deinde, si discedendum est a Choro:}

  \noindent ℣. Dóminus det nobis suam pacem.\\
  ℟. Et vitam ætérnam. Amen.

  \rubrica{Tunc dicitur Antiphona B.~M.~V. pro tempore, 
    cum \textnormal{℣.} et Oratione
    in Tono simplici, pg. \pageref{antiphonafinalis}. Deinde:}

  \noindent ℣. Divínum auxílium máneat semper nobíscum.\\
  ℟. Et cum frátribus nostris abséntibus. Amen.
}

% the two following prayers are from
% Breviarium monasticum, Romae sumptibus Josephi Salviucci 1831, p. lxxxiv.
% http://books.google.cz/books?id=GnFG6Z4Xuc8C&dq=breviarium%20monasticum&hl=cs&pg=PR1#v=onepage&q=breviarium%20monasticum&f=false
% but - this needs to be verified - it seems that exactly the same prayers
% were in the secular as well as monastic breviary, possibly since the
% post-Tridentine reform
\newcommand{\anteOfficiumOratio}{
\rubrica{Oratio dicenda ante inchoationem Divini Officii.}

\lettrine{A}{peri} Dómine os meum ad benedicéndum nomen sanctum tuum:
munda quoque cor meum ab ómnibus vanis, pervérsis, et aliénis
cogitatiónibus;
intelléctum illúmina, afféctum inflámma,
ut digne, atténte ac devóte hoc Offícium recitáre váleam,
et exaudíri mérear ante conspéctum Divínæ Majestátis tuæ.
Per Christum, Dominum nostrum.
℟. Amen.

Dómine, in unióne illíus Divínæ intentiónis,
qua ipse in terris laudes Deo persolvísti,
has tibi Horas \rubricatum{(vel \textnormal{hanc tibi Horam})} persólvo.
}

\newcommand{\postOfficiumOratio}{
\rubrica{
  Orationem sequentem devote post Officium recitantibus
  Leo Papa X. defectus, et culpas in eo persolvendo ex humana
  fragilitate contractas, indulsit, et dicitur flexis genibus.
}

\lettrine{S}{acrosánctæ} et indivíduæ Trinitáti,
crucifixi Domini nostri Jesu Christi humanitáti,
beatissimæ et gloriosíssimæ sempérque Virginis Maríæ
fecúndæ integritáti, 
et omnium Sanctórum universitáti,
sit sempitérna laus, honor, virtus et gloria
ab omni creatúra,
nobísque remíssio omnium peccatórum,
per infiníta sæcula sæculórum.
℟. Amen.

\noindent ℣. Beáta viscera Maríæ Virginis, quæ portavérunt
ætérni Patris Fílium.\\
℟. Et beáta ubera, quæ lactavérunt Christum Dominum.

Pater noster. Ave María.
}


% there are only Laudes in this booklet so it would look weird to have
% each Laudes started with title ``Ad Laudes.'' :)
\renewcommand{\hora}[1]{\vspace{1cm}}

\newcommand{\annusEditionis}{2013}
% directories
\newcommand{\ccommunesAM}{../../cantuscommunes/amon33}
\newcommand{\ccommunesAR}{../../cantuscommunes/arom12}

% TOC: sections only
\setcounter{tocdepth}{1}

\begin{document}

% GREGORIO general settings:
% staff size
\newcommand{\setgrefactorNormal}{%
  \setgrefactor{15}}

\setgrefactorNormal

\newcommand{\spacearoundinitialNormal}{
  % space around the initial.
  \setspaceafterinitial{2.2mm plus 0em minus 0em}
  \setspacebeforeinitial{2.2mm plus 0em minus 0em}
}

\spacearoundinitialNormal

% initial font.
\def\greinitialformat#1{%
{\fontsize{38}{38}\selectfont #1}%
}

\pagestyle{empty}

% multicols setting
\setlength{\columnseprule}{1pt} % cara oddelujici sloupce
\setlength{\columnsep}{20pt} % prostor mezi sloupci

%%%% Titulni stranka
\begin{titulusOfficii}
\nomenFesti{Officium Parvum Beatæ Mariæ Virginis.}
\vspace{3mm}
\nomenFesti{Ad Laudes.}
\end{titulusOfficii}

% graphic

\vspace{2.3cm}

\begin{center}
\includegraphics[height=9cm]{imagines/annuntiatio.jpg}
\end{center}

\vfill

\begin{center}
Ad usum et secundum consuetudines chori \guillemotright Conventus Choralis\guillemotleft.

Editio Sancti Wolfgangi \annusEditionis
\end{center}

\cleardoublepage

\quasiHora{Regulæ generales}

\begin{multicols}{2}

1. Officium parvum B. Mariæ Virg. octo complectitur partes, quas Horas vocant;
Matutinum nempe, Laudes, Primam, Tertiam, Sextam, Nonam, Vesperas, denique
Completorium. Quamlibet autem Horam separatim ab alia recitare licet.

2. Horas hoc fere tempore recitare convenit, nempe Matutinum cum Laudibus 
pridie ab hora secunda pomeridiana, Primam, Tertiam, Sextam et Nonam mane,
Vesperas et Completorium post meridiem.
Tempore autem Quadragesimæ, id est a Sabbato~I in Quadragesima usque ad
Sabbatum sanctum inclusive, satius est Vesperas ante comestionem recitare.

3. Ad lucrandas indulgentias Officio parvo B.~Mariæ Virg. annexas, in publica
recitatione omnes lingua latina uti debent, in privata autem recitatione
qualibet versione fideli et probata uti valent.
Porro recitatio retinenda est adhuc privata, quamvis locum habeat in communi
intra sæpta domus religiosæ, immo et in ipsa Ecclesia vel publico Oratorio
prædictæ domui annexis, sed januis clausis (S.~C. Indulg. 13~Sept. 1888,
28~Aug. 1903, 18~Dec. 1906).

4. Quamvis Officium parvum B.~Mariæ Virg. infra annum sit unum idemque,
exceptis partibus propriis Tempore Adventus et post Nativitatem Domini,
communiter tamen in tria Officia distinguitur:
Primum ergo Officium est dicendum a Matutino diei 3 Februarii usque ad Nonam
Sabbati ante Dominicam I Adventus inclusive, præterquam in Festo Annuntiationis
B.~M.~V., in quo dicitur Officium ut in Adventu.
Secundum recitandum est a Vesperis Sabbati ante Dominicam~I Adventus usque ad
Nonam Vigiliæ Nativitatis Domini inclusive et in Festo Annuntiationis
B.~Mariæ Virg.
Tertium persolvendum est a Vesperis Diei 24 Decembris usque ad Completorium
diei 2 Februarii inclusive.

5. Hoc Officium quovis tempore persolvendum est plane, prout in Breviario
præscribitur. Quamobrem Tempore Passionis, incluso etiam ultimo Triduo 
sacro Majoris Hebdomadæ, non omittitur \rubricatum{Glória Patri} in Invitatorio
ac tertio Responsorio; nec Tempore Paschali numerus Antiphonarum imminuitur,
neque Invitatorio, Antiphonis, Versibus et Responsoriis additur in fine
\rubricatum{Allelúja} (Rubr. gen. Brev. tit.~37, num.~2. S.~R.~C. num. 1334 ad~6).

6. Tempore Paschali, idest a Vesperis Sabbati sancti usque ad Nonam Sabbati
infra Octavam Pentecostes inclusive, Officium parvum B.~Mariæ Virg.
dicitur sicuti per Annum; sed ad \rubricatum{Benedíctus}, ad \rubricatum{Magníficat}
et ad \rubricatum{Nunc dimíttis} dicitur Antiphona \rubricatum{Regína cæli}.

7. In Festo Annuntiationis (a Matutino usque ad Completorium inclusive)
idem Officium recitatur, quod præscribitur Tempore Adventus, et in fine
dicitur Antiphona \rubricatum{Ave, Regína cælórum} vel \rubricatum{Regina cæli}
juxta temporis diversitatem. Si hoc Festum celebretur Tempore Quadragesimæ,
loco \rubricatum{Allelúja} in principio omnium Horarum dicitur
\rubricatum{Laus tibi, Dómine, Rex ætérnæ glóriæ}.

8. Quando Officium parvum B.~Mariæ Virg. recitatur separatim ab Officio divino,
Hymnus \rubricatum{Te Deum} dicitur 
a Nativitate Domini usque ad Sabbatum ante Dominicam Septuagesimæ inclusive
et a Dominica Resurrectionis usque ad Sabbatum ante Dominicam I Adventus
pariter inclusive;
in Adventu autem et a Septuagesima usque ad Pascha nonnisi in Festis B.~Mariæ
Virg. (S.~R.~C. n.~3572 ad~1 et 3659), quæ in Ecclesia universali
celebrantur vel in Kalendario approbato respectivæ Dioecesis vel Instituti
religiosi vel Ecclesiæ sive Oratorii assignantur,
atque in Festo sancti Joseph.
Quando dicitur \rubricatum{Te Deum}, in fine secundi Responsorii adjungitur
℣. \rubricatum{Glória Patri, et Fílio, et Spirítui Sancto}, ac repetitur
altera pars Responsorii.

9. Ante Orationem, etiam quando quis solus recitat Officium,
semper dicitur Versus \rubricatum{Dominus vobiscum} et respondetur
\rubricatum{Et cum spiritu tuo}. Qui Versus non dicitur ab eo, qui non est saltem
in ordine Diaconatus. Si quis autem ad Diaconatus ordinem non pervenerit,
ejus loco dicat Versum \rubricatum{Dómine, exáudi oratiónem meam},
et ei respondetur \rubricatum{Et clamor meus ad te véniat}.
Deinde dicitur \rubricatum{Orémus}, postea Oratio.
Et post ultimam Orationem repetitur ℣.~\rubricatum{Dóminus vobíscum}
vel \rubricatum{Dómine exáudi}. (Rubr. gen. tit.~30, n.~3.)

10. Omnes manu extensa se signent signo crucis a fronte ad pectus et a sinistro
humero ad dexterum
ad \rubricatum{Benedíctus} et ad \rubricatum{Magníficat}. Ceterum servetur
consuetudo quoad sugnum crucis communiter faciendum ad alias Officii partes,
scilicet \rubricatum{Dómine, lábia mea; Convérte nos, Deus; Deus, in adjutórium;
Nunc dimíttis;} et ad benedictionem in fine Completorii.

\end{multicols}


\vfill

\pagebreak

\hora{Mutationes}

\begin{multicols}{2}

Fontes principales textuum officii (vide in fine libri) veteres sunt.
In hac editione mutationes induximus juxta rubricas 
a PP. Pio XII.\footnote{Sacra Congregatio Rituum:
  Decretum generale
  De rubricis ad simpliciorem formam redigendis.
  AAS 47 [1955], 218 et seq.
  Infra \textit{Rubricae 1955.}} 
et Joanne XXIII.\footnote{Sacra Congregatio Rituum:
  Rubricae Breviarii et Missalis Romani.
  AAS 52 [1960], 597 et seq.
  Infra \textit{Codex rubricarum 1960.}} reformatas,
ut sequitur.

Oratio \textit{Ave Maria} ante singulis horis omissa est.\footnote{Rubricae 1955, tit.~IV. §~1.}

Similimodo omissus est Versiculus \textit{Fidelium animæ} in fine horarum;
benedictio \textit{Benedicat et custodiat} in fine completorii.\footnote{ibidem, §~3.}

% Zohlednena nebyla norma Codexu 1960 cap. II §~2:
% Celebratio diei liturgici decurrit per se a Matutino ad Completorium. 
% Sunt tamen dies solemniores, quorum Officium inchoatur a I Vesperis, 
% die praecedenti.



\end{multicols}


\hora{Ante Officium parvum B.~M.~V.}

\anteOfficiumOratio

\hora{Post Officium parvum B.~M.~V.}

\postOfficiumOratio

\pagebreak

\renewcommand{\headrulewidth}{0pt} % no horiz. rule at the header
\fancyhf{}
\pagestyle{fancy}

% in the complete edition this is done by the \hora command
\fancyhead[LE,RO]{\thepage}

\divisio{Extra Adventum.}

\rubrica{Quod dicitur a Matutino diei 3 Februarii usque ad Nonam
Sabbati ante Dominicam I Adventus inclusive, 
præterquam in Festo Annuntiationis B.~M.~V., in quo dicitur,
ut in Adventu, pg. \pageref{tempus:adventus}.}

% Extra Adventum.

\hora{Ad Laudes.} %%%%%%%%%%%%%%%%%%%%%%%%%%%%%%%%%%%%%%%%

\vspace{1cm}

\rubricaLaudesPrincipium

\deusInAdiutorium

\vfill

\pagebreak

\pars{Psalmus 1.}

\antiphonaLaudI

\scriptura{Psalmus 92.}

\includescore{temporalia/extra-adventum-laudes/ps92-initium-vii-a-auto.tex}

\psalmusEtTranslatio{temporalia/extra-adventum-laudes/ps92-vii-a.tex}{temporalia/extra-adventum-laudes/ps92-boh.tex}

\vfill

\pagebreak

\pars{Psalmus 2.}

\antiphonaLaudII

\scriptura{Psalmus 99.}

\includescore{temporalia/extra-adventum-laudes/ps99-initium-viii-G-auto.tex}

\psalmusEtTranslatio{temporalia/extra-adventum-laudes/ps99-viii-g.tex}{temporalia/extra-adventum-laudes/ps99-boh.tex}

\vfill

\pagebreak

\pars{Psalmus 3.}

\antiphonaLaudIII

\scriptura{Psalmus 62.}

\includescore{temporalia/extra-adventum-laudes/ps62-initium-iv-A-auto.tex}

\psalmusEtTranslatio{temporalia/extra-adventum-laudes/ps62-iv-a.tex}{temporalia/extra-adventum-laudes/ps62-boh.tex}

\vfill

\pagebreak

\pars{Canticum trium Puerorum}

\antiphonaLaudIV

\scriptura{Dan. 3, 57-88.56}

\includescore{temporalia/extra-adventum-laudes/dan3-initium-vii-c2-auto.tex}

\psalmusEtTranslatio{temporalia/extra-adventum-laudes/dan3-vii-c2.tex}{temporalia/extra-adventum-laudes/dan3-boh.tex}

\vfill

\pagebreak

\pars{Psalmus 4.}

\antiphonaLaudV

\scriptura{Psalmus 148.}

\includescore{temporalia/extra-adventum-laudes/ps148-initium-i-g2-auto.tex}

\psalmusEtTranslatio{temporalia/extra-adventum-laudes/ps148-i-g2.tex}{temporalia/extra-adventum-laudes/ps148-boh.tex}

\vfill

\pagebreak

\pars{Capitulum}

\label{laudescapitulum}

\scriptura{Cant. 6, 8}

\includescore{cantus/arom12/capitulum-VideruntEam.tex}

\pars{Hymnus}

\superInitialam{II}
\includescore{temporalia/hymnus-OGloriosaVirginum.tex}

\includescore{cantus/arom12/versiculus-BenedictaTu.tex}

\pagebreak

\pars{Canticum Zachariæ}

\rubrica{Extra Tempus Paschale:}

\antiphona{VIII G}{cantus/arom12/antlaudben}{}

\scripturaBenedictus

\includescore{temporalia/extra-adventum-laudes/benedictus-initium-viii-G-auto.tex}

\psalmusEtTranslatioB{temporalia/extra-adventum-laudes/benedictus-viii-g.tex}{temporalia/extra-adventum-laudes/benedictus-boh.tex}{10cm}

\vfill

\rubrica{Sequitur \textnormal{Kýrie eléison} etc. usque ad finem horæ
ut in Vesperis, p. \pageref{vesperaefinis}.}

\pagebreak

\rubrica{Tempore Paschali:}

\paschaAntiphona

\scripturaBenedictus

\includescore{temporalia/tempore-paschali/benedictus-initium-i-D2-auto.tex}

\psalmusEtTranslatioB{temporalia/tempore-paschali/benedictus-i-d2.tex}{temporalia/tempore-paschali/benedictus-boh.tex}{10cm}

\vfill

\pagebreak

\label{laudesfinis}
\inFineHorarumExtraAdventumLaudes

\rubrica{Si terminatur Officium, alioquin, si alia subsequatur Hora,
  in fine ultimæ Horæ dicitur:}

\rubrica{\textnormal{Pater noster,} totum secreto,
  eoque recitato subjungitur}

\noindent ℣. Dóminus det nobis suam pacem.\\
℟. Et vitam ætérnam. Amen.

\rubrica{Et immediate dicitur, cum suis Versu et Oratione,
una ex finalibus beatæ Mariæ Virginis Antiphonis, pro diversitate Temporis,
pg. \pageref{antiphonaefinales}, flexis quidem genibus;
diebus autem Dominicis, a Vesperis Sabbati inclusive et Tempore Paschali,
stando.}

\rubrica{Postea concluditur:}

\noindent ℣. Divínum auxílium maneat semper nobíscum.\\
℟. Amen.


\pagebreak

\divisio{Antiphonae finales B.~M.~V.}


%%%
\rubrica{A Vesperis Sabbati ante Dominicam I. Adventus
usque ad secundas Vesperas Purificationis inclusive.}

\antiphona{V}{\ccommunesAR/an_alma_redemptoris_mater.tex}{\trFinAntAlma}

\rubrica{In Adventu:}

\textusEtTranslatio{
  ℣. Angelus Dómini nuntiávit Maríæ.
  ℟. Et concépit de Spíritu Sancto.
}{\trFinVersAlma}{10cm}

\textusEtTranslatio{
  Orémus.
  Grátiam tuam, quǽsumus Dómine, méntibus nostris infúnde:
  ut qui, Angelo nuntiánte, Christi Fílii tui incarnatiónem cognóvimus,
  per passiónem eius et crucem ad resurrectiónis glóriam perducámur.
  Per eúmdem Christum Dóminum nostrum. 
  ℟. Amen.
}{\trFinOraAlma}{10cm}

%%%
\rubrica{A primis Vesperis Nativitatis Domini et deinceps:}

\textusEtTranslatio{
℣. Post partum Virgo invioláta permansísti.
℟. Dei Génitrix intercéde pro nobis.
}{\trFinVersAlmaNati}{10cm}

\textusEtTranslatio{
Orémus.
Deus, qui salútis ætérnæ, 
beátæ Maríæ virginitáte fecúnda,
humáno géneri prǽmia præstitísti:
tríbue, quǽsumus; ut ipsam pro nobis intercédere sentiámus,
per quam merúimus auctórem vitæ suscípere,
Dóminum nostrum Jesum Christum Fílium tuum.
℟. Amen.
}{\trFinOraAlmaNati}{10cm}

\noindent \hrulefill

%%%
\rubrica{Post Purificationem, id est, a Completorio diei 2. Februarii,
etiam quando transferatur Festum Purificationis B.M.V., usque ad Nonam
Sabbati Sancti inclusive:}


\antiphona{VI}{\ccommunesAR/an_ave_regina_caelorum.tex}{\trFinAntAve}

\vspace{5mm}

\textusEtTranslatio{
℣. Dignáre me laudáre te Virgo sacráta.
℟. Da mihi virtútem contra hostes tuos.
}{\trFinVersAve}{10cm}

\textusEtTranslatio{
Orémus.
Concéde, miséricors Deus, fragilitáti nostræ præsídium:
ut qui sanctæ Dei Genitrícis memóriam ágimus,
intercessiónis ejus auxílio a nostris iniquitátibus resurgámus.
Per eúmdem Christum Dóminum nostrum.
℟. Amen. 
}{\trFinOraAve}{10cm}

\noindent \hrulefill

%%%
\rubrica{A Completorio Sabbati sancti usque ad Nonam Sabbati 
  infra Octavam Pentecostes inclusive:}

\antiphona{VI}{\ccommunesAR/an_regina_caeli.tex}{\trFinAntRegina}

\vspace{2mm}
\textusEtTranslatio{
℣. Gaude et lætáre Virgo María, allelúja.
℟. Quia surréxit Dóminus vere, allelúja.
}{\trFinVersRegina}{10cm}

\vspace{3mm}
\textusEtTranslatio{
Orémus.
Deus, qui per resurrectiónem Fílii tui Dómini nostri Jesu Christi
mundum lætificáre dignátus es:
præsta, quǽsumus; ut per ejus Genitrícem Vírginem Maríam,
perpétuæ capiámus gáudia vitæ.
Per eúmdem Christum Dóminum nostrum.
℟. Amen. 
}{\trFinOraRegina}{10cm}

\noindent \hrulefill

%%%
\rubrica{A Vesperis Sabbati infra Octavam Pentecostes usque ad Nonam Sabbati
ante Adventum inclusive.}

\antiphona{I}{\ccommunesAR/an_salve_regina.tex}{\trFinAntSalve}

\vspace{2mm}
\textusEtTranslatio{
℣. Ora pro nobis sancta Dei Génitrix.
℟. Ut digni efficiámur promissiónibus Christi.
}{\trFinVersSalve}{10cm}

\vspace{3mm}
\textusEtTranslatio{
Orémus.
Omnípotens sempitérne Deus,
qui gloriósæ Vírginis Matris Maríæ corpus et ánimam,
ut dignum Fílii tui habitáculum éffici mererétur,
Spíritu Sancto cooperánte præparásti:
da, ut cujus commemoratióne lætámur,
ejus pia intercessióne ab instántibus malis et a morte perpétua liberémur.
Per eúmdem Christum Dóminum nostrum. 
℟. Amen.
}{\trFinOraSalve}{10cm}

\vfill


\pagebreak

%%%%%%%%%%%%%%%%%%%%%%%%%%%%%%%%%%%%%%%%%%%%%%%%%%%%%%%%%%%%%%%%%%%%%%

\divisio{In Adventu.}

\label{tempus:adventus}

\rubrica{Quod dicitur a Vesperis Sabbati ante Dominicam I Adventus 
usque ad Nonam Vigiliae Nativitatis Domini inclusive.}

% In Adventu.

\hora{Ad Laudes.} %%%%%%%%%%%%%%%%%%%%%%%%%%%%%%%%%%%%%%%%

\rubricaLaudesPrincipium


\pars{Psalmus 1.}

\adventAntiphonaI

\scriptura{Psalmus 92.}

\includescore{temporalia/advent-laudes/ps92-initium-viii-G-auto.tex}

\psalmusEtTranslatio{temporalia/advent-laudes/ps92-viii-g.tex}{temporalia/advent-laudes/ps92-boh.tex}

\vfill

\pagebreak

\pars{Psalmus 2.}

\adventAntiphonaII

\scriptura{Psalmus 99.}

\includescore{temporalia/advent-laudes/ps99-initium-i-g-auto.tex}

\psalmusEtTranslatio{temporalia/advent-laudes/ps99-i-g.tex}{temporalia/advent-laudes/ps99-boh.tex}

\vfill

\pagebreak

\pars{Psalmus 3.}

\adventAntiphonaIII

\scriptura{Psalmus 62.}

\includescore{temporalia/advent-laudes/ps62-initium-viii-G-auto.tex}

\psalmusEtTranslatio{temporalia/advent-laudes/ps62-viii-g.tex}{temporalia/advent-laudes/ps62-boh.tex}

\vfill

\pagebreak

\pars{Canticum trium puerorum.}

\adventAntiphonaIV

\scriptura{Dan. 57-88 et 56.}

\includescore{temporalia/advent-laudes/dan3-initium-i-f-auto.tex}

\psalmusEtTranslatio{temporalia/advent-laudes/dan3-i-f.tex}{temporalia/advent-laudes/dan3-boh.tex}

\vfill

\pagebreak

\pars{Psalmus 4.}

\adventAntiphonaV

\scriptura{Psalmus 148.}

\includescore{temporalia/advent-laudes/ps148-initium-viii-c-auto.tex}

\psalmusEtTranslatio{temporalia/advent-laudes/ps148-viii-c.tex}{temporalia/advent-laudes/ps148-boh.tex}

\vfill

\pagebreak

\pars{Capitulum}

\scriptura{Is. 11,1-2}

\includescore{cantus/arom12/capitulum-EgredieturVirga.tex}

\pars{Hymnus}

\superInitialam{II}
\includescore{temporalia/hymnus-OGloriosaVirginum}

\includescore{cantus/arom12/versiculus-BenedictaTu}

\pagebreak

\pars{Canticum Zachariæ}

\adventAntiphonaSpiritusSanctus

\scripturaBenedictus

\includescore{temporalia/advent-laudes/benedictus-initium-viiisoll-G-auto.tex}

\psalmusEtTranslatioB{temporalia/advent-laudes/benedictus-viiisoll-g.tex}{temporalia/advent-laudes/benedictus-boh.tex}{10cm}

\rubricaFinisHoraeAdvent

% Commemorations would follow, but these were cancelled 
% by the reforms of 1955. 


\vfill

\pagebreak

%%%%%%%%%%%%%%%%%%%%%%%%%%%%%%%%%%%%%%%%%%%%%%%%%%%%%%%%%%%%%%%%%%%%%%

\divisio{Post Nativitatem.}

\rubrica{Quod dicitur a Vesperis diei 24 Decembris usque ad Completorium
diei 2 Februarii inclusive.}

% Post Nativitatem.

\hora{Ad Laudes.} %%%%%%%%%%%%%%%%%%%%%%%%%%%%%%%%%%%%%%%%

\vspace{1cm}

\rubricaLaudesPrincipium

\deusInAdiutorium

\vfill

\pagebreak

\pars{Psalmus 1.}

\nativitasAntiphonaI

\scriptura{Psalmus 92.}

\includescore{temporalia/ps92-initium-vi-F-auto.tex}

\psalmusEtTranslatio{temporalia/ps92-vi-f.tex}{temporalia/ps92-boh.tex}

\vfill

\pagebreak

\pars{Psalmus 2.}

\nativitasAntiphonaII

\scriptura{Psalmus 99.}

\includescore{temporalia/ps99-initium-iii-a3-auto.tex}

\psalmusEtTranslatio{temporalia/ps99-iii-a3.tex}{temporalia/ps99-boh.tex}

\vfill

\pagebreak

\pars{Psalmus 3.}

\nativitasAntiphonaIII

\scriptura{Psalmus 62.}

\includescore{temporalia/ps62-initium-iv-E-auto.tex}

\psalmusEtTranslatio{temporalia/ps62-iv-e.tex}{temporalia/ps62-boh.tex}

\vfill

\pagebreak

\pars{Canticum trium Puerorum}

\nativitasAntiphonaIV

\scriptura{Dan. 3, 57-88.56}

\includescore{temporalia/dan3-initium-i-f-auto.tex}

\psalmusEtTranslatio{temporalia/dan3-i-f.tex}{temporalia/dan3-boh.tex}

\vfill

\pagebreak

\pars{Psalmus 4.}

\nativitasAntiphonaV

\scriptura{Psalmus 148.}

\includescore{temporalia/ps148-initium-ii-D-auto.tex}

\psalmusEtTranslatio{temporalia/ps148-ii-d.tex}{temporalia/ps148-boh.tex}

\vfill


\pars{Capitulum et Hymnus}

\rubrica{Capitulum, Hymnus et Versiculus post Hymnum
  ut extra Adventum, pg. \pageref{laudescapitulum}.}

\pagebreak

\pars{Canticum Zachariæ}

\nativitasAntiphonaBenedictus

\scripturaBenedictus

\includescore{temporalia/benedictus-initium-viii-G-auto}

\psalmusEtTranslatioB{temporalia/benedictus-viii-g.tex}{temporalia/benedictus-boh.tex}{10cm}

\vfill

\rubrica{Sequitur \textnormal{Kýrie eléison} etc. usque ad finem horæ
ut in Vesperis, p. \pageref{nativitasvesperaefinis}.}


%\pagebreak

%%%%%%%%%%%%%%%%%%%%%%%%%%%%%%%%%%%%%%%%%%%%%%%%%%%%%%%%%%%%%%%%%%%%%%

\vspace{2cm}

\divisio{Tempore Paschali.}

% no hour in the headers of pages
\fancyhead[LE]{\thepage\ / }
\fancyhead[RO]{ / \thepage}

\rubrica{Officium B. Mariæ Virginis dicitur sicut extra Adventum,
sed ad Benedictus, ad Magnificat et ad Nunc dimittis, dicitur
Ant. Regina cæli.}

% those special characters are undefined in the italic version of Junicode
\rubrica{Antiphonis autem, \textnormal{℣℣.} et \textnormal{℟℟.} 
non additur in fine Allelúia.}

\vfill


\pagebreak

%%%%%%%%%%%%%%%%%%%%%%%%%%%%%%%%%%%%%%%%%%%%%%%%%%%%%%%%%%%%%%%%%%%%%%

\divisio{Toni Communes.}

\subhora{In principio Horarum.}


\label{tc:deusinadiutorium}

\pars{Tonus festivus.}

\includescore{\ccommunesAR/deusinadiutorium-festivus.tex}

\rubrica{Hoc tono utendum est in 
  %Duplicibus, Semiduplicibus 
  Festis
  et Dominicis
  ad Matutinum, Laudes et Vesperas: et ad Tertiam ante Missam Pontificalem.}

\rubrica{A Completorio Sabbati ante Dominicam Septuagesimæ 
  usque ad Nonam Sabbati sancti inclusive dicitur loco Alleluia:}

% really the same as for the ferial tone
\includescore{\ccommunesAR/laustibi-ferialis.tex}



\pars{Tonus ferialis.}

\includescore{\ccommunesAR/deusinadiutorium-ferialis.tex}

\rubrica{A Completorio Sabbati ante Dominicam Septuagesimæ 
  usque ad Nonam Sabbati sancti inclusive dicitur loco Alleluia:}

\includescore{\ccommunesAR/laustibi-ferialis.tex}

\rubrica{Hoc tono utendum est supradictis diebus 
  ad Primam, Tertiam, Sextam, Nonam, et Completorium;
  et in 
  %Festis Simplicibus 
  Commemorationibus
  et Feriis ad omnes Horas.}


\pagebreak

\subhora{Toni Orationum.}

\pars{Tonus festivus.}

\rubrica{Hic tonus servatur quando Officium est 
  %Duplex, vel Semiduplex,
  de Festo
vel de Dominica, ad Orationes Missæ, in Matutinis, Laudibus, Vesperis,
et ad Tertiam ante Missam pontificalem.}

\includescore{cantus/arom12/oratio-exemplar.tex}

\rubrica{In ipsa Oratione fit primo metrum, deinde flexa.
  In conclusione vero prius flexa, deinde metrum.
  Metrum in Oratione fit plerumque ubi in textu habetur duplex punctum;
  flexa, ubi habetur punctum cum virgula, vel si non adsit,
  ad primam virgulam post metrum ubi permittit sensus; secus, omittitur.}

\rubrica{In conclusione \textnormal{Qui vivis} vel \textnormal{Qui tecum vivit,}
  fit solummodo metrum.}

\pars{Tonus ferialis.}

\rubrica{Diebus supra memoratis ad Horas minores,
  in 
  %Festis Simplicibus 
  Commemorationibus
  et Feriis ad omnes Horas et ad Missam,
  Orationes cantantur in tono, ut aiunt, Feriali, hoc est:
  recta voce a principio ad finem, solummodo sustentando tenorem
  ubi alias fieret metrum et flexa, et in fine.}


\pagebreak

\subhora{Toni \textnormal{℣.} Benedicamus Domino.}

\rubrica{[Notitia editoris: Rubricæ Antiphonalis atque Breviarii Romani
    tacent circa tonos \textnormal{Benedicámus Dómino,} quorum utendum est 
    in Officio parvo. Videtur autem opportunum in festis majoribus
    Beatæ Virginis toni solemnis uti, in diebus vero aliis ad libitum
    toni pro festis Beatæ Mariæ Virginis aut toni ferialis.]}

\label{tc:benedicamus:vesperae}


\pars{In Festis Solemnibus.}

\superInitialam{V}
\includescore{\ccommunesAR/benedicamus-solemnis-laud.tex}


\pars{In Festis B. Mariæ Virginis.}

\rubrica{(In festis B. Mariæ Virginis majoribus cantatur \textnormal{Benedicámus
    Dómino} ut in aliis solemnibus Festis.)}

\superInitialam{I}
\includescore{\ccommunesAR/benedicamus-maria.tex}



\pars{In Feriis.}

\label{tc:benedicamus:minor}

\superInitialam{IV}
\includescore{\ccommunesAR/benedicamus-feria.tex}

\vfill


\pagebreak



% colophon
\pagestyle{empty}

\tableofcontents

\vfill

Fontes. 
Textus secundum 
\textit{Officium parvum beatæ Mariæ Virginis et Officium
defunctorum cum Psalmis gradualibus et pænitentialibus ac Litaniis sanctorum
e Breviario romano a Pio papa X reformato excerpta.}
Editio I juxta typicam III Vaticanam Breviarii romani. 
Ratisbonæ, sumptibus et typis Friderici Pustet 1925. /
%Ordo horarum incipiens a Vesperis et non a Matutino est secundum
%\textit{Antiphonale Romanum.} Ut infra. 
/
Omnes cantus Horarum cursus diurni %atque Completorii 
%necnon prima Antiphona Matutini 
secundum
\textit{Antiphonale sacrosanctæ Romanæ Ecclesiæ pro diurnis horis
SS. D. N. Pii X. Pontificis maximi jussu restitutum et editum.}
Romæ typis Polyglottis Vaticanis 1912. /
%Reliqui cantus pro Matutino secundum 
%\textit{Nocturnale Romanum. Antiphonale sacrosanctæ Romanæ Ecclesiæ
%pro nocturnis horis.}
%2002. /
Textus et toni psalmorum sunt secundum consuetudines Conventus Choralis
ex \textit{Antiphonale Monasticum.} Solesmis 1934. %Psalmi nocturnales,
%qui in Antiphonali Monastico desunt, sumpti sunt de Breviario Monastico.

Translationes psalmorum ex
Hejčl Jan: Žaltář čili Kniha žalmů, Praha~1922.
Translationes canticorum evangelicorum ex Hejčl Jan: 
Od Květné neděle do Bílé soboty. Příručka pro každého, kdo se chce účastniti
bohoslužby ve Svatém týdnu, Olomouc 1928. /
Translationes lectionum secundum 
\textit{Jeruzalémská bible.} Praha - Kostelní Vydří 2009.

Collaborantes.
Textus latinos cantusque transcripsit et omnem laborem typographicum peregit
Jakub Pavlík. Idem typographus indignus terminationem antiphonæ
\textit{Spiritus Sanctus} sine alleluia composuit, quam in libris invenire
non poterat. /
Psalmos in lingua bohemica de libro supra dicto transcripsit
Barbora Maturová et idem typographus. /
Václav Ondráček textus hymnorum antiphonarum etc. 
in linguam bohemicam transtulit. /
Filip Srovnal librum istum præparare mandavit; idem etiam librum examinavit
erroresque correxit. /
Štěpán Němec primam libri partem diligentissime examinavit, errores multos
inveniens. /
Idem Štěpán Němec et Jakub Zentner in dubiis circa rubricas officii
consilio adiuti sunt. /
Terezie Regnerová pinxit imaginem titulum libri ornantem.

Instrumenta adhibita.
LuaTeX, %http://www.luatex.org / 
Gregorio, %http://home.gna.org/gregorio /
typi Junicode. %http://junicode.sourceforge.net

\begin{center}
Liber hic imprimis ad usum chori 
\guillemotright Conventus Choralis\guillemotleft\ 
paratus est
et secundum eius consuetudines.
http://www.introitus.cz

\vspace{1cm}

{\large Editio Sancti Wolfgangi \annusEditionis .}

\vspace{2mm}

Series \guillemotright Conventus\guillemotleft, vol. IV.

\vspace{5mm}

http://stiwolfgangi.xf.cz

\vfill

\today

\end{center}

\end{document}
