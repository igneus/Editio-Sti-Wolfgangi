\newcommand{\colophonFontes}{
  Fontes. 
  Textus secundum 
  \textit{Officium parvum beatæ Mariæ Virginis et Officium
    defunctorum cum Psalmis gradualibus et pænitentialibus ac Litaniis sanctorum
    e Breviario romano a Pio papa X reformato excerpta.}
  Editio I juxta typicam III Vaticanam Breviarii romani. 
  Ratisbonæ, sumptibus et typis Friderici Pustet 1925. /
  Ordo horarum incipiens a Vesperis et non a Matutino est secundum
  \textit{Antiphonale Romanum.} Ut infra. /
  Omnes cantus Horarum cursus diurni atque Completorii 
  necnon prima Antiphona Matutini secundum
  \textit{Antiphonale sacrosanctæ Romanæ Ecclesiæ pro diurnis horis
    SS. D. N. Pii X. Pontificis maximi jussu restitutum et editum.}
  Romæ typis Polyglottis Vaticanis 1912. /
  Reliqui cantus pro Matutino secundum 
  \textit{Nocturnale Romanum. Antiphonale sacrosanctæ Romanæ Ecclesiæ
    pro nocturnis horis.}
  2002. /
  Textus et toni psalmorum sunt secundum consuetudines Conventus Choralis
  ex \textit{Antiphonale Monasticum.} Solesmis 1934. Psalmi nocturnales,
  qui in Antiphonali Monastico desunt, sumpti sunt de Breviario Monastico.
}

% shortenned version of the command above for the extracts
\newcommand{\colophonFontesParvus}{
  Fontes. 
  Textus secundum 
  \textit{Officium parvum beatæ Mariæ Virginis et Officium
    defunctorum cum Psalmis gradualibus et pænitentialibus ac Litaniis sanctorum
    e Breviario romano a Pio papa X reformato excerpta.}
  Editio I juxta typicam III Vaticanam Breviarii romani. 
  Ratisbonæ, sumptibus et typis Friderici Pustet 1925. /
  %Ordo horarum incipiens a Vesperis et non a Matutino est secundum
  %\textit{Antiphonale Romanum.} Ut infra. /
  Omnes cantus Horarum cursus diurni %atque Completorii 
  %necnon prima Antiphona Matutini 
  secundum
  \textit{Antiphonale sacrosanctæ Romanæ Ecclesiæ pro diurnis horis
    SS. D. N. Pii X. Pontificis maximi jussu restitutum et editum.}
  Romæ typis Polyglottis Vaticanis 1912. /
  %Reliqui cantus pro Matutino secundum 
  %\textit{Nocturnale Romanum. Antiphonale sacrosanctæ Romanæ Ecclesiæ
  %  pro nocturnis horis.}
  % 2002. /
  Textus et toni psalmorum sunt secundum consuetudines Conventus Choralis
  ex \textit{Antiphonale Monasticum.} Solesmis 1934. 
  % Psalmi nocturnales,
  % qui in Antiphonali Monastico desunt, sumpti sunt de Breviario Monastico.
}

\newcommand{\colophonTranslationes}{
  Translationes psalmorum ex
  Hejčl Jan: Žaltář čili Kniha žalmů, Praha~1922.
  Translationes canticorum evangelicorum ex Hejčl Jan: 
  Od Květné neděle do Bílé soboty. Příručka pro každého, kdo se chce účastniti
  bohoslužby ve Svatém týdnu, Olomouc 1928. /
  Translationes lectionum secundum 
  \textit{Jeruzalémská bible.} Praha - Kostelní Vydří 2009.
}

\newcommand{\colophonCollaborantes}{
  Collaborantes.
  Textus latinos cantusque transcripsit et omnem laborem typographicum peregit
  Jakub Pavlík. Idem typographus indignus terminationem antiphonæ
  \textit{Spiritus Sanctus} sine alleluia composuit, quam in libris invenire
  non poterat. /
  Psalmos in lingua bohemica de libro supra dicto transcripsit
  Barbora Maturová et idem typographus. /
  Václav Ondráček textus hymnorum antiphonarum etc. 
  in linguam bohemicam transtulit. /
  Filip Srovnal librum istum præparare mandavit; idem etiam librum examinavit
  erroresque correxit. /
  Štěpán Němec primam libri partem diligentissime examinavit, errores multos
  inveniens. /
  Idem Štěpán Němec et Jakub Zentner in dubiis circa rubricas officii
  consilio adiuti sunt. /
  Terezie Regnerová pinxit imaginem titulum libri ornantem.
}

\newcommand{\colophonInstrumenta}{
  Instrumenta adhibita.
  LuaTeX, %http://www.luatex.org / 
  Gregorio, %http://home.gna.org/gregorio /
  typi Junicode. %http://junicode.sourceforge.net
}

\newcommand{\colophonCetera}{
  \begin{center}
    Liber hic imprimis ad usum chori 
    \guillemotright Conventus Choralis\guillemotleft\ 
    paratus est
    et secundum eius consuetudines.
    http://www.introitus.cz

    \vspace{1cm}

    {\large Editio Sancti Wolfgangi \annusEditionis .}

    \vspace{2mm}

    Series \guillemotright Conventus\guillemotleft, vol. IV.

    \vspace{1cm}

    http://stiwolfgangi.xf.cz

    \vfill

    \today
  \end{center}
}

% colophon for the complete version
\newcommand{\colophonMagnus}{
  \colophonFontes

  \colophonTranslationes

  \colophonCollaborantes

  \colophonInstrumenta

  \colophonCetera
}

% colophon for the extracts (laudes/vespers only)
\newcommand{\colophonParvus}{
  \colophonFontesParvus

  \colophonTranslationes

  \colophonCollaborantes

  \colophonInstrumenta

  \colophonCetera
}
