\hora{Mutationes}

\begin{multicols}{2}

Fontes principales textuum officii (vide in fine libri) veteres sunt.
In hac editione mutationes induximus juxta rubricas 
a PP. Pio XII.\footnote{Sacra Congregatio Rituum:
  Decretum generale
  De rubricis ad simpliciorem formam redigendis.
  AAS 47 [1955], 218 et seq.
  Infra \textit{Rubricae 1955.}} 
et Joanne XXIII.\footnote{Sacra Congregatio Rituum:
  Rubricae Breviarii et Missalis Romani.
  AAS 52 [1960], 597 et seq.
  Infra \textit{Codex rubricarum 1960.}} reformatas,
ut sequitur.

Oratio \textit{Ave Maria} ante singulis horis omissa est.\footnote{Rubricae 1955, tit.~IV. §~1.}

Similimodo omissus est Versiculus \textit{Fidelium animæ} in fine horarum;
benedictio \textit{Benedicat et custodiat} in fine completorii.\footnote{ibidem, §~3.}

% Zohlednena nebyla norma Codexu 1960 cap. II §~2:
% Celebratio diei liturgici decurrit per se a Matutino ad Completorium. 
% Sunt tamen dies solemniores, quorum Officium inchoatur a I Vesperis, 
% die praecedenti.



\end{multicols}
