% LuaLaTeX

\documentclass[a4paper, twoside, 12pt]{article}
\usepackage[latin]{babel}
%\usepackage[landscape, left=3cm, right=1.5cm, top=2cm, bottom=1cm]{geometry} % okraje stranky
\usepackage[landscape, a4paper, mag=1166, truedimen, left=2cm, right=1.5cm, top=1.6cm, bottom=0.95cm]{geometry} % okraje stranky

\usepackage{fontspec}
\setmainfont[FeatureFile={junicode.fea}, Ligatures={Common, TeX}, RawFeature=+fixi]{Junicode}
%\setmainfont{Junicode}

% shortcut for Junicode without ligatures (for the Czech texts)
\newfontfamily\nlfont[FeatureFile={junicode.fea}, Ligatures={Common, TeX}, RawFeature=+fixi]{Junicode}

\usepackage{multicol}
\usepackage{color}
\usepackage{lettrine}
\usepackage{fancyhdr}

% usual packages loading:
\usepackage{luatextra}
\usepackage{graphicx} % support the \includegraphics command and options
\usepackage{gregoriotex} % for gregorio score inclusion
\usepackage{gregoriosyms}
\usepackage{wrapfig} % figures wrapped by the text
\usepackage{parcolumns}
\usepackage[contents={},opacity=1,scale=1,color=black]{background}
\usepackage{tikzpagenodes}
\usepackage{calc}
\usepackage{longtable}
\usetikzlibrary{calc}

\setlength{\headheight}{14.5pt}

% Commands used to produce a typical "Conventus" booklet

\newenvironment{titulusOfficii}{\begin{center}}{\end{center}}
\newcommand{\dies}[1]{#1

}
\newcommand{\nomenFesti}[1]{\textbf{\Large #1}

}
\newcommand{\celebratio}[1]{#1

}

\newcommand{\hora}[1]{%
\vspace{0.5cm}{\large \textbf{#1}}

\fancyhead[LE]{\thepage\ / #1}
\fancyhead[RO]{#1 / \thepage}
\addcontentsline{toc}{subsection}{#1}
}

% larger unit than a hora
\newcommand{\divisio}[1]{%
\begin{center}
{\Large \textsc{#1}}
\end{center}
\fancyhead[CO,CE]{#1}
\addcontentsline{toc}{section}{#1}
}

% rubricated inline text
\newcommand{\rubricatum}[1]{\textit{#1}}

% standalone rubric
\newcommand{\rubrica}[1]{\vspace{3mm}\rubricatum{#1}}

\newcommand{\notitia}[1]{\textcolor{red}{#1}}

\newcommand{\scriptura}[1]{\hfill \small\textit{#1}}

\newcommand{\translatioCantus}[1]{\vspace{1mm}%
{\noindent\footnotesize \nlfont{#1}}}

% pruznejsi varianta nasledujiciho - umoznuje nastavit sirku sloupce
% s prekladem
\newcommand{\psalmusEtTranslatioB}[3]{
  \vspace{0.5cm}
  \begin{parcolumns}[colwidths={2=#3}, nofirstindent=true]{2}
    \colchunk{
      \input{#1}
    }

    \colchunk{
      \vspace{-0.5cm}
      {\footnotesize \nlfont
        a
        \input{#2}
      }
    }
  \end{parcolumns}
}

\newcommand{\psalmusEtTranslatio}[2]{
  \psalmusEtTranslatioB{#1}{#2}{8.5cm}
}

% volne misto nad antifonami, kam si zpevaci dokresli neumy
\newcommand{\hicSuntNeumae}{\vspace{0.5cm}}

% prepinani mista mezi notovymi osnovami: pro neumovane a neneumovane zpevy
\newcommand{\cantusCumNeumis}{
  \setgrefactor{17}
  \global\advance\grespaceabovelines by 5mm%
}
\newcommand{\cantusSineNeumas}{
  \setgrefactor{17}
  \global\advance\grespaceabovelines by -5mm%
}

% znaky k umisteni nad inicialu zpevu
\newcommand{\superInitialam}[1]{\gresetfirstlineaboveinitial{\small {\textbf{#1}}}{\small {\textbf{#1}}}}

% pars officii, i.e. "oratio", ...
\newcommand{\pars}[1]{\textbf{#1}}

\newenvironment{psalmus}{
  \setlength{\parindent}{0pt}
  \setlength{\parskip}{5pt}
}{
  \setlength{\parindent}{10pt}
  \setlength{\parskip}{10pt}
}

%%%% Prejmenovat na latinske:
\newcommand{\nadpisZalmu}[1]{
  \hspace{2cm}\textbf{#1}\vspace{2mm}%
  \nopagebreak%

}

% mode, score, translation
\newcommand{\antiphona}[3]{%
\hicSuntNeumae
\superInitialam{#1}
\includescore{#2}

#3
}
 % Often used macros
%%%% Translations of the proper chants

% HOURS ---

% Translated by Václav Ondráček

\newcommand{\translatioAntI}{\translatioCantus{
Hle v oblacích z nebes přijde Pán s velikou mocí, all.
}}
\newcommand{\translatioAntII}{\translatioCantus{
Město a pevnost naše je Sión, 
Spasitel jej opevní příkopem a zdí. 
Brány otevřete, neboť s námi je Bůh, all.
}}
\newcommand{\translatioAntIII}{\translatioCantus{
Hle Pán se zjeví, to není lež; 
jestliže prodlí, čekej jej, 
vždyť přijde a nebude meškat, all.
}}
\newcommand{\translatioAntIV}{\translatioCantus{
Hory a vrchy před Bohem chválu zpívat budou 
a lesní dříví rukama zatleská, 
neboť Pán a Vládce přijde na věky kralovat, all. all.
}}
\newcommand{\translatioAntV}{\translatioCantus{
Ejhle náš Pán s mocí přijde, by rozzářil oči sluhů svých, all.
}}

\newcommand{\translatioCapituli}{\translatioCantus{}}

\newcommand{\translatioRespVesp}{\translatioCantus{
Ukaž nám Pane své milosrdenství – a uděl nám spasení své.
}}

\newcommand{\translatioRespLaud}{\translatioCantus{
Přijď a vysvoboď nás, Hospodine, Bože silný. – Ukaž svou tvář a budeme spaseni.
}}

\newcommand{\translatioVersus}{\translatioCantus{
Hlas volajícího na poušti: Připravte cestu Páně
Urovnejte jeho cestu.
}}

\newcommand{\translatioAntMagnificatI}{\translatioCantus{
Přijď Pane, navštiv nás v míru, ať srdcem celistvým před tebou zajásáme.
}}

\newcommand{\translatioAntBenedictus}{\translatioCantus{
Když viděl Jan v řetězech Kristovy skutky, poslal svých učedníků dvé říci mu: Ty jsi ten, kdo přichází, nebo jiného čekat máme?
}}

\newcommand{\translatioAntMagnificatII}{\translatioCantus{
Ty jsi ten, kdo přichází, nebo jiného čekat máme? 
Rcete Janovi, co jste viděli: 
Světlo vzchází slepým, mrtví vstávají, chudým se hlásá radostná zvěst, aleluja.
}}

\newcommand{\translatioOrationis}{\translatioCantus{
Probuď Pane naše srdce k přípravě cesty tvého Jednorozeného, 
abychom díky jeho příchodu ti mohli sloužit s očištěnou myslí.
}}

\newcommand{\translatioHymnusVesp}{
Štědrý nebes stvořiteli
Věčný osvítiteli
Kriste vykupiteli
prosby vyslyš věrných milý.

Nad zánikem věku v smrti
smiloval ses, milý choti
a světu spásy lačnému
lék jsi přinesl vinnému.

Když se večer světa chýlí,
vyšels jako ženich milý
z domu matky, panny čestné,
z její komnaty milostné.

Mocí tvojí velmi silnou
kolena se k zemi ohnou
a nebe, země tvorové
se poddávají vůli tvé.

Jenž přicházíš věku soudce, 
prosíme tě světovládce,
uchovej nás v našem čase1 
před úkladnou ranou zhoubce.

Chválu, sílu, čest a slávu,
Bohu Otci, jeho Synu ,
Duchu také rady svaté 
vzdej na věky věků světe.
Amen.
}

\newcommand{\translatioHymnusLaud}{
Jasný hlas k nebi zní
temnoty všechny zahání
a trestá sen, ten prchá, jak
z výšin se Kristův zaskví zrak 

Probuď se mysli strnulá,
hluchotou povstaň zraněná,
nová se hvězda zažíhá,
škody od tebe odnímá.

Shůry přichází Beránek 
polehčit vinným jejich stesk,
v slzách o milost volejme,
smilování si žádejme.

Až pak podruhé zabuší
a svět jeho hrůza zkruší
za hříchy nás nepotrestal
a ochranu svou by nám dal.

Chválu, sílu, čest a slávu,
Bohu Otci, jeho Synu ,
Duchu také rady svaté 
vzdej na věky věků světe.
Amen.
}

% MASS ---

\newcommand{\translatioIntroitus}{\translatioCantus{}}

\newcommand{\translatioGraduale}{\translatioCantus{}}

\newcommand{\translatioAlleluia}{\translatioCantus{}}

\newcommand{\translatioOffertorium}{\translatioCantus{}}

\newcommand{\translatioCommunio}{\translatioCantus{}}
 % Czech translations of the proper texts
\newfontface\GreGall{gregall.ttf}
\newfontface\GreGallModern{SGModern.ttf}
\directlua{dofile('gregall.lua')}
\newcommand{\gregallcharno}[3]{{\directlua{
  tex.sprint(gregallparse_neumes("\luaescapestring{#1}", "\luaescapestring{#2}", \luaescapestring{#3}))
}}}
\def\gregallchar{%
  \begingroup %
    \catcode`\~=12{}%
    \fontsize{8}{8}%
    \color{red}%
    \dogregallchar%
}
\def\dogregallchar#1{
    \gregallcharno{#1}{gregall}{0.8}%
  \endgroup %
}
\def\gregallmodchar{%
  \begingroup %
    \catcode`\~=12{}%
    \fontsize{16}{16}%
    \color{red}%
    \dogregallmodchar%
}
\def\dogregallmodchar#1{
    \gregallcharno{#1}{gregallmod}{1.6}%
  \endgroup %
}


\newcommand{\annusEditionis}{2015}

%%%% Vicekrat opakovane kousky

\newcommand{\anteOrationem}{
  \rubrica{Ante Orationem, cantatur a Superiore:}

  \pars{Supplicatio Litaniæ.}

  \includescore{temporalia/supplicatiolitaniae.tex}

  \pars{Oratio Dominica.}

  \includescore{temporalia/oratiodominica.tex}

  \rubrica{Deinde dicitur ab Hebdomadario:}

  \includescore{temporalia/dominusvobiscum-solemnis.tex}

  \rubrica{In choro monialium loco Dominus vobiscum dicitur:}

  \includescore{temporalia/domineexaudi.tex}
}

\setlength{\columnsep}{30pt} % prostor mezi sloupci

%%%%%%%%%%%%%%%%%%%%%%%%%%%%%%%%%%%%%%%%%%%%%%%%%%%%%%%%%%%%%%%%%%%%%%%%%%%%%%%%%%%%%%%%%%%%%%%%%%%%%%%%%%%%%
\begin{document}

% Here we set the space around the initial.
% Please report to http://home.gna.org/gregorio/gregoriotex/details for more details and options
\setspaceafterinitial{2.2mm plus 0em minus 0em}
\setspacebeforeinitial{2.2mm plus 0em minus 0em}

% Here we set the initial font. Change 38 if you want a bigger initial.
% Emit the initials in red.
\def\greinitialformat#1{%
{\color{red}\fontsize{38}{38}\selectfont #1}%
}

\pagestyle{empty}

%%%% Titulni stranka
\begin{titulusOfficii}
\nomenFesti{In Festo Pentecostes.}
\celebratio{Duplex 1. classis.} % puvodne "cum Octava." Oktavy byly ale zruseny
\end{titulusOfficii}

% graphic
\vspace{1.5cm}
\begin{center}
%\includegraphics[width=15cm]{imagines/feriaiv.jpg}
\end{center}

\vfill

\begin{center}
Ad usum et secundum consuetudines chori \guillemotright{}Conventus Choralis\guillemotleft.

Editio Sancti Wolfgangi \annusEditionis
\end{center}

\pagebreak

\renewcommand{\headrulewidth}{0pt} % no horiz. rule at the header
\fancyhf{}
\pagestyle{fancy}

\pars{Oratio ante divinum Officium.}

\lettrine{{\color{red}A}}{peri,} Dómine, os meum ad benedicéndum nomen sanctum tuum:
munda quoque cor meum ab ómnibus vanis, pervérsis, et aliénis
cogitatiónibus:
intelléctum illúmina, afféctum inflámma,
ut digne, atténte ac devóte hoc Offícium recitáre váleam,
et exaudíri mérear ante conspéctum Divínæ Majestátis tuæ.
Per Christum, Dominum nostrum.
\Rbardot{} Amen.

Dómine, in unióne illíus divínæ intentiónis,
qua ipse in terris laudes Deo persolvísti,
has tibi Horas \rubricatum{(vel \textnormal{hanc tibi Horam})} persólvo.

\trOratioAnteOfficium

\vfill

\pars{Oratio post divinum Officium.}

\rubrica{
  Orationem sequentem devote post Officium recitantibus
  Leo Papa X. defectus, et culpas in eo persolvendo ex humana
  fragilitate contractas, indulsit, et dicitur flexis genibus.
}

\lettrine{{\color{red}S}}{acrosánctæ} et indivíduæ Trinitáti,
crucifíxi Dómini nostri Jesu Christi humanitáti,
beatíssimæ et gloriosíssimæ sempérque Vírginis Maríæ
fecúndæ integritáti, 
et ómnium Sanctórum universitáti
sit sempitérna laus, honor, virtus et glória
ab omni creatúra,
nobísque remíssio ómnium peccatórum,
per infiníta sǽcula sæculórum.
\Rbardot{} Amen.

\noindent \Vbardot{} Beáta víscera Maríæ Virginis, quæ portavérunt
ætérni Patris Fílium.\\
\Rbardot{} Et beáta úbera, quæ lactavérunt Christum Dominum.

\rubrica{Et dicitur secreto \textnormal{Pater noster.} et \textnormal{Ave María.}}

\trOratioPostOfficium

\vfill

\hora{In Vesperis.} %%%%%%%%%%%%%%%%%%%%%%%%%%%%%%%%%%%%%%%%%%%%%%%%%%%%%
\sideThumbs{Vesperæ}

\cantusSineNeumas

\label{deusinadiutorium}

\includescore{temporalia/deusinadiutorium-solemnis.tex}

\vfill
\pagebreak

\cantusCumNeumis

\pars{Psalmus 1.} \scriptura{\textbf{H270}}

\antiphona{III a2}{temporalia/ant1.tex}

\trAntI

\cantusSineNeumas

\scriptura{Psalmus 109.}

\includescore{temporalia/ps109-initium-iii-a2-auto.tex}

\psalmusEtTranslatioT{temporalia/ps109-comb.tex}{10cm}

\vfill
\pagebreak

\pars{Psalmus 2.} \scriptura{\textbf{H270?}}

\antiphona{VIII G}{temporalia/ant2.tex}

\trAntII

\scriptura{Psalmus 110.}

\includescore{temporalia/ps110-initium-viii-G-auto.tex}

\psalmusEtTranslatioT{temporalia/ps110-comb.tex}{10cm}

\vfill
\pagebreak

\pars{Psalmus 3.} \scriptura{\textbf{H270}}

\antiphona{VIII G}{temporalia/ant3.tex}

\trAntIII

\scriptura{Psalmus 111.}

\includescore{temporalia/ps111-initium-viii-G-auto.tex}

\psalmusEtTranslatioT{temporalia/ps111-comb.tex}{10cm}

\vfill
\pagebreak

\pars{Psalmus 4.} \scriptura{\textbf{H271}}

\antiphona{VII c2}{temporalia/ant5.tex}

\trAntIV

\scriptura{Psalmus 112.}

\includescore{temporalia/ps112-initium-vii-c2-auto.tex}

\psalmusEtTranslatioT{temporalia/ps112-comb.tex}{10cm}

\vfill
\pagebreak

\raggedcolumns

% Capitulum. %%%
\cantusSineNeumas

\label{capitulum}
\pars{Capitulum.} \scriptura{Act. 2, 1 - 2}

\includescore{temporalia/capitulum-CumComplerentur.tex}

% preklad Jeruz. bible
\trCapituli

\vspace{1cm}
\pars{Responsorium breve.}\rubrica{ - in I. vesperis:}

\superInitialam{VI}
\includescore{temporalia/resp1v.tex}

%\trRespVesp2

\pars{Responsorium breve.}\rubrica{ - in II. vesperis:}

\superInitialam{VI}
\includescore{temporalia/resp2v.tex}

%\trRespVesp2

\vfill
\pagebreak

% Hymnus. %%%
\pars{Hymnus.}

\superInitialam{VIII}
\includescore{temporalia/hym-VeniCreator.tex}
%\begin{translatioMulticol}{4}
Duchu Tvůrce, přijď, navštiv nás,\\
přijď do srdce všech věrných svých\\
a naplň je svou milostí,\\
vždyť ty sám jsi nás utvořil.\\
\\
Ty se nazýváš Těšitel,\\
Bůh nejvyšší nám tebe dal,\\
tys pramen živý, lásky žár,\\
pomazání jsi duchovní.\columnbreak

Dárce darů jsi sedmera,\\
prst pravice jsi Otcovy,\\
ty jsi ten Otcem slíbený,\\
správně dáváš nám promlouvat.\\
\\
V duši světlo nám rozžehni\\
a do srdce nám lásku vlej\\
a naše tělo ubohé\\
stále silou svou posiluj.\columnbreak

Nepřítele pryč odežeň\\
a ve světě svůj nastol mír,\\
nás cestou správnou vždycky veď,\\
ať nás mine vše škodlivé.\\
\\
Boha Otce nás nauč znát,\\
dej vyznávat nám Ježíše\\
a tobě, Otci i Synu\\
vírou se vždy víc otvírat.\columnbreak

Sláva Otci, i Synu\\
zrozenému, který z mrtvých\\
vstal, i Utěšiteli,\\
na věky věků.\\
Amen.
\end{translatioMulticol}


\vfill

\pars{Versus.}\rubrica{ - in I. vesperis:}

% Versus. %%%
\includescore{temporalia/versus-repleti.tex}

%\noindent \trVersus

\vfill

\pars{Versus.}\rubrica{ - in II. vesperis:}

% Versus. %%%
\includescore{temporalia/versus-loquebantur.tex}

%\noindent \trVersus

\vfill
\pagebreak

\cantusCumNeumis

\pars{Canticum B. Mariæ V.}\rubrica{ - in I. vesperis:} \scriptura{\textbf{H267}}

\antiphona{I D*}{temporalia/ant-magn-vesp1.tex}

%\trAntMagnificatI

\vfill

\scriptura{Lucæ 1, 46-55}

\cantusSineNeumas
\includescore{temporalia/magnificat-initium-i-D_.tex}

\psalmusEtTranslatioT{temporalia/magnificat-comb.tex}{10cm}

%\includescore{temporalia/ant-magn-vesp1.tex} % repeat the antiphon - new page

\vfill
\pagebreak

\pars{Canticum B. Mariæ V.}\rubrica{ - in II. vesperis:} \scriptura{\textbf{Sg. 388 p. 247}}
\label{magnificatIIvesp}

\cantusCumNeumis

\antiphona{I D*}{temporalia/ant-magn-vesp2.tex}

%\trAntMagnificatII

\vfill

\scriptura{Lucæ 1, 46-55}

\cantusSineNeumas
\includescore{temporalia/magnificat-initium-i-D_.tex}

% maly svindl, ale prizvukova struktura VIIIsoll-G2 a Isoll-D* je stejna
\psalmusEtTranslatioT{temporalia/magnificat-comb.tex}{10cm}

\includescore{temporalia/ant-magn-vesp2.tex} % repeat the antiphon - new page

\vfill
\pagebreak

\label{oratio}
\anteOrationem

\pagebreak

% Oratio. %%%
\pars{Oratio.}

\includescore{temporalia/oratio.tex}
\trOrationis

\vspace{1cm}
\rubrica{Hebdomadarius dicit iterum Dominus vobiscum. Postea cantatur a cantore:}
\vspace{2mm}

\rubrica{In I. Vesperis:}

%\includescore{temporalia/benedicamus-solemnis-1vesp.tex}

\rubrica{In II. Vesperis:}

%\includescore{temporalia/benedicamus-solemnis-2vesp.tex}

\vfill
\pagebreak

\hora{Ad Completorium.} %%%%%%%%%%%%%%%%%%%%%%%%%%%%%%%%%%%%%%%%%%%%%%%%%%%%%%%%%%
\sideThumbs{{\scriptsize{}Completorium}}

\rubrica{Lector petit benedictionem, dicens:}

\includescore{temporalia/jubedomnebenedicere.tex}

\trJubeDomne

\vfill

\pars{Benedictio.}

\includescore{temporalia/benedictio-noctemquietam.tex}

\trComplBenedictio

\vfill

\pars{Lectio brevis.} \scriptura{1 Petr. 5, 8-9}

\includescore{temporalia/lectiobrevis-fratressobrii.tex}

\trComplLectioBr

\vfill

\noindent \Vbardot{} Adjutórium nostrum in nómine Dómini. \Rbardot{} Qui fecit cælum, et terram.

\vfill

\noindent Pater noster \rubricatum{quod dicitur totum secreto.}

\vfill
\pagebreak

\pars{Confessio.}

\noindent Confíteor Deo omnipoténti, beátæ Maríæ semper Vírgini, beáto
Michaéli Archángelo, beáto Joánni Baptístæ, sanctis Apóstolis Petro
et Paulo, ómnibus Sanctis, et vobis fratres: quia peccávi nimis cogitatióne,
verbo et ópere: mea culpa, mea culpa, mea máxima culpa.
Ídeo precor beátam Maríam semper Vírginem, beátum Michaélum
Archángelum, beátum Joánnem Baptístam, sanctos Apóstolos Petrum
et Paulum, omnes Sanctos, et vos fratres, oráre pro me ad Dóminum
Deum nostrum.

\vfill

\noindent \Vbardot{} Misereátur nostri omnípotens Deus, et, dimíssis peccátis nostris, perdúcat
nos ad vitam ætérnam. \Rbardot{} Amen.

\vfill

\noindent \Vbardot{} Indulgéntiam, absolutiónem et remissiónem peccatórum nostrórum tríbuat nobis
omnípotens et miséricors Dóminus. \Rbardot{} Amen.

\vfill

\rubrica{Et facta absolutione dicitur:}

\includescore{temporalia/convertenosdeus.tex}

\vfill

\includescore{temporalia/deusinadiutorium-communis.tex}

\vfill
\pagebreak

\pars{Psalmus 1.}

\antiphona{VIII G}{temporalia/ant-alleluia.tex}

\trComplAntI

\scriptura{Psalmus 4.}

\includescore{temporalia/ps4-initium-viii-G-auto.tex}

\psalmusEtTranslatioT{temporalia/ps4-comb.tex}{10cm}

\vfill
\pagebreak

\pars{Psalmus 2.} \scriptura{Psalmus 90.}

\psalmusEtTranslatioT{temporalia/ps90-comb.tex}{10cm}

\pagebreak

\pars{Psalmus 3.} \scriptura{Psalmus 133.}

\psalmusEtTranslatioT{temporalia/ps133-comb.tex}{10cm}

\vfill

\antiphona{VIII G}{temporalia/ant-alleluia.tex}

\vfill

\pars{Hymnus.}

\antiphona{VIII}{temporalia/hym-TeLucis.tex}
\begin{translatioMulticol}{3}
Než světlo zhasne prosíme\\
Tebe tvůrce všech pokorně,\\
abys nám ve své milosti\\
byl ochranou a pomocí.\columnbreak

Ať vzdáleny jsou od nás sny\\
a těžké noční přízraky.\\
Zdrť našeho nepřítele,\\
těla poskvrn ať ujdeme.\columnbreak

To, Otče mocný nejvýše,\\
dej skrze Krista Ježíše,\\
s nímž a se svatým Duchem též\\
po všechny věky kraluješ.\\
Amen.
\end{translatioMulticol}


\pagebreak

\pars{Capitulum.} \scriptura{Jer. 14, 9}

\includescore{temporalia/capitulum-tuautem.tex}

% preklad Jeruz. bible
\trComplCapituli

\vfill

\pars{Responsorium breve.} \scriptura{Psalmus 30, 6}

\superInitialam{VI}
\includescore{temporalia/resp-inmanus.tex}

\trRespCompl
\vfill

\pars{Versus.} \scriptura{Psalmus 16, 8}

\includescore{temporalia/versus-custodi.tex}

\noindent \trComplVersus

\vfill
\pagebreak

\cantusCumNeumis

\pars{Canticum Simeonis.}

\antiphona{III a}{temporalia/ant-salvanos-antiquo.tex}

\trAntSalvaNos

\scriptura{Lucæ 2, 29-32}

\includescore{temporalia/nuncdimittis-initium-iii-a-auto.tex}

\psalmusEtTranslatioT{temporalia/nuncdimittis-comb.tex}{10cm}

\vfill
\pagebreak

\pars{Oratio.}

\cantusSineNeumas

\includescore{temporalia/oratio-visita.tex}

\trComplOrationis

\vfill

\includescore{temporalia/domineexaudi.tex}

\vfill

\includescore{temporalia/benedicamus-minor.tex}

\vfill

\pars{Benedictio.}

\noindent Benedícat et custódiat nos omnípotens et miséricors Dóminus, \gredagger{}
Pater, et Fílius, et Spíritus Sanctus. \Rbardot{} Amen.

\vfill
\pagebreak

\pars{Antiphona finalis B. M. V.}

\antiphona{V}{temporalia/an_regina_caeli_simplex.tex}

%\trAlmaRedemptoris

\vfill
\pagebreak

\hora{Ad Matutinum.} %%%%%%%%%%%%%%%%%%%%%%%%%%%%%%%%%%%%%%%%%%%%%%%%%%%%%%%%%%
\sideThumbs{Matutinum}

\pars{Psalmus 1.} \scriptura{Act. 2, 2; \textbf{H268}}

\antiphona{VIII c}{temporalia/matant1.tex}

%\trMatAntI

\scriptura{Psalmus 47.}

\includescore{temporalia/ps47-initium-viii-c-auto.tex}

\psalmusEtTranslatioT{temporalia/ps47-comb.tex}{10cm}

\includescore{temporalia/matant1.tex} % repeat the antiphon - new page

\vfill
\pagebreak

\pars{Psalmus 2.} \scriptura{Psalmus 67, 29.30; \textbf{H268}}

\antiphona{VIII c}{temporalia/matant2.tex}

%\trMatAntII

\scriptura{Psalmus 67.}

\includescore{temporalia/ps67-initium-viii-c-auto.tex}

\psalmusEtTranslatioT{temporalia/ps67-comb.tex}{10cm}

\includescore{temporalia/matant2.tex} % repeat the antiphon - new page

\vfill
\pagebreak

\pars{Psalmus 3.} \scriptura{Psalmus 103, 30; \textbf{H268}}

\antiphona{VIII c}{temporalia/matant3.tex}

%\trMatAntIII

\scriptura{Psalmus 103.}

\includescore{temporalia/ps103-initium-viii-c-auto.tex}

\psalmusEtTranslatioT{temporalia/ps103-comb.tex}{10cm}

\vfill
\pagebreak

\pars{Versus.} \scriptura{Sap. 1, 7}

\includescore{temporalia/versus-spiritus.tex}

%\noindent \trMatVersusI

\vfill
\pagebreak

\pars{Responsorium 1.} \scriptura{\Rbar{} Act. 2, 1.2 \Vbar{} ibidem; \textbf{H269}}

%\responsorium{III}{temporalia/matresp1.tex}{\trMatRespI}
\responsorium{III}{temporalia/matresp1.tex}{}

\vfill
\pagebreak

\pars{Responsorium 2.} \scriptura{\Rbar{} Act. 2, 4.6 \Vbar{} ibid. 2, 4.11; \textbf{H269}}

%\responsorium{II}{temporalia/matresp2.tex}{\trMatRespII}
\responsorium{II}{temporalia/matresp2.tex}{}

\vfill
\pagebreak

\hora{Ad Laudes.} %%%%%%%%%%%%%%%%%%%%%%%%%%%%%%%%%%%%%%%%%%%%%%%%%%%%%%%%%%
\sideThumbs{Laudes}

% Psalmi festivi (AM33, pg. 721):
% 92, 99, 62, Dan3, 148+149+150

\vspace{1cm}
\includescore{temporalia/deusinadiutorium-communis.tex}
\vspace{1cm}

\vfill
\pagebreak

\cantusSineNeumas

\pars{Psalmus 1.} \scriptura{Psalmus 66.}

\includescore{temporalia/ps66-initium-dir-auto.tex}

\psalmusEtTranslatioT{temporalia/ps66-comb.tex}{10cm}

\vfill
\pagebreak

\pars{Psalmus 2.} \scriptura{\textbf{H270}}

\antiphona{III a2}{temporalia/ant1.tex}

\trAntI

\scriptura{Psalmus 92.}

\includescore{temporalia/ps92-initium-iii-a2-auto.tex}

\psalmusEtTranslatioT{temporalia/ps92-comb.tex}{10cm}

\vfill
\pagebreak

\pars{Psalmus 3.} \scriptura{\textbf{H270}}

\antiphona{VIII G}{temporalia/ant2.tex}

\trAntII

\scriptura{Psalmus 99.}

\includescore{temporalia/ps99-initium-viii-G-auto.tex}

\psalmusEtTranslatioT{temporalia/ps99-comb.tex}{10cm}

\vfill
\pagebreak

\pars{Psalmus 4.} \scriptura{\textbf{H270}}

\antiphona{VIII G}{temporalia/ant3.tex}

\trAntIII

\scriptura{Psalmus 62.}

\includescore{temporalia/ps62-initium-viii-G-auto.tex}

\psalmusEtTranslatioT{temporalia/ps62-comb.tex}{10cm}

\vfill
\pagebreak

\pars{Psalmus 5.} \scriptura{\textbf{H270}}

\antiphona{I a2}{temporalia/ant4.tex}

\trAntIV

\scriptura{Canticum trium puerorum, Dan. 3, 57-88 et 56.}

\includescore{temporalia/dan3-initium-i-a2-auto.tex}

\psalmusEtTranslatioT{temporalia/dan3-comb.tex}{10cm}

\rubrica{Hic non dicitur Gloria Patri, neque Amen.}
\vspace{1cm}

\hicSuntNeumae
\includescore{temporalia/ant4.tex} % repeat the antiphon - new page

\vfill
\pagebreak

\pars{Psalmus 6.} \scriptura{\textbf{H271}}

\antiphona{VII c2}{temporalia/ant5.tex}

\trAntV

%
\scriptura{Psalmus 148.}

\includescore{temporalia/ps148-initium-vii-c2-auto.tex}

\newlength{\psVItransW}
\setlength{\psVItransW}{10.5cm}

\psalmusEtTranslatioT{temporalia/ps148-comb.tex}{10cm}

\rubrica{Hic non dicitur Gloria Patri.}

%
\scriptura{Psalmus 149.}

\includescore{temporalia/ps149-initium-vii-c2-auto.tex}

\psalmusEtTranslatioT{temporalia/ps149-comb.tex}{10cm}

\rubrica{Hic non dicitur Gloria Patri.}

\vfill
\pagebreak

%
\scriptura{Psalmus 150.}

\includescore{temporalia/ps150-initium-vii-c2-auto.tex}

\psalmusEtTranslatioT{temporalia/ps150-comb.tex}{10cm}

\includescore{temporalia/ant5.tex} % repeat the antiphon - new page

\vfill
\pagebreak

\cantusSineNeumas

\pars{Capitulum.} \scriptura{Act. 2, 1 - 2}

\includescore{temporalia/capitulum-CumComplerentur.tex}

% preklad Jeruz. bible
\trCapituli

\vspace{1cm}
\pars{Responsorium breve.}

\superInitialam{VI}
\includescore{temporalia/respl.tex}

\trRespLaudes
\vfill
\pagebreak

\pars{Hymnus.}

\includescore{temporalia/hym-BeataNobis.tex}
%\begin{translatioMulticol}{4}
Přinesl nám den radosti\\
roční běh času opět dnes,\\
když duch Pomocník září svou\\
kruh učedníků prostoupil.\\
\\
Za třepotání plamenů\\
uchvátil tvářnost jazyka,\\
aby z něj slova skanula,\\
jež láskou zapalovala.\columnbreak

Hovoří všemi jazyky,\\
strachem se chvějí zástupy,\\
soudíce, že jsou opilí,\\
kdo Duchem jsou naplněni.\\
\\
A to vše se přihodilo\\
když skončil čas velikonoc\\
ve svaté lhůtě, za kterou\\
káže vše Zákon odpustit.\columnbreak

Nyní Tě Bože přesvatý\\
s tváří skloněnou prosíme,\\
abys nám z nebes seslav je\\
tajemné dary ducha dal.\\
\\
Srdce kdysi posvěcené\\
naplnil jsi svou milostí,\\
odpusť nám naše přečiny\\
pokojný čas též dopřej nám.\columnbreak

Otci a Pánu sláva buď,\\
i Synu, který z mrtvých vstal,\\
i Duchu přímluv a pomoci\\
po všechny věků okruhy.\\
Amen.
\end{translatioMulticol}


\vfill

\pars{Versus.}

% Versus. %%%
\includescore{temporalia/versus-repleti.tex}

\noindent \trVersusLaudes

\vfill
\pagebreak

\cantusCumNeumis

\pars{Canticum Zachariæ.} \scriptura{\textbf{H271}}

\antiphona{VII a}{temporalia/ant-ben-laud.tex}

\trAntBenedictus

\scriptura{Lucæ 1, 68-79}

\includescore{temporalia/benedictus-initium-viisoll-a-auto.tex}

\psalmusEtTranslatioT{temporalia/benedictus-comb.tex}{10cm}

\includescore{temporalia/ant-ben-laud.tex} % repeat the antiphon - new page

\vfill
\pagebreak

\cantusSineNeumas

\anteOrationem

\pagebreak

% Oratio. %%%
\pars{Oratio.}

\includescore{temporalia/oratio.tex}
\trOrationis

\vspace{1cm}
\rubrica{Hebdomadarius dicit iterum Dominus vobiscum. Postea cantatur a cantore:}
\vspace{2mm}

%\includescore{temporalia/benedicamus-solemnis-laud.tex}

\vfill
\pagebreak

\hora{Ad Missam} %%%%%%%%%%%%%%%%%%%%%%%%%%%%%%%%%%%%%%%%%%%%%%%%%%%%%
\sideThumbs{Missa}

\pars{Antiphona ad introitum.} \scriptura{\textbf{E255}}

\cantusCumNeumis

\antiphona{VIII}{temporalia/introitus-SpiritusDomini.tex}

%\trIntroitus

\vfill

\pars{Kyrie VIII \textit{(De angelis)}.} \scriptura{XV. - XVI. s.}

\superInitialam{V}
\includescore{temporalia/viii-kyrie.tex}

\vfill
\pagebreak

\pars{Gloria VIII.} \scriptura{XVI. s.}

\superInitialam{V}
\includescore{temporalia/viii-gloria.tex}

\vfill
\pagebreak

%\pars{Graduale.} \scriptura{\textbf{C46}}

%\antiphona{V}{temporalia/graduale-OmnesDeSaba.tex}

%\trGraduale

\vfill
\vspace{2cm}

\pars{Alleluia.} \scriptura{\textbf{C117}}

\antiphona{IV}{temporalia/alleluia-EmitteSpiritum.tex}

%\trAlleluia

\vfill
\vspace{2cm}

\pars{Alleluia.} \scriptura{\textbf{Sg. 376 p. 296}}

\antiphona{II}{temporalia/alleluia-VeniSancte.tex}

%\trAlleluia

\vfill
\pagebreak

\pars{Sequentia.}

\antiphona{I}{temporalia/seq-VeniSancte.tex}

\vfill
\pagebreak

\vfill

\pars{Credo III.} \scriptura{XVII. s.}

\superInitialam{V}
\includescore{temporalia/credo-iii.tex}

\vfill

\vspace{2.6cm}

\pars{Offertorium.} \scriptura{\textbf{E256}}

\antiphona{IV}{temporalia/offertorium-ConfirmaHoc.tex}

\vfill
\pagebreak

%\trOffertorium

\vfill

\pars{Sanctus VIII.} \scriptura{(XI) XII. s.}

\superInitialam{VI}
\includescore{temporalia/viii-sanctus.tex}

\vspace{1cm}
\pars{Agnus Dei VIII.} \scriptura{XV. s.}

\superInitialam{VI}
\includescore{temporalia/viii-agnusdei.tex}

\vfill
\pagebreak

\pars{Communio.} \scriptura{\textbf{E257}}

\antiphona{VII}{temporalia/communio-FactusEst.tex}

%\trCommunio

%\scriptura{Psalmus 71., 1.2.3.7.8.10.11.12.17ab.17cd.18}

\cantusSineNeumas
%\includescore{temporalia/communio-versus-DeusJudicium-initium.tex}

%\psalmusEtTranslatioT{temporalia/communio-versus-DeusJudicium-comb.tex}{10cm}
%\begin{psalmus}
\textit{\color{red}2.} Judicáre pópulum tu\textit{um} \textit{in} \textit{ju}\textbf{stí}tia,~\grestar{}
et páuperes tuos \textbf{in} judício. \Abardot{}

\textit{\color{red}3.} Suscípiant mon\textit{tes} \textit{pa}\textit{cem} \textbf{pó}pulo~:~\grestar{}
et colles \textbf{jus}títiam. \Abardot{}

\textit{\color{red}4.} Oriétur in diébus ejus justítia, et abun\textit{dán}\textit{ti}\textit{a} \textbf{pa}cis~:~\grestar{}
donec aufe\textbf{rá}tur luna. \Abardot{}

\textit{\color{red}5.} Et dominábitur a mari \textit{us}\textit{que} \textit{ad} \textbf{ma}re~:~\grestar{}
et a flúmine usque ad términos or\textbf{bis} terrárum. \Abardot{}

\textit{\color{red}6.} Reges Tharsis, et ínsulæ \textit{mú}\textit{ne}\textit{ra} \textbf{óf}ferent~:~\grestar{}
reges Arabum et Saba do\textbf{na} addúcent. \Abardot{}

\textit{\color{red}7.} Et adorábunt eum om\textit{nes} \textit{re}\textit{ges} \textbf{ter}ræ~:~\grestar{}
omnes Gentes sér\textbf{vi}ent ei~: \Abardot{}

\textit{\color{red}8.} Quia liberábit páupe\textit{rem} \textit{a} \textit{po}\textbf{tén}te~:~\grestar{}
et páuperem, cui non e\textbf{rat} adjútor. \Abardot{}

\textit{\color{red}9.} Sit nomen ejus bene\textit{dí}\textit{ctum} \textit{in} \textbf{sǽ}cula~:~\grestar{}
ante solem pérmanet \textbf{no}men ejus. \Abardot{}

\textit{\color{red}10.} Et benedicéntur in ipso om\textit{nes} \textit{tri}\textit{bus} \textbf{ter}ræ~:~\grestar{}
omnes Gentes magnifi\textbf{cá}bunt eum. \Abardot{}

\textit{\color{red}11.} Benedíctus Dómi\textit{nus}, \textit{De}\textit{us} \textbf{Is}raël,~\grestar{}
qui facit mira\textbf{bí}lia solus. \Abardot{}
\end{psalmus}


\vfill
\pagebreak

\newpage
\RemoveSideThumbs
\pagestyle{empty}

%%% COLOPHON
\vfill

Fontes.
Textus et cantus officii divini secundum
Antiphonale Sacrosanctæ Romanæ Eclesiæ Pro Diurnis Horis, Romæ 1912
et Nocturnale Romanum, 2002, præter: psalmi 149 et 150 post
psalmum 148 in Laudibus additi secundum Antiphonale Monasticum pro Diurnis Horis,
Solesmis 1934; lectio sancti Evangelii et hymnus Te Decet Laus post hymnum
Ambrosianum additi secundum ritum monasticum vetum; responsorium breve
in Laudibus et Vesperis additum secundum Antiphonale Monasticum. /
Textus et cantus missæ secundum
Graduale triplex, Solesmis 1979. /
Translatio capituli et lectionis sumpta est ex:
Jeruzalémská bible, Praha-Kostelní Vydří 2009. /
Translationes psalmorum ex
Hejčl Jan: Žaltář čili Kniha žalmů, Praha 1922. /
Neumæ super canto missæ de codicibus Cantatorium, Stiftsbibl. 359 et Einsiedeln,
Stiftsbibl. 121 et neumæ super canto officii divini de codice Hartker,
Stiftsbibl. 390.

Collaborantes.
Textus latinos cantusque transcripsit et omnem laborem typographicum peregit
Jakub Jelínek et Jakub Pavlík. /
Psalmos in lingua bohemica de libro supra dicto transcripsit
Barbora Maturová et idem Jakub Jelínek. /
Václav Ondráček textus hymnorum, antiphonarum, homiliarum, benedictionum etc.
in linguam bohemicam transtulit. /
Nea Marie Kuchařová textus sermonum in linguam bohemicam transtulit. /
Filip Srovnal librum istum præparare mandavit et laborem exprobrationibus
utilissimis comitabatur. /
Štěpán Němec librum istum diligentissime examinavit, errores multos
inveniens. /
Terezie Regnerová imaginem titulum libri ornantem pinxit.

Instrumenta adhibita.
LuaTeX, %http://www.luatex.org /
Gregorio, %http://home.gna.org/gregorio /
typi Junicode. %http://junicode.sourceforge.net

\begin{center}
Liber hic imprimis ad usum chori
\guillemotright Conventus Choralis\guillemotleft\
paratus est
et secundum eius consuetudines.
http://www.introitus.cz

\vfill

{\large Editio Sancti Wolfgangi \annusEditionis.}

\vfill

Series \guillemotright Conventus\guillemotleft, vol. VII.

\vfill

http://stiwolfgangi.xf.cz

\end{center}

\vfill

\end{document}
