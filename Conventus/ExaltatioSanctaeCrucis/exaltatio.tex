% LuaLaTeX

\documentclass[a4paper, twoside, 12pt]{article}

\usepackage[latin]{babel} 
\usepackage{ecclesiastic}

\usepackage[landscape, left=3cm, right=1.5cm, top=2cm, bottom=1cm]{geometry}

\usepackage{fontspec}
\setmainfont[Ligatures={Common, TeX}]{Junicode}

% shortcut for Junicode without ligatures (for the Czech texts)
\newfontfamily\nlfont[Ligatures={Common, TeX}]{Junicode}

\usepackage{multicol}
\usepackage{color}
\usepackage{lettrine}
\usepackage{fancyhdr}

% usual packages loading:
\usepackage{luatextra}
\usepackage{graphicx} % support the \includegraphics command and options
\usepackage{gregoriotex} % for gregorio score inclusion
\usepackage{gregoriosyms}
\usepackage{parcolumns}
\usepackage{multicol}

% Commands used to produce a typical "Conventus" booklet

\newenvironment{titulusOfficii}{\begin{center}}{\end{center}}
\newcommand{\dies}[1]{#1

}
\newcommand{\nomenFesti}[1]{\textbf{\Large #1}

}
\newcommand{\celebratio}[1]{#1

}

\newcommand{\hora}[1]{%
\vspace{0.5cm}{\large \textbf{#1}}

\fancyhead[LE]{\thepage\ / #1}
\fancyhead[RO]{#1 / \thepage}
\addcontentsline{toc}{subsection}{#1}
}

% larger unit than a hora
\newcommand{\divisio}[1]{%
\begin{center}
{\Large \textsc{#1}}
\end{center}
\fancyhead[CO,CE]{#1}
\addcontentsline{toc}{section}{#1}
}

% rubricated inline text
\newcommand{\rubricatum}[1]{\textit{#1}}

% standalone rubric
\newcommand{\rubrica}[1]{\vspace{3mm}\rubricatum{#1}}

\newcommand{\notitia}[1]{\textcolor{red}{#1}}

\newcommand{\scriptura}[1]{\hfill \small\textit{#1}}

\newcommand{\translatioCantus}[1]{\vspace{1mm}%
{\noindent\footnotesize \nlfont{#1}}}

% pruznejsi varianta nasledujiciho - umoznuje nastavit sirku sloupce
% s prekladem
\newcommand{\psalmusEtTranslatioB}[3]{
  \vspace{0.5cm}
  \begin{parcolumns}[colwidths={2=#3}, nofirstindent=true]{2}
    \colchunk{
      \input{#1}
    }

    \colchunk{
      \vspace{-0.5cm}
      {\footnotesize \nlfont
        a
        \input{#2}
      }
    }
  \end{parcolumns}
}

\newcommand{\psalmusEtTranslatio}[2]{
  \psalmusEtTranslatioB{#1}{#2}{8.5cm}
}

% volne misto nad antifonami, kam si zpevaci dokresli neumy
\newcommand{\hicSuntNeumae}{\vspace{0.5cm}}

% prepinani mista mezi notovymi osnovami: pro neumovane a neneumovane zpevy
\newcommand{\cantusCumNeumis}{
  \setgrefactor{17}
  \global\advance\grespaceabovelines by 5mm%
}
\newcommand{\cantusSineNeumas}{
  \setgrefactor{17}
  \global\advance\grespaceabovelines by -5mm%
}

% znaky k umisteni nad inicialu zpevu
\newcommand{\superInitialam}[1]{\gresetfirstlineaboveinitial{\small {\textbf{#1}}}{\small {\textbf{#1}}}}

% pars officii, i.e. "oratio", ...
\newcommand{\pars}[1]{\textbf{#1}}

\newenvironment{psalmus}{
  \setlength{\parindent}{0pt}
  \setlength{\parskip}{5pt}
}{
  \setlength{\parindent}{10pt}
  \setlength{\parskip}{10pt}
}

%%%% Prejmenovat na latinske:
\newcommand{\nadpisZalmu}[1]{
  \hspace{2cm}\textbf{#1}\vspace{2mm}%
  \nopagebreak%

}

% mode, score, translation
\newcommand{\antiphona}[3]{%
\hicSuntNeumae
\superInitialam{#1}
\includescore{#2}

#3
}

%%%% Translations of the proper chants

% HOURS ---

% Translated by Václav Ondráček

\newcommand{\translatioAntI}{\translatioCantus{
Hle v oblacích z nebes přijde Pán s velikou mocí, all.
}}
\newcommand{\translatioAntII}{\translatioCantus{
Město a pevnost naše je Sión, 
Spasitel jej opevní příkopem a zdí. 
Brány otevřete, neboť s námi je Bůh, all.
}}
\newcommand{\translatioAntIII}{\translatioCantus{
Hle Pán se zjeví, to není lež; 
jestliže prodlí, čekej jej, 
vždyť přijde a nebude meškat, all.
}}
\newcommand{\translatioAntIV}{\translatioCantus{
Hory a vrchy před Bohem chválu zpívat budou 
a lesní dříví rukama zatleská, 
neboť Pán a Vládce přijde na věky kralovat, all. all.
}}
\newcommand{\translatioAntV}{\translatioCantus{
Ejhle náš Pán s mocí přijde, by rozzářil oči sluhů svých, all.
}}

\newcommand{\translatioCapituli}{\translatioCantus{}}

\newcommand{\translatioRespVesp}{\translatioCantus{
Ukaž nám Pane své milosrdenství – a uděl nám spasení své.
}}

\newcommand{\translatioRespLaud}{\translatioCantus{
Přijď a vysvoboď nás, Hospodine, Bože silný. – Ukaž svou tvář a budeme spaseni.
}}

\newcommand{\translatioVersus}{\translatioCantus{
Hlas volajícího na poušti: Připravte cestu Páně
Urovnejte jeho cestu.
}}

\newcommand{\translatioAntMagnificatI}{\translatioCantus{
Přijď Pane, navštiv nás v míru, ať srdcem celistvým před tebou zajásáme.
}}

\newcommand{\translatioAntBenedictus}{\translatioCantus{
Když viděl Jan v řetězech Kristovy skutky, poslal svých učedníků dvé říci mu: Ty jsi ten, kdo přichází, nebo jiného čekat máme?
}}

\newcommand{\translatioAntMagnificatII}{\translatioCantus{
Ty jsi ten, kdo přichází, nebo jiného čekat máme? 
Rcete Janovi, co jste viděli: 
Světlo vzchází slepým, mrtví vstávají, chudým se hlásá radostná zvěst, aleluja.
}}

\newcommand{\translatioOrationis}{\translatioCantus{
Probuď Pane naše srdce k přípravě cesty tvého Jednorozeného, 
abychom díky jeho příchodu ti mohli sloužit s očištěnou myslí.
}}

\newcommand{\translatioHymnusVesp}{
Štědrý nebes stvořiteli
Věčný osvítiteli
Kriste vykupiteli
prosby vyslyš věrných milý.

Nad zánikem věku v smrti
smiloval ses, milý choti
a světu spásy lačnému
lék jsi přinesl vinnému.

Když se večer světa chýlí,
vyšels jako ženich milý
z domu matky, panny čestné,
z její komnaty milostné.

Mocí tvojí velmi silnou
kolena se k zemi ohnou
a nebe, země tvorové
se poddávají vůli tvé.

Jenž přicházíš věku soudce, 
prosíme tě světovládce,
uchovej nás v našem čase1 
před úkladnou ranou zhoubce.

Chválu, sílu, čest a slávu,
Bohu Otci, jeho Synu ,
Duchu také rady svaté 
vzdej na věky věků světe.
Amen.
}

\newcommand{\translatioHymnusLaud}{
Jasný hlas k nebi zní
temnoty všechny zahání
a trestá sen, ten prchá, jak
z výšin se Kristův zaskví zrak 

Probuď se mysli strnulá,
hluchotou povstaň zraněná,
nová se hvězda zažíhá,
škody od tebe odnímá.

Shůry přichází Beránek 
polehčit vinným jejich stesk,
v slzách o milost volejme,
smilování si žádejme.

Až pak podruhé zabuší
a svět jeho hrůza zkruší
za hříchy nás nepotrestal
a ochranu svou by nám dal.

Chválu, sílu, čest a slávu,
Bohu Otci, jeho Synu ,
Duchu také rady svaté 
vzdej na věky věků světe.
Amen.
}

% MASS ---

\newcommand{\translatioIntroitus}{\translatioCantus{}}

\newcommand{\translatioGraduale}{\translatioCantus{}}

\newcommand{\translatioAlleluia}{\translatioCantus{}}

\newcommand{\translatioOffertorium}{\translatioCantus{}}

\newcommand{\translatioCommunio}{\translatioCantus{}}

\newcommand{\antiphonaI}{
  \antiphona{VII c}{cantus/amon33/crux_laud_ant1.tex}{\trAntiphonaI}}
\newcommand{\antiphonaII}{
  \antiphona{III a}{cantus/amon33/crux_laud_ant2.tex}{\trAntiphonaII}}
\newcommand{\antiphonaIII}{
  \antiphona{I f}{cantus/amon33/crux_laud_ant3.tex}{\trAntiphonaIII}}
\newcommand{\antiphonaIV}{
  \antiphona{VII c}{cantus/amon33/crux_laud_ant4.tex}{\trAntiphonaIV}}
\newcommand{\antiphonaV}{
  \antiphona{II D}{cantus/amon33/crux_laud_ant5.tex}{\trAntiphonaV}}

\newcommand{\capitulumHocEnim}{
  \scriptura{Phil 2, 5-7.}
  
  \includescore{cantus/amon33/capitulum-HocEnimSentite}

  \trCapitulum
}

\newcommand{\anteOrationem}{
  \rubrica{Ante Orationem, cantatur a Superiore:}

  \pars{Supplicatio Litaniæ.}

  \includescore{\ccommunesAM/supplicatiolitaniae.tex}

  \pars{Oratio Dominica.}

  \includescore{\ccommunesAM/oratiodominica.tex}

  \rubrica{Deinde dicitur ab Hebdomadario:}

  \includescore{\ccommunesAM/dominusvobiscum-solemnis.tex}

  % original rubric from the Antiphonale Monasticum:
  %\rubrica{In choro monialium loco Dominus vobiscum dicitur:}

  \rubrica{Absente sacerdote vel diacono, loco \textnormal{Dóminus vobíscum}
  dicitur:}

  \includescore{\ccommunesAM/domineexaudi.tex}
}

\newcommand{\oratio}{
  \pars{Oratio.}

  \includescore{cantus/amon33/crux_oratio}

  \trOratio
}

\newcommand{\rubricaBenedicamus}{
  \rubrica{Repetito \textnormal{Dóminus vobíscum} in eodem tono ut antea,
  cantatur a cantore:}
}

\newcommand{\postBenedicamus}{
  \rubrica{Postea dicitur voce recta et paululum depressa:}

  \noindent ℣. Fidélium ánimæ per misericórdiam Dei requiéscant in pace.\\
  ℟. Amen.

  \noindent Pater noster. \rubricatum{totum secreto.}

  \rubrica{Deinde, si discedendum est a Choro:}

  \noindent ℣. Dóminus det nobis suam pacem.\\
  ℟. Et vitam ætérnam. Amen.

  \rubrica{Tunc dicitur Antiphona B.~M.~V. pro tempore, 
    cum \textnormal{℣.} et Oratione
    in Tono simplici, pg. \pageref{antiphonafinalis}. Deinde:}

  \noindent ℣. Divínum auxílium máneat semper nobíscum.\\
  ℟. Et cum frátribus nostris abséntibus. Amen.
}

% the two following prayers are from
% Breviarium monasticum, Romae sumptibus Josephi Salviucci 1831, p. lxxxiv.
% http://books.google.cz/books?id=GnFG6Z4Xuc8C&dq=breviarium%20monasticum&hl=cs&pg=PR1#v=onepage&q=breviarium%20monasticum&f=false
% but - this needs to be verified - it seems that exactly the same prayers
% were in the secular as well as monastic breviary, possibly since the
% post-Tridentine reform
\newcommand{\anteOfficiumOratio}{
\rubrica{Oratio dicenda ante inchoationem Divini Officii.}

\lettrine{A}{peri} Dómine os meum ad benedicéndum nomen sanctum tuum:
munda quoque cor meum ab ómnibus vanis, pervérsis, et aliénis
cogitatiónibus;
intelléctum illúmina, afféctum inflámma,
ut digne, atténte ac devóte hoc Offícium recitáre váleam,
et exaudíri mérear ante conspéctum Divínæ Majestátis tuæ.
Per Christum, Dominum nostrum.
℟. Amen.

Dómine, in unióne illíus Divínæ intentiónis,
qua ipse in terris laudes Deo persolvísti,
has tibi Horas \rubricatum{(vel \textnormal{hanc tibi Horam})} persólvo.
}

\newcommand{\postOfficiumOratio}{
\rubrica{
  Orationem sequentem devote post Officium recitantibus
  Leo Papa X. defectus, et culpas in eo persolvendo ex humana
  fragilitate contractas, indulsit, et dicitur flexis genibus.
}

\lettrine{S}{acrosánctæ} et indivíduæ Trinitáti,
crucifixi Domini nostri Jesu Christi humanitáti,
beatissimæ et gloriosíssimæ sempérque Virginis Maríæ
fecúndæ integritáti, 
et omnium Sanctórum universitáti,
sit sempitérna laus, honor, virtus et gloria
ab omni creatúra,
nobísque remíssio omnium peccatórum,
per infiníta sæcula sæculórum.
℟. Amen.

\noindent ℣. Beáta viscera Maríæ Virginis, quæ portavérunt
ætérni Patris Fílium.\\
℟. Et beáta ubera, quæ lactavérunt Christum Dominum.

Pater noster. Ave María.
}


\newcommand{\annusEditionis}{2013}
% directories
\newcommand{\ccommunesAM}{../../cantuscommunes/amon33}

\begin{document}

% GREGORIO general settings:
% staff size
\setgrefactor{15}
% space around the initial.
\setspaceafterinitial{2.2mm plus 0em minus 0em}
\setspacebeforeinitial{2.2mm plus 0em minus 0em}
% initial font. 
\def\greinitialformat#1{%
{\fontsize{40}{40}\selectfont #1}%
}

\pagestyle{empty}

% multicols setting
\setlength{\columnseprule}{1pt} % cara oddelujici sloupce
\setlength{\columnsep}{20pt} % prostor mezi sloupci

%%%% Titulni stranka
\begin{titulusOfficii}
\dies{Die 14. Septembris.} 
\nomenFesti{Exaltatio Sanctæ Crucis.}
\celebratio{Duplex majus.}
\end{titulusOfficii}

% graphic

\vspace{1.6cm}

\begin{center}
\includegraphics[height=8cm]{imagines/crux.jpg}
\end{center}

\vfill

\begin{center}
Ad usum et secundum consuetudines chori \guillemotright Conventus Choralis\guillemotleft.

Editio Sancti Wolfgangi \annusEditionis
\end{center}

\pagebreak

\renewcommand{\headrulewidth}{0pt} % no horiz. rule at the header
\fancyhf{}
\pagestyle{fancy}

\begin{multicols}{2}
\anteOfficiumOratio
\columnbreak

\postOfficiumOratio
\end{multicols}

% multicols setting
\setlength{\columnseprule}{0pt} % cara oddelujici sloupce

\pagebreak

\hora{In I. Vesperis.}

\includescore{\ccommunesAM/deusinadiutorium-alter}

\vfill

\pars{Psalmus 1.}

\antiphonaI

\scriptura{Psalmus 109}

\input{temporalia/ps109-initium-vii-c-auto}

\psalmusEtTranslatioB{temporalia/ps109-vii-c}{temporalia/ps109-boh}{11cm}

\vfill

\pagebreak

\pars{Psalmus 2.}

\antiphonaII

\scriptura{Psalmus 110}

\input{temporalia/ps110-initium-iii-a-auto}

\psalmusEtTranslatio{temporalia/ps110-iii-a}{temporalia/ps110-boh}

\vfill

\pagebreak

\pars{Psalmus 3.}

\antiphonaIII

\scriptura{Psalmus 111}

\input{temporalia/ps111-initium-i-f-auto}

\psalmusEtTranslatio{temporalia/ps111-i-f}{temporalia/ps111-boh}

\vfill

\pagebreak

\pars{Psalmus 4.}

\antiphonaV

\scriptura{Psalmus 112}

\input{temporalia/ps112-initium-ii-D-auto}

\psalmusEtTranslatio{temporalia/ps112-ii-d}{temporalia/ps112-boh}

\vfill

\pagebreak

\pars{Capitulum.}

\capitulumHocEnim

\vspace{1cm}

\superInitialam{VI}
\includescore{cantus/amon33/crux_vespi_resp.tex}

\trResponsoriumVesperae

\pars{Hymnus}

\rubrica{Stropha \textnormal{O Crux ave} dicitur flexis genibus.}

\superInitialam{I}
\includescore{temporalia/hymnus-VexillaRegis.tex}

\begin{multicols}{4}

\trHymnusVexillaRegis

\end{multicols}

\vspace{1cm}

\includescore{cantus/amon33/versiculus-HocSignum}

\trVersiculusVesperae

\vfill

\pagebreak

\pars{Canticum Beatæ Mariæ Virginis}

\antiphona{I D2}{cantus/amon33/crux_vespi_antmag.tex}{\trAntiphonaMagnificat}

\scriptura{Luc. 1, 46-55.}

\includescore{../../tonipsalmorum/arom12/magnificat-initium-i-D2}

\psalmusEtTranslatioB{temporalia/magnificat-isoll-d2}{temporalia/magnificat-boh}{11cm}

\vfill

\pagebreak

\anteOrationem

\pagebreak

\oratio

\rubricaBenedicamus

\includescore{\ccommunesAM/benedicamus-duplexmajus-vesperae}

\vspace{5mm}

\postBenedicamus

\vfill

\pagebreak

%%%%%%%%%%%%%%%%%%%%%%%%%%%%%%%%%%%%%%%%%%%%%%%%%%%%%%%%%%

\hora{Ad Laudes.}

\includescore{\ccommunesAM/deusinadiutorium-alter}

\vspace{2cm}

\pars{Psalmus 1.}

\scriptura{Psalmus 66}

\input{temporalia/ps66-initium-dir-auto}

\psalmusEtTranslatio{temporalia/ps66-dir}{temporalia/ps66-boh}

\vfill

\pagebreak

\pars{Psalmus 2.}

\antiphonaI

\scriptura{Psalmus 92}

\input{temporalia/ps92-initium-vii-c-auto}

\psalmusEtTranslatio{temporalia/ps92-vii-c}{temporalia/ps92-boh}

\vfill

\pagebreak

\pars{Psalmus 3.}

\antiphonaII

\scriptura{Psalmus 99}

\input{temporalia/ps99-initium-iii-a-auto}

\psalmusEtTranslatio{temporalia/ps99-iii-a}{temporalia/ps99-boh}

\vfill

\pagebreak

\pars{Psalmus 4.}

\antiphonaIII

\scriptura{Psalmus 62}

\input{temporalia/ps62-initium-i-f-auto}

\psalmusEtTranslatio{temporalia/ps62-i-f}{temporalia/ps62-boh}

\vfill

\pagebreak

\pars{Canticum Trium puerorum}

\antiphonaIV

\scriptura{Dan 3, 57-88 et 56.}

\input{temporalia/dan3-initium-vii-c-auto}

\psalmusEtTranslatioB{temporalia/dan3-vii-c}{temporalia/dan3-boh}{10cm}

\rubrica{Hic non dicitur \textnormal{Glória Patri.}}

\vfill

\pagebreak

\pars{Psalmus 5.}

\antiphonaV

\rubrica{Tres psalmi dicuntur sicut unus, sub una antiphona.
  \textnormal{Glória Patri} dicitur tantum post ultimum psalmum.}

\scriptura{Psalmus 148}

\input{temporalia/ps148-initium-ii-D-auto}

\psalmusEtTranslatioB{temporalia/ps148-ii-d}{temporalia/ps148-boh}{11cm}

\vspace{5mm}
\scriptura{Psalmus 149}

\psalmusEtTranslatioB{temporalia/ps149-ii-d}{temporalia/ps149-boh}{11cm}

\vspace{5mm}
\scriptura{Psalmus 150}

\psalmusEtTranslatioB{temporalia/ps150-ii-d}{temporalia/ps150-boh}{11cm}

% repeat the antiphon
\vspace{5mm}
\antiphona{}{cantus/amon33/crux_laud_ant5.tex}{}

\vfill

\pagebreak

\pars{Capitulum.}

\capitulumHocEnim

\vspace{1cm}

\superInitialam{VI}
\includescore{cantus/amon33/crux_laud_resp}

\trResponsoriumLaudes

\vspace{5mm}

\pars{Hymnus}

\superInitialam{I}
\includescore{temporalia/hymnus-LustrisSex}

\begin{multicols}{6}

\trHymnusLustraSex

\end{multicols}

\vspace{1cm}

\includescore{cantus/amon33/versiculus-AdoramusTe}

\trVersiculusLaudes

\vfill

\pagebreak

\pars{Canticum Zachariæ.}

\antiphona{I D2}{cantus/amon33/crux_laud_antben}{\trAntiphonaBenedictus}

\scriptura{Luc 1, 68-79.}

\includescore{temporalia/benedictus-initium-isoll-D2-auto}

\psalmusEtTranslatioB{temporalia/benedictus-isoll-d2}{temporalia/benedictus-boh}{9.5cm}

\vfill

\pagebreak

\anteOrationem

\pagebreak

\oratio

\rubricaBenedicamus

\includescore{\ccommunesAM/benedicamus-duplexmajus-laudes}

\postBenedicamus

\vfill

\pagebreak

\label{antiphonafinalis}

\pars{Antiphona finalis B. M. V. pro tempore per annum.}

\includescore{\ccommunesAM/ant-salveregina-simplex}

\noindent ℣. Ora pro nobis, sancta Dei \textbf{Gé}nitrix.\\
℟. Ut digni efficiámur promissiónibus \textbf{Chris}ti.

\includescore{\ccommunesAM/oratio-SalveRegina}

\pagebreak

\hora{Ad Missam.}

% I wasn't able to determine from the 1969 Graduale which ordinarium
% ought to be chosen, so I looked in the 1961 Graduale:
% in 1961 Exaltatio S. C. was a II. class feast
% and II. class feasts had "Cunctipotens Genitor Deus".

(Ordinarium IV Cunctipotens Genitor Deus; Credo I)

\cantusCumNeumis

\pars{Introitus.}
\scriptura{Cf. Gal. 6, 14; Psalmus 66}

\hicSuntNeumae

\superInitialam{IV}

\includescore{cantus/triplex79/crux_missa_introitus}

\trIntroitus


\cantusSineNeumas

\pars{Kyrie.}

\superInitialam{I}

\includescore{../../cantuscommunes/kyriale76/iv-kyrie}

\pagebreak


\pars{Gloria.}

\superInitialam{IV}

\includescore{../../cantuscommunes/kyriale76/iv-gloria}

\vfill

\pagebreak


\cantusCumNeumis

\pars{Graduale.}
\scriptura{Phil. 2, 8. \textnormal{℣.} 9}

\hicSuntNeumae

\superInitialam{V}

\includescore{cantus/triplex79/crux_missa_graduale}

\trGraduale


\pars{Alleluia.}

\superInitialam{VIII}

\includescore{cantus/triplex79/crux_missa_alleluia}

\trAlleluia

\vfill

\pagebreak


\setgrefactor{14} % to fit the Credo on one page

\cantusSineNeumas

\pars{Credo.}

\superInitialam{IV}

\includescore{../../cantuscommunes/kyriale76/credo-i}

\vfill

\pagebreak


\setgrefactor{15} % after Credo, back to the original size

\cantusCumNeumis

\pars{Offertorium.}

\hicSuntNeumae

\superInitialam{II}

\includescore{cantus/triplex79/crux_missa_offertorium}

\trOffertorium


\cantusSineNeumas

\vspace{0.5cm}

\pars{Sanctus.}

\superInitialam{VIII}

\includescore{../../cantuscommunes/kyriale76/iv-sanctus}


\vspace{0.5cm}

\pars{Agnus Dei.}

\superInitialam{VI}

\includescore{../../cantuscommunes/kyriale76/iv-agnusdei}

\vfill

\pagebreak

\cantusCumNeumis

\pars{Communio}

\hicSuntNeumae

\superInitialam{IV}

\includescore{cantus/triplex79/crux_missa_communio}

\trCommunio

\scriptura{Psalmus 17, 2-3a. 3bc. 4. 18. 38. 39. 41. 48. 49. 50}

\includescore{cantus/triplex79/initium-ps17-iv-comm}

%\input{temporalia/communio_versus}

% dirty hack: insert the psalm setting (generated by psalmpreprocessor)
% directly. Because there should be no Gloria Patri and I have no time
% to extend hiram to look for an option in the hiramfile ...
\begin{psalmus}
Deus meus, adjútor meus, et spe\emph{rábo in }\textbf{e}\-um;~*  
protéctor meus et cornu salútis meæ et sus\emph{céptor }\textbf{me}\-us.

Laudábilem in\emph{vocábo }\textbf{Dó}\-minum,~* 
et ab inimícis meis \emph{salvus }\textbf{e}\-ro.

Erípuit me de inimícis meis fortíssimis et ab \emph{his, qui o}\textbf{dé}\-runt me,~* 
quóniam confortá\emph{ti sunt }\textbf{su}\-per me.

Persequébar inimícos meos et compre\emph{hendébam }\textbf{il}\-los~* 
et non convertébar, do\emph{nec de}\textbf{fí}\-cerent.

Confringébam illos, nec \emph{póterant }\textbf{sta}\-re,~* 
cadébant subtus \emph{pedes }\textbf{me}\-os.

Et inimícos meos dedís\emph{ti mi\-hi }\textbf{dor}\-sum~* 
et odiéntes me dis\emph{per\-di}\textbf{dís}\-ti.

Deus, qui das vindíctas mihi et subdis \emph{pó\-pulos }\textbf{sub} me,~* 
liberator meus de inimicis meis \emph{i\-ra}\textbf{cún}\-dis.

Et ab insurgéntibus \emph{in me ex}\textbf{ál}\-tas me,~* 
a viro iní\emph{\-quo e}\textbf{rí}\-pis me.

Proptérea confitébor tibi in nati\emph{ó\-nibus, }\textbf{Dó}\-mine,~* 
et nómini tuo \emph{psal\-mum }\textbf{di}\-cam.
\end{psalmus}


\vfill

\pagebreak

% colophon
\pagestyle{empty}

\tableofcontents

\vspace{3cm}

Fontes. 
Textus et cantus officii divini secundum Antiphonale Monasticum, Solesmis~1934. /
Orationes ante et post officium ex: 
Breviarium monasticum, Romae 1831 (apud GoogleBooks). /
Textus et cantus missæ secundum Graduale Triplex, Solesmis~1979. /
Translationes psalmorum ex Hejčl Jan: Žaltář čili Kniha žalmů, Praha~1922. /
Translationes canticorum evangelicorum ex Hejčl Jan: 
Od Květné neděle do Bílé soboty. Příručka pro každého, kdo se chce účastniti
bohoslužby ve Svatém týdnu, Olomouc 1928. /
Translatio capituli sumpta est ex: 
Jeruzalémská bible, Praha-Kostelní Vydří 2009. /
Translatio hymni Vexilla Regis cum permissione auctoris sumpta est ex:
Koronthályová Markéta: Vexilla Regis. Výbor z latinské duchovní poesie,
Praha 2004. 

Collaborantes.
Textus latinos cantusque transcripsit et omnem laborem typographicum peregit
Jakub Pavlík. /
Proprios cantus festi in linguam bohemicam Václav Ondráček transtulit. /
Psalmos in lingua bohemica de libro supra dicto transcripsit
Barbora Maturová. /
Filip Srovnal laborem exprobrationibus utilissimis comitabatur,
librumque diligenter inspexit, plurimos errores inveniens atque corrigens. /
Štěpán Němec et Xavier Chancerelle consilio adiuti sunt et errores
nonnullos correxerunt. /
Cantus proprii et ordinarii missæ 
iam \guillemotright typis impressos\guillemotleft\mbox{ }
apud GregoBase
http://gregobase.selapa.net invenimus et grati utimur. Hic laborem typographicum
peregit Andrew Hinkley, nos autem correctiones minores fecimus
iuxta Graduale Triplex. /
Imaginem, quæ paginam tituli ornat, Tereza Regnerová pinxit.

Instrumenta adhibita.
LuaTeX, %http://www.luatex.org / 
Gregorio, %http://home.gna.org/gregorio /
typi Junicode. %http://junicode.sourceforge.net

\begin{center}
Liber hic ad usum chori 
\guillemotright Conventus Choralis\guillemotleft\ 
paratus est
et secundum eius consuetudines -
specialiter pro die nuptiarum Zuzana Malá et Jan Horák.

\vspace{1cm}

{\large Editio Sancti Wolfgangi \annusEditionis .}

\vspace{2mm}

Series \guillemotright Conventus\guillemotleft, vol. III.

\vspace{1cm}

http://stiwolfgangi.xf.cz
\vfill

\today

\end{center}

\end{document}
