% LuaLaTeX



\documentclass[a4paper, twoside, 12pt]{article}
\usepackage[latin]{babel} 
\usepackage[left=3cm, right=1.5cm, top=2cm, bottom=2cm]{geometry} % okraje stranky

\usepackage{fontspec}

\usepackage{color}

\usepackage{luatextra}
\usepackage{graphicx} % support the \includegraphics command and options
\usepackage{gregoriotex} % for gregorio score inclusion

% Here we set the space around the initial.
% Please report to http://home.gna.org/gregorio/gregoriotex/details for more details and options
\setspaceafterinitial{2.2mm plus 0em minus 0em}
\setspacebeforeinitial{2.2mm plus 0em minus 0em}

% Here we set the initial font. Change 43 if you want a bigger initial.
\def\greinitialformat#1{%
{\fontsize{43}{43}\selectfont #1}%
}

% ===================================================================

% Definitions of common commands and environments
% used in the Proprium Antiphonalis Provinciae Pragensis

\newcommand{\modusinfo}[1]{\gresetfirstlineaboveinitial{\small #1}{\small #1}}

\newcommand{\fonsinfo}[1]{\commentary{{\footnotesize #1}}}

\newcommand{\rubrum}[1]{\textcolor{red}{#1}}

\newcommand{\rubrica}[1]{\vspace{3mm} \rubrum{\small #1}}

\newcommand{\notitiaEditorialis}[1]{
  \rubrum{\small \textsc{Notitia editorialis:} #1}
}

% pieces of the title of a feast
\newcommand{\diesFesti}[1]{{\small #1} \vspace{2mm}}
\newcommand{\nomenFesti}[1]{\textbf{\Large #1}\vspace{3mm}
}
\newcommand{\descriptioFesti}[1]{\rubrum{\textbf{#1}}
}
\newcommand{\dignitasFesti}[1]{\rubrum{\small #1}
}

\newenvironment{caputFesti}{\begin{center}}{\end{center}}

% headings of various parts
\newcommand{\hora}[1]{\vspace{1cm}\noindent\textbf{#1}\vspace{5mm}
}
\newcommand{\parsHorae}[1]{\rubrum{#1}\vspace{2mm}}

% often repeated content
\newcommand{\postUltimumResponsoriumNocturniRubrica}[1]{
  \rubrica{Sequitur Gloria Patri in modo ultimi responsorii. Postea repetitur
    altera pars responsi (\textit{#1}).}
}

\newcommand{\EditioSanctiWolfgangiInFine}{
  \vfill

  \begin{center}
    Editio Sancti Wolfgangi \annusPublicationis.
  \end{center}
}


\begin{document}

% Here we set the space around the initial.
% Please report to http://home.gna.org/gregorio/gregoriotex/details for more details and options
\setspaceafterinitial{2.2mm plus 0em minus 0em}
\setspacebeforeinitial{2.2mm plus 0em minus 0em}

% Here we set the initial font. Change 43 if you want a bigger initial.
\def\greinitialformat#1{%
{\fontsize{43}{43}\selectfont #1}%
}


\begin{caputFesti}
\diesFesti{Die 15 Junii}

\nomenFesti{S. Viti}

\descriptioFesti{Martyris, Patroni Regni minus princ. et Titul. Eccl. Metrop.}

\dignitasFesti{Duplex II classis
(In archidioec. Pragen. duplex I classis cum Octava com.)}
\end{caputFesti}

\hora{In I Nocturno}

\notitiaEditorialis{
In manuscripto, de quo cantus sequentes sumpti sunt (KA2),
in principio folia desunt, quae cantus ad vesperas I et ad matutinum
usque ad responsorium II primi nocturni continebant.
Propterea cantus isti omittuntur usque ad diem inventionis manuscripti alii,
quod eos continet.
}

\includescore{cantus/vitus-noct1-r2.tex}

\pagebreak

\includescore{cantus/vitus-noct1-r3.tex}

\postUltimumResponsoriumNocturniRubrica{Sanetur}

\hora{In II Nocturno}

\parsHorae{Antiphonae.}

\modusinfo{VIII.G}
\includescore{cantus/vitus-noct2-a1.tex}

\modusinfo{I.g}
\includescore{cantus/vitus-noct2-a2.tex}

\modusinfo{VII.a}
\includescore{cantus/vitus-noct2-a3.tex}


\parsHorae{Responsoria.}

\includescore{cantus/vitus-noct2-r1.tex}

\includescore{cantus/vitus-noct2-r2.tex}

\modusinfo{IV}
\includescore{cantus/vitus-noct2-r3.tex}

\postUltimumResponsoriumNocturniRubrica{Accelera}

\hora{In III Nocturno}

\parsHorae{Antiphonae.}

\modusinfo{VIII.G}
\includescore{cantus/vitus-noct3-a1.tex}

\modusinfo{II.D}
\includescore{cantus/vitus-noct3-a2.tex}

\modusinfo{IV.E}
\includescore{cantus/vitus-noct3-a3.tex}

\pagebreak

\parsHorae{Responsoria.}

\includescore{cantus/vitus-noct3-r1.tex}

\modusinfo{II}
\includescore{cantus/vitus-noct3-r2.tex}

\postUltimumResponsoriumNocturniRubrica{Et perduc}

% Medieval manuscript has also a 9nt responsory, 
% "Dum complesset sanctus Vitus orationem";
% but the after-Trent office of nine readings has only eight responsories;
% after the ninth reading Te Deum is chanted.

\pagebreak

\hora{Ad laudes et per horas.}

\modusinfo{VIII.G}
\includescore{cantus/vitus-laudes-a1.tex}

\modusinfo{VI*.C}
\includescore{cantus/vitus-laudes-a2.tex}

\modusinfo{VIII.G}
\includescore{cantus/vitus-laudes-a3.tex}

\modusinfo{VI.F}
\includescore{cantus/vitus-laudes-a4.tex}

\modusinfo{I.g}
\includescore{cantus/vitus-laudes-a5.tex}

\notitiaEditorialis{Hymnus festi in manuscripto KA2 deest.}

Ad Benedictus antiphona.

\modusinfo{VIII.G}
\includescore{cantus/vitus-laudes-aben.tex}

\hora{In II Vesperis.}

\rubrica{Antiphonae de Laudibus.}

Ad Magnificat antiphona.

\modusinfo{VI*.C}
\includescore{cantus/vitus-vesp2-amag.tex}

\vfill

\pagebreak

Fontes:

\vspace{1cm}
\textbf{KA2} - Kolínský antifonář - II. díl; 
Knihovna národního muzea v Praze, Praha, Česká republika, sig. XII A 21. 
(www.manuscriptorium.com)

\vspace{1cm}
\textsc{Officia propria provinciae Pragensis}
a S. Sede approbata et indulta,
auctoritate reverendissimi ac celsissimi domini, domini Francisci Kordač,
archiepiscopi Pragensis atque ill.morum et r.morum episcoporum
provinciae Pragensis suffraganeorum denuo edita,
\textit{pars aestiva}. Sumptibus et typis Friderici Pustet, Ratisbonae 1928.

\vfill

\begin{center}
Editio Sancti Wolfgangi 2012.
\end{center}

\end{document}
